\documentclass[10pt,a4paper]{article}
\usepackage[utf8]{inputenc}
\usepackage[english]{babel}
\usepackage[T1]{fontenc}
\usepackage{hyperref}
\usepackage[english]{babel}
\usepackage{amssymb,amsthm,amsmath,amsfonts}
\usepackage{units}
\usepackage{lmodern}
\usepackage{graphicx}
\bibliographystyle{plain}
\usepackage{epsfig}
\usepackage{epstopdf}
\usepackage{makeidx}
\usepackage{gloss}
\usepackage{xcolor}
\usepackage{algorithmic}
\usepackage{algorithm}
\def\glossname{Glossaire}
\usepackage{tikz}
\usepackage{fancyhdr}
\pagestyle{fancy}

\theoremstyle{plain}
\newtheorem{thm}{Theoreme}
\theoremstyle{definition} 
\newtheorem{lem}[thm]{Lemma}
\theoremstyle{definition} 
\newtheorem{cor}[thm]{Corollary}
\theoremstyle{definition} 
\newtheorem{prop}[thm]{Proposition}
\theoremstyle{definition} 
\newtheorem{defi}[thm]{Definition}
\theoremstyle{remark} 
\newtheorem{rem}[thm]{Remark}
\theoremstyle{remark} 
\newtheorem{exe}[thm]{Example}
\author{Luca DeFeo,Cyril Hugounenq,Jérôme Plût, Eric Schost}

\begin{document}
\section{Reminder on Couveignes's algorithm}


\section{Reminder on isogeny volcano}
\begin{defi}
For $E$ an ordinary elliptic curve defined over $\mathbb{F}_q$ we denote  his endomorphism ring (associated up to isomorphism) by $\mathcal{O}$. $\mathcal{O}$ is an order include in a quadratic imaginary field denoted $K$, we denote $\mathcal{O}_K$ the algebraic integers of $K$.
\end{defi}

\begin{lem}[Kohel 1996]
Let $E$ and $E'$ two elliptic curves defined over $\mathbb{F}_q$, $\phi :E \rightarrow E'$ an $\ell$-isogeny, with $\ell \neq p$. Then
\begin{enumerate}
\item either $\ell|[\mathcal{O} : \mathcal{O}']$ we say then that $\phi$ is a descending isogeny,
\item either $\ell|[\mathcal{O}':\mathcal{O}]$ we say then that $\phi$ is an ascending isogeny,
\item either $\mathcal{O}=\mathcal{O}'$ we say then that $\phi$ is an horizontal isogeny.
\end{enumerate}
\end{lem}

\begin{proof}
See Kohel cité les deux résultats en un
\end{proof}

\begin{prop}
Let $E$ and $E'$ two elliptic curves defined over $\mathbb{F}_q$, $\phi :E \rightarrow E'$ an $\ell$-isogeny, with $\ell \neq p$. Then we can denote $[1,\omega]$ a $\mathbb{Z}$ basis of $\mathcal{O}$. For $f=[\mathcal{O} : \mathcal{O}']$ we can denote $[1,f\omega]$ a $\mathbb{Z}$ basis of $\mathcal{O'}$.
\end{prop}

\begin{defi}
Let $E$, $E'$ two elliptics curve such that there exists an isogeny $\phi: E \rightarrow E'$. If $E$ is separable then $deg(\phi)=ker(\phi)$. For $\ell$ an integer we call an $\ell$ isogeny, an isogeny of degree $\ell$
\end{defi}

\begin{defi}
A volcano of $\ell$ isogeny is a graph of $\ell$ isogenies of degree $\ell+1$.
\end{defi}
\subsection{Structure of a volcano}


\section{Computing a canonical basis}

\section{Interpolating the two basis}



\end{document}
