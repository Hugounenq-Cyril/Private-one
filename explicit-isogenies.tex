\documentclass{lms}
\usepackage[utf8]{inputenc}
\usepackage[american]{babel}
\usepackage[T1]{fontenc}
\usepackage{amssymb,amsmath,amsfonts}
\usepackage{algorithmic}
\usepackage{algorithm}
\usepackage{tikz}
\usepackage{nicefrac}
\usepackage{hyperref}
\usepackage{unicode}
\usepackage{stmaryrd}
 
\newcommand{\todo}[1]{{\color{red}TODO: #1}}
\newcommand{\todoblue}[1]{{\color{blue}#1}}
\def\snote#1{}


%\theoremstyle{plain}
\newtheorem{thm}{Theorem}[section]
\newtheorem{lem}[thm]{Lemma}
\newtheorem{cor}[thm]{Corollary}
\newtheorem{prop}[thm]{Proposition}

\newnumbered{defi}[thm]{Definition}
\newnumbered{rem}{Remark}
\newnumbered{exe}{Example}
\newnumbered{prob}{Problem}

\makeatletter
\newenvironment{localblock}[1]{\@exmplstar{\relax}{#1}}{\@endexample}

\def\mat#1{\begin{pmatrix}#1\end{pmatrix}}
\def\smat#1{{\def\arraystretch{.7}\mat{#1}}}
\def\pa#1{\left(#1\right)}
\def\acco#1{\left\{#1\right\}}
\def\bcro#1{\left\llbracket#1\right\rrbracket}
\def\chev#1{\left\langle#1\right\rangle}
% pour les coûts des opérations M, F etc. (?)
\def\cout#1{\mathsf{#1}}
% let's decide on one!
\let\leq\leqslant \let\geq\geqslant
\def\sfdiv{\mathsf{divide}}

\newcommand{\M}{\mathsf{M}}
\newcommand{\Z}{\mathbb{Z}}
\newcommand{\F}{\mathbb{F}}
\newcommand{\Q}{\mathbb{Q}}
\newcommand{\C}{\mathbb{C}}
\newcommand{\tildO}{\tilde{O}}
\newcommand{\MM}{\cout{M}}
\newcommand{\FF}{\cout{F}}
\newcommand{\RR}{\cout{R}}
\DeclareMathOperator{\loglog}{loglog}

\def\algorithmicrequire{\textbf{Input:}}
\def\algorithmicensure{\textbf{Output:}}

\makeatletter
\def\full@line{8pt}
\def\doublefull@line{12pt}
% \def\@nprf{% nprf environment modified OT 12/11/02
%   \list{}{\topsep 8pt \leftmargin 0pt
%   \itemindent\parindent \labelsep .5em 
%   \listparindent\parindent
%   \settowidth\labelwidth{{\normalfont\rmfamily(iii)}}}%
%   \item{\normalfont\itshape\proofname.}\advance\itemindent\labelsep%
%   \advance\itemindent\labelwidth%
%   \hskip 1em\normalfont\rmfamily\ignorespaces}
\makeatother


% Colors after XKCD color name survey
\definecolor{purple}{rgb}{.49,.11,.61}
\definecolor{green}{rgb}{.08,.69,.10}
\definecolor{blue}{rgb}{.01,.26,.87}
\definecolor{pink}{rgb}{1,.5,.75}
\definecolor{brown}{rgb}{.39,.21,.00}
\definecolor{red}{rgb}{.89,0,0}
\definecolor{lightblue}{rgb}{.58,.81,.98} 
\definecolor{teal}{rgb}{0,.57,.52}
\definecolor{orange}{rgb}{.97,.45,.02}
\definecolor{lightgreen}{rgb}{.53,.99,.01}
\definecolor{magenta}{rgb}{.76,0,.47}
\hypersetup{colorlinks,linkcolor={red!70!black},
  citecolor={green!70!black},urlcolor={blue!70!black}}

\title[Explicit isogenies in any characteristic]{Explicit isogenies in quadratic time in any characteristic}
\author[L. De Feo, C. Hugounenq, J. Plût, É. Schost]%
  {Luca De Feo, Cyril Hugounenq, Jérôme Plût, and Éric Schost}

\classno{11Y40 (primary), 11G20, 14H52 (secondary)}

\extraline{This work was partially supported by the
  \href{http://www.digiteo.fr/}{DIGITEO} grant 2013-0531D (ARGC).}

\begin{document}
\maketitle

\begin{abstract}
\todo{
Given two elliptic curves and the knowledge of the existence of
an isogeny between them, of known degree~$r$,
we compute this isogeny by using
the structure of the $ℓ$-torsion of the curves,
where $ℓ$~is a prime different from the characteristic~$p$ of the base field.
\todoblue{say that we are working over a finite field}
Our algorithm has a complexity of~$\tildO(r^2)$ operations in the base field.
\todoblue{dependency in $\log(q)$ missing}
This is to be compared with Couveignes' algorithm,
which uses the $p$-torsion, and has a complexity of~$\tildO(r^2 p^{O(1)})$.
Our algorithm is therefore an interesting alternative
for the medium- and large-characteristic cases.
}
%% The problem we will consider here is the computation of an isogeny between two elliptics curves with the knowledge of the domain and the codomain of the isogeny and it's degree $r$. Couveignes's algorithm is an algorithm which solves this problem in $O(r^2)$ operations using the $p$-torsion. We want to extend the method used by Couveignes  We try to adapt his method here to the case of the $2$ torsion and more generally to the $\ell$ torsion, thus we propose an alternative for medium characteristic with this algorithm.
\end{abstract}

% \section*{Proposed notation}

% This section is for internal reference only: erase after the paper has
% stabilized.

% \begin{itemize}
% \item $\mathbb{F}_q$ is the field we are working on
% \item $\ell$ is for the $\ell$ torsion we are working on
% \item $r$ is the degree of the isogeny we want to compute
% \item $k$ is the integer such that $\ell^{2k}>4r+1$
% \item we thus work with a tower which has for top level $F_{q^{\ell^k}}$
% \item $E$ is for ordinary elliptic curves defined over the finite field $\mathbb{F}_q$
% \item $\mathcal{O}$ (resp. $\mathcal{O}_x$) is the notation for the endomorphism ring associated (up to isomorphism) to $E$ (resp. $E_x$)
% \item $K$ is the notation for the imaginary quadratic field in which $\mathcal{O}$ is defined
% \item $d_K$ is the negative integer such that $K=\mathbb{Z}[d_K]$  
% \end{itemize}

%%%%%%%%%%%%%%%

\section{Introduction}
\label{sec:introduction}

Isogenies are non-zero morphisms of elliptic curves, that is,
non-constant rational maps preserving the point at infinity. They are
also algebraic group morphisms. Isogeny computations play a central
role in the algorithmic theory of elliptic curves. They are notably
used to speed up Schoof's point counting
algorithm~\cite{schoof85,atkin88,elkies92,schoof95,elkies98}. They are
also widely applied in cryptography, where they are used to speed up
point multiplication~\cite{gallant+lambert+vanstone01,birkner+sica11},
to perform cryptanalysis~\cite{mauer+menezes+teske01}, and to
construct new
cryptosystems~\cite{teske06,charles+lauter+goren09,Stol,defeo+jao+plut12,jao+soukharev2014-signatures}.

The \emph{degree} of an isogeny is its degree as a rational map. If an
isogeny has degree $r$, we call it an $r$-isogeny, and we say that two
elliptic curves are $r$-isogenous if there exists an $r$-isogeny
relating them. Accordingly, we say that two field elements $j$ and
$j'$ are $r$-isogenous if there exist $r$-isogenous elliptic curves
$E$ and $E'$ such that $j(E)=j$ and $j(E')=j'$. The
\emph{explicit isogeny} problem has many incarnations. In this paper,
we are interested in the variant defined below.

\begin{localblock}{Explicit isogeny problem} \label{prob:isogeny-problem}
  Given two $j$-invariants $j$ and $j'$, and a positive integer
  $r$, determine if they are $r$-isogenous. In that case, compute
  curves $E$, $E'$ with $j(E)=j$ and $j(E')=j'$, and the
  kernel of an $r$-isogeny $ψ:E\to E'$.
\end{localblock}

Once the kernel of the isogeny is computed, the rational maps
associated to it can be computed in optimal time using Velu's
formulas~\cite{velu71}.

This paper focuses on the explicit isogeny problem for \emph{ordinary}
elliptic curves over finite fields. A famous theorem by Tate states
that two curves are isogenous over a finite field if and only if they
have the same cardinality over that field. The explicit isogeny
problem stated here appears naturally in the Schoof-Elkies-Atkin point
counting algorithm (SEA). There, $E$ is the curve of which we want to
compute the cardinality, and $E'$ is an $r$-isogenous curve, with $r$
a prime of size approximately $\log(\#E)$. For this reason, the explicit
isogeny problem is customarily solved without prior knowledge of the
cardinality of $E$. We will abide by this convention here.

A good measure of the computational difficulty of the problem is given
by the isogeny degree~$r$. Indeed the output \todo{rational functions? kernel?} is represented by
$O(r)$ base field elements, hence an asymptotically optimal algorithm
would solve the problem using $O(r)$ field operations. Many algorithms
have been suggested over the years to solve the explicit isogeny
problem. Early algorithms were due to Atkin~\cite{atkin91} and
Charlap, Coley and
Robbins~\cite{charlap1991enumeration}. Elkies'~\cite{elkies92,elkies98,Bostan}
was the first algorithm targeted to finite fields (of large enough
characteristic). Assuming $r$ is prime, its complexity is dominated by
the computation of the modular polynomial $\Phi_r$, which is an object
of (binary) size $O(r^3\log(r))$. Later Bröker, Lauter and
Sutherland~\cite{sutherland10:modpol} optimized the modular polynomial
computation in the context of the SEA
algorithm. Finally Lercier and Sirvent~\cite{lercier+sirvent08,1602.00244}
generalized Elkies' algorithm to work in any characteristic. Despite
these advances, the overall cost of Elkies' algorithm and its
variants is still at least cubic in $r$.

Another line of work to solve the explicit isogeny problem was
initiated by Couveignes~\cite{couveignes94,couveignes96,couveignes00},
and later improved by De Feo and Schost~\cite{df10,df+schost12}. These
algorithms use an interpolation approach combined with ad-hoc
constructions for towers of finite fields of characteristic $p$. Their
complexity is quasi-quadratic in $r$, but exponential in $\log(p)$,
hence they are only practical for very small characteristic.

In this paper we present a variant of Couveignes' algorithm with
complexity polynomial in $\log(p)$ and quasi-quadratic in $r$. Together
with the Lercier-Sirvent algorithm, they are the only polynomial-time
isogeny computation algorithms working in any characteristic, hence
they are especially relevant for counting points in \emph{medium}
characteristic (i.e., counting points over $\F_{p^n}$, when
$n\gg p/\log (p)$).

Note that, although Couveignes-type algorithms do not make use of the
modular polynomial $\Phi_r$, its computation is still necessary in the
context of the SEA algorithm. Thus our new algorithm does not improve
the \todo{overall complexity} state of the art on point counting. It
gives, however, an effective algorithm for solving the explicit
isogeny problem, with potential applications in other contexts, e.g.,
cryptography.

\subsection{Notation}

Throughout this paper: $r$~is a positive integer, $p$~an odd prime,
$q$~a power of $p$, and $\mathbb F_q$ is the finite field with
$q$~elements. $E$ ~is an ordinary elliptic curve over~$\mathbb F_q$,
its group of $n$-torsion points is denoted by~$E[n]$, its
$q$-Frobenius endomorphism by~$π$.  The endomorphism ring of $E$ is
denoted by~$\mathcal O$, with~$K = \mathcal O ⊗ ℚ$ the corresponding
number field, $\mathcal O_K$ is its maximal order, and $d_K$~the
discriminant of~$\mathcal O_K$.
For a prime~$ℓ$ different from~$p$ and not dividing~$r$,
we denote by~$E[ℓ^k]$ the group of $ℓ^k$-torsion points of~$E$,
$E[ℓ^{∞}] = \varinjlim E[ℓ^k]$ the reunion of all $E[ℓ^k]$,
and $T_ℓ(E) = \varprojlim E[ℓ^k]$ the $ℓ$-adic Tate module~\cite[III.7]{Sil},
which is free of rank~two over~$ℤ_ℓ$.
The factorization of the characteristic polynomial of~$π$
in~$ℤ_ℓ$ is determined by the Kronecker symbol~$(d_K/ℓ)$.
If $(d_K/ℓ) = +1$ then we also define $λ,μ$ as
the eigenvalues of~$π$ in~$ℤ_ℓ$ and write~$h = v_ℓ(λ - μ)$,
where $v_ℓ$ is the $ℓ$-adic valuation.

We measure all computational complexities in terms of operations in
$\mathbb{F}_q$; the binary costs associated to the algorithms
presented next are negligible compared to the algebraic costs, and
will be ignored. We use the Landau notation $O(\ )$ to express
asymptotic complexities, and the notation $\tildO(\ )$ to neglect
(poly)logarithmic factors.  We let $\MM(n)$ be a function such that
polynomials in $\F_q[x]$ of degree less than $n$ can be multiplied
using $\MM(n)$ operations in $\F_q$, under the assumptions
of~\cite[Chapter~8.3]{vzGG}. Using FFT multiplication, one can take
$\MM(n)∈ O(n\log(n)\loglog(n))$.

\subsection{Couveignes' algorithm}
\label{sec:couv-algor}

Couveignes' isogeny algorithm takes as input two \emph{ordinary}
elliptic curves $E$ and $E'$ defined over $\F_q$, and given in Weierstrass form
\todo{check} \todo{Couveignes ne précise pas dans son article, dans l'implantation on se sert des formes de Weierstrass} and a positive integer $r$ not
divisible by $p$, and returns, if it exists, the kernel of an
$r$-isogeny $ψ:E\to E'$.\todo{Aren't we taking only the $j$-invariants as input?} \todo{ Couveignes dans son article prend des courbes, donc autant faire pareil} It is based on the observation that the
isogeny $ψ$ must put $E[p^k]$ in bijection with $E'[p^k]$, in a way
that is compatible with their structure of cyclic groups. 
It proceeds in three steps:
\begin{enumerate}
\item\label{alg:orig-couveignes:tower} Compute generators $P,P'$ of
  $E[p^k]$ and $E'[p^k]$ respectively, for $k$ large enough;
\item\label{alg:orig-couveignes:interp} Compute the interpolation
  polynomial $L$ sending $x(P)$ to $x(P')$, and the abscissas of
  their scalar multiples accordingly;
\item\label{alg:orig-couveignes:rational} Deduce a rational fraction $g(x)/h(x)$
  that coincides with $L$ at all points of $E[p^k]$, and verify that
  it defines the first component of an isogeny of degree $r$. If it
  does, return it \todo{shouldn't we return $h$, to be consistent?}, 
otherwise replace $P'$ with a scalar multiple of
  itself and go back to Step~\ref{alg:orig-couveignes:interp}.
\end{enumerate}

For this algorithm to succeed, enough interpolation points are
required. Given that the isogeny $ψ$ is defined by a rational fraction
of degree $(r,r-1)$ \todo{not clear, there should be 2 components; explain better, or give Velu's 
formulas somewhere}, it is necessary that $p^k∈\Omega(r)$. However, most of
the time, those points are not going to be defined in the base field
$\F_q$, thus Couveignes' algorithm must be based on efficient
algorithms to construct and compute in towers of extensions of finite
fields. Indeed, Couveignes and his successors go at great length in
studying the arithmetic of \emph{Artin-Schreier
  towers}~\cite{couveignes00,df+schost12}, and the adaptation of the
fast interpolation algorithm to that setting~\cite{df10}.  Using these
highly specialized constructions,
Steps~\ref{alg:orig-couveignes:tower}
and~\ref{alg:orig-couveignes:interp} are both executed in quasi-linear
\todo{? in what?}
time $\tildO(p^{k+O(1)})=\tildO(rp^{O(1)})$. However the last step
only succeeds for one pair of torsion points $P,P'$, in general, thus
$O(r)$ trials are expected on average.

Hence, the overall complexity of Couveignes' algorithm is
$\tildO(r^2p^{O(1)})$, i.e., quadratic in $r^2$, but exponential in
$\log(p)$. Although the exponent of $p$ is relatively small,
Couveignes algorithm quickly becomes impractical as $p$ grows.

\subsection{Our contributions}

In this paper we
introduce a variant of Couveignes' algorithm with the same quadratic
complexity in $r$, and \textbf{no exponential dependency in $\log(p)$}.

The bottom line of our algorithm is elementary: replace $E[p^k]$ in
the algorithm with $E[ℓ^k]$, for some small prime $ℓ$. However a
naive application of this idea fails to yield a quadratic-time
algorithm. Indeed, in the worst case one has $ℓ^{2k}∈\Theta(r)$, with
$E[ℓ^k]≃(ℤ/ℓ^kℤ)^2$. Hence, two generators $P,Q$ of $E[ℓ^k]$ must
be mapped onto two generators of $E'[ℓ^k]$. This can be done in
$O(ℓ^{4k})$ possible ways, with a best possible cost of $O(\ell^{2k})$
per trial,
thus yielding an
algorithm of complexity $O(ℓ^{6k})=O(r³)$ at best.

To avoid this pitfall, we carefully study
in Section~\ref{sec:isogeny-volcanoes} the structure of
$E[ℓ^k]$, and its relationship with the Frobenius endomorphism $π$.
With that knowledge, we can put some restrictions on the generators $P,Q$,
as explained in Section~\ref{sec:acti-frob-endm},
thus limiting the number of trials to $O(ℓ^{2k})$.
In Section~\ref{sec:interpolation}
we present an interpolation algorithm adapted to the setting of
$ℓ$-adic towers, and in Section~\ref{sec:complete-algorithm} we put
all steps together and analyze the full algorithm. Finally in
Section~\ref{sec:implem} we discuss our implementation and the
performance of the algorithm.


%%%%%%%%%%%%%%%

\subsection{Towers of finite fields}
\label{sub:towers}

The algorithms presented next operate on elements defined in finite
extensions of $\F_q$. Specifically, we will work in a \emph{tower} of finite
fields $\F_q=F₀⊂F₁⊂\cdots⊂F_n$, with $\ell$ dividing $\#F_1-1$, $d_1=[F₁:F₀]$
dividing $ℓ-1$, and $[F_{i+1}:F_i]=ℓ$ for any $i>0$. For $ℓ=2$,
we build upon the work of Doliskani and Schost~\cite{DoSc12}, whereas for
general $ℓ$ we use towers of Kummer extensions in a way similar
to~\cite[\S2]{DeDoSc13}.  Both constructions represent elements of
$F_i$ as univariate polynomials with coefficients in $\F_q$, thus
basic arithmetic operations can be performed with classic modular
polynomial arithmetic. While constructing the tower, we also enforce
special relations between the generators of each level, so that moving
elements up and down the tower, and testing membership, can be done at
negligible cost.

We briefly sketch the construction for odd $\ell$. We first look for a
primitive polynomial $P_1∈\F_q[x]$ of degree equal to $[F₁:F₀]$. There
are many probabilistic algorithms to compute $P_1$ in time polynomial
in $\ell$ and $\log(q)$; since their cost does not depend on the height
$n$ of the tower, we neglect it. Then, the image $x₁$ of $x$ in
$F_1=\F_q[x]/P₁(x)$ is an element of multiplicative order $\#F₁-1$,
and in particular it is not a $ℓ$-th power. Hence for any $i>1$ we
define $F_i$ as $\F_q[x]/P_1\bigl(x^{ℓ^{i-1}}\bigr)$, the computation
of the polynomials $P_1\bigl(x^{ℓ^{i-1}}\bigr)$ incurring no algebraic
cost. Using this representation, elements of $F_i$ can be expressed as
elements of a higher level $F_{i+j}$, and reciprocally, by a simple
rearrangement of the coefficients. Another fundamental operation can
be done much more efficiently than in generic finite fields, as the
following generalization of~\cite[\S2.3]{DoSc12} shows.

\begin{lem}\label{lemma:frob-ell}
  Let $F_0⊂\cdots⊂F_n$ be a Kummer tower as defined above, and let
  $a∈F_i$ for some $0≤i≤n$. For any integer $j$, we can compute the
  $(\#F_j)$-th power of $a$ using $O(ℓ^i\MM(ℓ) + \MM(ℓ^i))$ operations
  in $\F_q$, after a precomputation independent of $a$ that uses
  $O(ℓ\MM(ℓ)\log(q))$ operations in $\F_q$.
\end{lem}
\begin{proof}
  Without loss of generality, we can assume that $j<i$; otherwise, the
  output is simply $a$ itself. Let $τ=\#F_i$ and $σ=\#F_j$. By
  assumption, $a$ is written as
  $a =a_0 + a_1 x + \cdots + a_{τ-1} x^{τ-1}$, for some $x$ that
  satisfies $x^{\ell^{i-1}}=x_1$, where $x₁∈F₁$ is a root of $P₁$.

  The first step, independently of $a$, is to compute $y=x^σ$. Writing
  $σ = u \ell^{i-1} + r$, with $r < \ell^{i-1}$, we see that $y$ is
  given by $x₁^{u \bmod \#F₁}x^r$. We compute $x₁^{u\bmod\#F₁}$ using
  $O(ℓ\MM(ℓ)\log(q))$ operations in $\F_q$, and we keep this element
  as a monomial of $F₁[x]$.

  Finally, once we know $y$, we compute $a(y)$ by a Horner scheme. All
  powers $y^i$ we need are themselves monomials in $F₁[x]$, each
  computed from the previous one using $O(\MM(ℓ))$ operations in
  $\F_q$, for a total of $O(ℓ^i\MM(ℓ))$. Finally the monomials
  $a_iy^i$ are combined together to form a polynomial in $x$ of degree
  $O(ℓ^i)$, and then brought to a canonical form in $F_i$ via a
  modular reduction at a cost of $O(\MM(ℓ^i))$ operations.
  \todo{?? Why do we need a reduction? We compute all $y_i$'s reduced modulo 
    $x^{\ell^{i-1}}-x_1$; they are still monomials after reduction.}
\end{proof}

Note that the complexity of the previous algorithm can be
significantly improved when $F₁=F₀$, as shown in~\cite{DoSc12} for the
case $ℓ=2$ \todo{?}.  Summarizing, the following computations can be performed
in a Kummer tower at the indicated asymptotic costs, all expressed in 
terms of operations in $\F_q$.
\begin{itemize}
\item basic arithmetic operations (addition, multiplication) in $F_i$,
  using $O(\MM(ℓ^i))$ operations;
\item inversion in $F_i$ using $O(\MM(ℓ^i)\log(ℓ^i))$
  operations (when $ℓ=2$, a factor of $i$ can be saved
    here~\cite{DoSc12}, but we will disregard this optimization for
    simplicity.)
\item mapping elements from $F_{i-1}$ to $F_i$ and \emph{vice versa}
  at no arithmetic cost;
\item multiplication and Euclidean division of polynomials of degree
  at most $d$ in $F_i[x]$ using $O(\MM(dℓ^i))$ operations, via
  Kronecker's
  substitution, as already done in e.g.~\cite{vzgathen+shoup92};
\item computing a $(\#F_j)$-th power in $F_i$ using
  $O(ℓ^i\MM(ℓ) + \MM(ℓ^i))$ operations, after a precomputation that
  uses $O(ℓ\MM(ℓ)\log(q))$ operations.
\end{itemize}

For one fundamental operation, we only have an efficient algorithm in
the case $ℓ=2$, hence we introduce the following notation:
\begin{itemize}
\item $\RR(i)$ is the cost of finding a root of a polynomial of degree
  $ℓ$ in $F_i[x]$.
\end{itemize}
For $\ell=2$, Doliskani and Schost show that
$\RR(i)=O(\MM(ℓ^i)\log(ℓ^iq))$. For general $ℓ$, we have
$\RR(i)=O(ℓ^i\MM(ℓ^{i+1})\log(ℓ)\log(ℓq))$ using the variant of the
Cantor-Zassenhaus algorithm described in~\cite[Chapter~14.5]{vzGG}, or
$\RR(i)=O\bigl((ℓ^{i(ω+1)/2}+\MM(ℓ^{i+1}\log (q)))i\log(ℓ)\bigr)$
using~\cite{kaltofen+shoup97}. Here, $\omega$ is such that matrix
multiplication in size $m$ over any ring can be done in $O(m^\omega)$
base ring operations (so we can take $\omega =2.38$ using the 
Coppersmith-Winograd algorithm). In any case, $\RR(i)$ is between 
linear and quadratic in the degree $\ell^{i}$.

\section{The Frobenius and the volcano}
\label{sec:isogeny-volcanoes}

In this section we explore some fundamental properties of ordinary
elliptic curves over finite fields: the structure of their isogeny
classes, its relationship with the rational $ℓ^∞$-torsion points, and
with the Frobenius endomorphism $π$.

\subsection{Isogeny volcanoes}

For an extensive introduction to isogeny volcanoes we address the
reader to~\cite{sutherland2013isogeny}.  We recall here, without their
proof, two results about $ℓ$-isogenies between ordinary elliptic
curves.

\begin{prop}[{\cite[Proposition~21]{Kohel}}] \label{prop:isogeny-asc-desc}
Let~$ϕ: E → E'$ be an $ℓ$-isogeny between ordinary elliptic curves
and~$\mathcal O, \mathcal O'$ be their endomorphism rings.
Then one of the three following cases is true:
\begin{enumerate}
\item $[\mathcal O':\mathcal O] = ℓ$,
in which case we call $ϕ$ \emph{ascending};
\item $[\mathcal O:\mathcal O'] = ℓ$,
in which case we call $ϕ$ \emph{descending};
\item $\mathcal O' = \mathcal O$,
in which case we call $ϕ$ \emph{horizontal}.
\end{enumerate}
\end{prop}
\begin{prop}[{\cite[Proposition~23]{Kohel}; \cite[Lemma~6]{sutherland2013isogeny}}] \label{prop:isogeny-count}
Let~$E$ be an ordinary elliptic curve with endomorphism ring~$\mathcal O$.
\begin{enumerate}
\item If $\mathcal O$~is $ℓ$-maximal then
there are $(d_K/ℓ)+1$~horizontal $ℓ$-isogenies from~$E$
(and no ascending $ℓ$-isogenies).
\item If $\mathcal O$~is not $ℓ$-maximal then
there are no horizontal $ℓ$-isogenies from~$E$,
and one ascending $ℓ$-isogeny.
\end{enumerate}
\end{prop}

A \emph{volcano of $ℓ$-isogenies} is a connected component
of the graph of rational $ℓ$-isogenies between curves defined on~$\mathbb F_q$.
The \emph{crater} is the subgraph corresponding to curves
having an $ℓ$-maximal endomorphism ring.
The shape of the crater is given by the Kronecker symbol~$(d_K/ℓ)$,
as per Proposition~\ref{prop:isogeny-count}.
For any~$k ≥ 0$, a $ℓ^k$-isogeny is \emph{horizontal}
if it is the composite of $k$~horizontal $ℓ$-isogenies.
The \emph{depth} of a curve is its distance from the crater.
It is also the $ℓ$-adic valuation of the conductor
of~$\mathcal O = \mathrm{End}(E)$.

		\begin{figure}[h]
		\begin{center}
		
        \begin{tikzpicture}[scale=0.3]
        \coordinate (A) at (0,0);
		\coordinate (B) at (230:5);
		\coordinate (C) at (270:4);
		\coordinate (D) at (310:5);
		\draw (A) node{$\bullet$};
		\draw (B) node{$\bullet$};
		\draw (C) node{$\bullet$};
		\draw (D) node{$\bullet$};
		\node at (0,-6.7) {``Stromboli'': $(d_K/ℓ) = -1$};
		\draw (A)--(B);
		\draw (A)--(C);
		\draw (A)--(D);
		
		\begin{scope}[xshift=10cm]
		\coordinate (A) at (0,0);
		\coordinate (B) at (5.5,0);
		\coordinate (C) at (240:4.2);
		\coordinate (D) at (300:4.2);
		\draw (A) node{$\bullet$};
		\draw (B) node{$\bullet$};
		\draw (C) node{$\bullet$};
		\draw (D) node{$\bullet$};
		\node at (2.8,-6.7) {``Vesuvius'': $(d_K/ℓ) = 0$};
		\draw (A)--(B);
		\draw (A)--(C);
		\draw (A)--(D);
		\end{scope}
		
		\begin{scope}[xshift=15.5cm]
		\coordinate (A) at (0,0);
		\coordinate (C) at (240:4.2);
		\coordinate (D) at (300:4.2);
		\draw (A) node{$\bullet$};
		\draw (C) node{$\bullet$};
		\draw (D) node{$\bullet$};
		\draw (A)--(C);
		\draw (A)--(D);
		\end{scope}
		
		\begin{scope}[xshift=25cm]
		\node (A) at (-3,0) {$\bullet$};
		\node (B) at (3,0) {$\bullet$};
		\node (C) at (270:1.2) {$\bullet$};
		\node (D) at (90:1.5) {};
		\node at (0,-6.7) {``Etna'': $(d_K/ℓ) = +1$};
		%\draw[-] (A.center) to[bend right=25] (C.center);
		\draw[-,dashed] (A.center) to[bend left=40] (B.center);
		%\draw[-] (B.center) to[bend left=25] (C.center);
		%\draw[-,dashed] (B.center) to[bend right] (D.center);
		\draw[-] (A.center) to[bend right=40] (B.center);
			\begin{scope}[xshift=-3cm]
			\coordinate (A) at (0,0);
			\coordinate (C) at (270:4.2);
			\draw (A) node{$\bullet$};
			\draw (C) node{$\bullet$};
			\draw (A)--(C);
			\end{scope}
			\begin{scope}[xshift=3cm]
			\coordinate (A) at (0,0);
			\coordinate (C) at (270:4.2);
			\draw (A) node{$\bullet$};
			\draw (C) node{$\bullet$};
			\draw (A)--(C);
			\end{scope}
			\begin{scope}[yshift=-1.2cm]
			\coordinate (A) at (0,0);
			\coordinate (C) at (270:4.2);
			\draw (A) node{$\bullet$};
			\draw (C) node{$\bullet$};
			\draw (A)--(C);
			\end{scope}
		\end{scope}
		\end{tikzpicture}
		%\end{center}
		\caption{The three shapes of volcanoes of $2$-isogenies }
		
		\end{center} 
		\end{figure}
\subsection{The $ℓ$-adic Frobenius}

In the rest of this paper we consider only a volcano with a cyclic
crater (i.e. we assume $(d_K/\ell) = +1$),
so that $ℓ$~is an Elkies prime for these curves.
This implies that the Frobenius automorphism on~$T_ℓ(E)$,
which we write~$π|T_ℓ(E)$, has two distinct eigenvalues~$λ ≠ μ$.
The depth of the volcano of $\F_q$-rational $ℓ$-isogenies
is~$h = v_ℓ(λ-μ)$.

\begin{prop}\label{prop:matrice-frobenius}
Let~$E$ be an ordinary elliptic curve with Frobenius endomorphism~$π$.
Assume that the characteristic polynomial of~$π$
has two distinct roots~$λ, μ$ in~$ℤ_ℓ$,
so that the $ℓ$-isogeny volcano has a cyclic crater.
Then there exists a unique $a ∈ \llbracket 0, ℓ^h - 1 \rrbracket$
such that $π|T_ℓ(E)$~is conjugate, over~$ℤ_ℓ$,
to the matrix $\smat{λ & a\\ 0 & μ}$.
% where $a ∈ ℤ$, $0 ≤ a ≤ ℓ^{h} - 1$,
Moreover $a = 0$ if~$E$ lies on the crater,
and else $h - v_{ℓ}(a)$~is the depth of~$E$ in the volcano.
\end{prop}
\begin{proof}
Since the characteristic polynomial of~$π$ splits over~$ℤ_ℓ$,
the matrix of~$π|T_ℓ(E)$ is trigonalizable.
Conjugating the matrix $\smat{λ & a\\0 & μ}$
by~$\smat{1 & b\\0 & 1}$ replaces~$a$ by~$a + b (λ - μ)$,
so that $a$~is well-defined modulo~$ℓ^h$.
Finally, by Tate's theorem~\cite[Isogeny theorem 7.7 (a)]{Sil},
$\mathcal O ⊗ ℤ_ℓ$~is isomorphic to the order in~$ℚ_ℓ[π_ℓ]$
of matrices with integer coefficients,
which is generated by the identity and~$ℓ^{-\min (h, v_ℓ(a))} (π_ℓ-λ)$.
\end{proof}

We now study the action of $ℓ$-isogenies on the $ℓ$-adic Frobenius by
showing the link between two related notions of diagonalization.

\begin{defi}[(Horizontal and diagonal bases)]
  Let~$E$ be a curve lying on the crater. We call a
  basis of~$E[ℓ^k]$ \emph{diagonal} if $π$~is diagonal in it; we call
  it \emph{horizontal} if both basis points generate the kernel of
  horizontal $ℓ^k$-isogenies. Accordingly, we also call diagonal
  (resp. horizontal) the generators of a diagonal (resp. horizontal)
  basis.
\end{defi}

\begin{prop} \label{prop:diagonal-horizontal}
Let~$E$ be a curve lying on the crater and~$P$ be a point of~$E[ℓ^k]$.
Then $ℓ^h P$~is horizontal if, and only if, $P$~is an eigenvector for~$π$.
If $π(P) = λ P$ then we say that $ℓ^h P$~has the direction~$λ$.
\end{prop}
Let $R$ be a point of $E$ of order~$ℓ^k$, let $ϕ$ be the isogeny 
with kernel~$\chev{R}$, and let $E'$ be its image. The subgroup~$\chev{R}$ defines a point in
the projective space of~$E[ℓ^k]$,
which is a projective line over~$ℤ/ℓ^k ℤ$.
There exists a canonical bijection~\cite[II.1.1]{SL2} between
this projective line and
the set of lattices of index~$ℓ^k$ in the $ℤ_ℓ$-module $T_ℓ(E)$:
it maps a line~$\chev{R}$ to the lattice~$Λ_R = \chev{R} + ℓ^k T_ℓ(E)$.
This lattice is also the preimage by~$ϕ$
of the lattice~$ℓ^k T_ℓ(E')$.

Fix a basis~$(P, Q)$ of~$E[ℓ^k]$, let $Π$ be the matrix of $π$
in this basis, and let~$R = x P + y Q$.
The lattice~$Λ_R$ is generated by the columns of the matrix
%% $L_R = \smat{ℓ^k & 0 & x\\0 & ℓ^k & y}$.
$L_R = \left (\begin{smallmatrix}ℓ^k & 0 & x\\0 & ℓ^k & y\end{smallmatrix} \right )$.
The Hermite normal form of~$L_R$
is 
%% $M_R = \smat{ℓ^{k-m} & x/y' \\ 0 & ℓ^m}$,
$M_R = \left (\begin{smallmatrix}ℓ^{k-m} & x/y' \\ 0 & ℓ^m\end{smallmatrix}\right )$,
where we write~$y = ℓ^m y'$ with~$ℓ ∤y'$,
and the columns of $M_R$ also generate the lattice~$Λ_R$.
We check that $M_R$ has determinant~$ℓ^k$.
Since $Λ_R = ϕ_R^{-1} (ℓ^k T_{ℓ} (E'))$,
there exists a basis of~$T_ℓ(E')$
in which $ϕ_R$ has matrix~$ℓ^k M_R^{-1}$.
Therefore, in that basis of~$T_ℓ(E')$,
the matrix of~$π|T_ℓ(E')$ is $M_R^{-1} · Π · M_R^{}$.
\begin{proof}[of Proposition~\ref{prop:diagonal-horizontal}]
Fix a basis~$(R, S)$ of~$E[ℓ^k]$ that diagonalizes~$π$.
We can write $P = x R + y S$;
without loss of generality we may assume $y=1$.
Let~$ϕ$ be the isogeny determined by~$ℓ^h P$, and let~$E'$ be its image.
Then $π|T_ℓ(E')$ has matrix~$\left ( \begin{smallmatrix}λ& ℓ^{h-k} x (λ-μ)\\ 0&μ
\end{smallmatrix}\right )$.
This matrix is diagonalizable only if~$v_{ℓ}(x) ≥ k - h$.
On the other hand, we can compute~$(π - μ) P = x (λ - μ) R$,
so that $P$~is an eigenvector on the same condition~$v_{ℓ}(x) ≥ k-h$.
\end{proof}

While horizontal bases are our main interest,
diagonal bases are easier to compute in practice.
Algorithms computing both kind of bases
are given in Section~\ref{sec:acti-frob-endm}.
The main tool for this is the next proposition:
given a horizontal point of order~$ℓ^k$,
it allows us to compute a horizontal point of order~$ℓ^{k+1}$.

\begin{prop}\label{prop:push-horizontal}
Let~$ψ: E → E'$ be a horizontal $ℓ$-isogeny with direction~$λ$.
For any point~$Q ∈ E[ℓ^∞]$,
if $ℓ · Q$~is horizontal with direction~$μ$,
then $ψ(Q)$ is horizontal with direction~$μ$.
\end{prop}
(Since $Q$~has direction $μ$,
its image~$ψ(Q)$ has the same multiplicative order as~$Q$).
\begin{proof}
Let~$Q' = ψ(Q)$ and~$\widehat{ψ}$ be the isogeny dual to~$ψ$.
Since both $\widehat{ψ}$~and~$\widehat{ψ}(Q') = ℓ Q$ are horizontal
with direction~$μ$, $Q'$~is also horizontal.
\end{proof}
\begin{prop}\label{prop:parallel}
Let~$ψ: E → E'$ be an isogeny of degree~$r$ prime to~$ℓ$.
\begin{enumerate}
\item The curves~$E$ and~$E'$ have the same depth
in their $ℓ$-isogeny volcanoes.
\item\label{prop:parallel:func} For any point~$P ∈ E[ℓ^k]$,
the isogenies with kernel $\chev{P}$ and~$\chev{ψ(P)}$ have the same type
(ascending, descending, or horizontal with the same direction).
\item\label{prop:parallel:ascent} If $P ∈ E[ℓ]$ and $P' ∈ E'[ℓ]$ are both ascending,
or both horizontal with the same direction,
then $E/P$ and~$E'/P'$ are again $r$-isogenous.
\end{enumerate}
\end{prop}
\begin{proof}
Points~(i) and~(ii) are consequences of Proposition~\ref{prop:matrice-frobenius}
and of the fact that $ψ$, being rational and of degree prime to~$ℓ$,
induces an isomorphism between the Tate modules of~$E$ and~$E'$,
commuting to the Frobenius endomorphisms.
For point~(iii), we just note that
since there exists a unique point of order~$ℓ$
either ascending or horizontal with a given direction,
we must have~$P' = ψ(P)$.
\end{proof}

%%%%%%%%%%%%%%%
\subsection{Galois classes in the $ℓ$-torsion}
Here we assume that $E$~has a $ℓ$-maximal endomorphism ring.
If $ℓ$~is odd, let~$α = v_ℓ(λ^{ℓ-1}-1)$ and~$β=v_ℓ(μ^{ℓ-1}-1)$;
if $ℓ=2$, let~$α=v_2(λ^2-1)-1$ and~$β = v_2(μ^2-1)-1$,
% Let~$α = v_{ℓ} (λ^{2(ℓ-1)}-1)$ and~$β = v_{ℓ} (μ^{2(ℓ-1)}-1)$%
 % \footnote{This trick ensures that for $ℓ$ odd, $α = v_{ℓ} (λ^{ℓ-1}-1)$,
% whereas for $ℓ=2$, $α = v_2 (λ^2-1)$ for~$ℓ=2$.}
% and assume without loss of generality that $α ≥ β$.
% Let~$α = v_{ℓ} (λ^{ℓ-1}-1)$ and~$β = v_{ℓ} (μ^{ℓ-1}-1)$
and assume without loss of generality that $α ≥ β$.
Since $λ ≢ μ \pmod{ℓ^{h+1}}$, it is impossible that $λ ≡ μ ≡ 1 \pmod{h}$,
so that one at least of the two valuations~$α, β$ is~$≤ h$,
and therefore~$β ≤ h$.
\label{sub:classes}
\begin{prop}\label{prop:classes}
% déjà supposé :
% Assume that $E$~has a $ℓ$-maximal endomorphism ring.
For any~$k$, let~$d_k$ be the degree of the smallest field extension $F/\F_q$
such that all the points of~$E[ℓ^k]⊂E(F)$. Then:
\begin{enumerate}
\item The order of $q$ in $(ℤ/ℓℤ)^×$ divides $d_1$,
and $d_1$ divides~$(ℓ-1)$.
\item If $ℓ$~is odd then for all $k ≥ 1$,
$d_k = \mathrm {lcm} (d_1, ℓ^{k-β})$.
\item If $ℓ=2$ then $d_2 ∈ \acco{1,2}$ and, for all~$k ≥ 2$,
$d_k = \mathrm{lcm}(d_2, ℓ^{k-β})$.
\item For any~$n$, the group $E[ℓ^{∞}](F_n)$
is isomorphic to~$(ℤ/ℓ^{n+α} ℤ) × (ℤ/ℓ^{n+β} ℤ)$.
\item\label{prop:classes:count} The group~$E[ℓ^k]$ contains at most
$k · ℓ^{k+β}$ Galois conjugacy classes over~$F_1 = \F_{q^{d_1}}$.
\end{enumerate}
\end{prop}
\begin{proof}
The degree~$d_k$ is exactly the order of the matrix~$π|E[ℓ^k]$.
It is therefore the least common multiple of the multiplicative orders
of~$λ, μ$ modulo~$ℓ^k$.
This proves~(i) using the fact that~$λ · μ = q$.
% We conclude using the multiplicative structure of~$(ℤ/ℓ^k ℤ)^×$, and the fact
% that $λμ=q$.
% Note that in general the order of~$q$ is a strict divisor of~$d_1$,
% as is for example the case for~$π^2 - π + 29 = 0$ and~$ℓ = 7$,
% where $q = 29 ≡ 1 \pmod{7}$ and $d_1 = 6$.

For points~(i)--(v) we may assume that $d_1 = 1$.
Then, for any~$n$, $v_ℓ(λ^{2n}-1) = α + v_{ℓ} (2n)$.
Let~$(P, Q)$ be a diagonal basis of~$E[ℓ^k]$.
The point $(π^n - 1) (x P + y Q) = (λ^n-1) x P + (μ^n-1) y Q$
is zero iff $v_{ℓ} (x) + α + v_{ℓ} (n) ≥ k$
and~$v_{ℓ} (y) + β + v_{ℓ} (n) ≥ k$. This shows~(iv).
The largest Galois classes
are those for which~$v_{ℓ} (y) = 0$ and their size is~$ℓ^{k - β}$,
proving~(ii) and~(iii).
Moreover, for any~$i ≤ k-β$ the points in an orbit of size~$≤ ℓ^i$
are those for which~$v_{ℓ} (x) ≥ k - α - i$ and~$v_{ℓ} (y) ≥ k - β - i$;
there are at most $ℓ^{\min(α+i, k) + \min (β+i, k)}$ such points,
and therefore $ℓ^{\min(α+i, k) + \min(β, k-i)} ≤ ℓ^{k-i+β}$
corresponding classes.
Summing this over all~$i$ proves~(v).
\end{proof}
\section{Computing the action of the Frobenius endomorphism}
\label{sec:acti-frob-endm}

We continue here our study on the action of the Frobenius $π$ on
$E[ℓ^k]$.  Given an elliptic curve~$E$ with $ℓ$-maximal endomorphism
ring, we explicitly compute diagonal and horizontal bases of $E[ℓ^k]$
as defined in the previous section.  We will use the latter basis of
$E[ℓ^k]$ in Section~\ref{sub:final-interp}, to put restrictions on the
interpolation problem of our algorithm.

By Proposition~\ref{prop:classes}, there exists a Kummer tower
$F₀⊂\cdots⊂F_{k-\beta}$ such that all the points of~$E[ℓ^k]$ are
rational over~$F_{k-\beta}$. The algorithms presented next assume that
the tower has already been computed.

\subsection{Computation of a diagonal basis}
\label{ss:diagonal}

In Algorithm~\ref{alg:diagonal} below, we describe how to compute
eigenvalues of the Frobenius $\bmod \ell^k$ and corresponding
eigenvectors in the $\ell^{k}$-torsion subgroup.  We
write~$Q ← \sfdiv(ℓ, P)$ for the computation of a preimage of~$P$ by
multiplication by~$ℓ$.

\begin{algorithm}
\caption{\label{alg:diagonal}Computing a diagonal basis of $E[ℓ^k]$}
\begin{algorithmic}[1]
\REQUIRE $E$: an ordinary, $ℓ$-maximal elliptic curve; $k$: an integer.
\ENSURE $(P_k, Q_k )$: a basis of $E[\ell^k]$;
$λ, μ ∈ ℤ/ℓ^k ℤ$
such that $\pi(P_k)= λ P_k$, $ \pi(Q_k)= μ Q_k$.
\STATE Compute $(P_1,Q_1)$, a basis of $E[ℓ]$
\STATE $h:=1, u:=1$
\FOR {$i=1$  to  $k-1$}
\STATE\label{alg:diagonal:divide} $P' \leftarrow \sfdiv(\ell, P_i)$; $Q' \leftarrow \sfdiv(\ell, Q_i)$.
\STATE\label{alg:diagonal:frobenius} compute $\pi|(P',Q')=\left( \begin{array}{cc}
λ + a\ell^{i} & b\ell^{i}\\
c\ell^{i} & μ + d\ell^{i}
\end{array} \right) \pmod {\ell^{i+1}}$
with $a,b,c,d \in \mathbb{Z}/\ell\mathbb{Z}$
\STATE \textbf{if} {$λ \neq μ$} \textbf{then}
$u \leftarrow (λ -μ)/\ell^h$ \textbf{endif}
\STATE $(λ, μ) \gets
  (λ + a\ell^i, μ + d\ell^i)$
\STATE $(b',c') \gets (-b/u , c/u) \pmod{ℓ}$
\STATE $(P_{i+1},Q_{i+1}) \gets
  (P'+\ell^{i-1-h}b' Q',\;Q'+\ell^{i-1-h}c' P')$
\STATE \textbf{if} {$λ = μ$} \textbf{then} $h \leftarrow h+1$ \textbf{endif}
\ENDFOR
\RETURN $(P_{k},Q_{k},λ,μ)$
\end{algorithmic}
\end{algorithm}
\begin{prop}\label{th:diagonal}
  Algorithm~\ref{alg:diagonal} computes a diagonal basis of~$E[ℓ^k]$
  using
  $O(\RR(k-\beta) + ℓ^2\MM(ℓ^{k-β}) + ℓ\MM(\ell^2)\log(\ell)\log(\ell
  q))$ operations in $\F_q$.
\end{prop}
\begin{proof}
  \todo{correctness}

  To bootstrap the algorithm, we need to compute a basis of $E[ℓ]$
  over the field $F_1$. We do this by factoring the $ℓ$-division
  polynomial at a cost of $O(ℓ\MM(\ell^2)\log(\ell)\log(\ell q))$
  operations using the Cantor-Zassenhaus algorithm.

  Once $E[ℓ]$ has been computed, we can factor the
  multiplication-by-$ℓ$ map as a product of two $ℓ$-isogenies. Then,
  for any $P$ defined in $E(F_{i-β})$, the computation of
  $\sfdiv(ℓ, P)$ at Step~\ref{alg:diagonal:divide} costs $O(\RR(i-β+1))$
  operations.

  Evaluating~$π(P')$ in Step~\ref{alg:diagonal:frobenius} has a cost
  of~$O(\ell^{i-\beta+1}\MM(\ell)+\MM(\ell^{i-\beta+1}))$.
  Writing~$π(P')$ as a linear combination~$α P' + β Q'$ needs at
  most~$ℓ^2$ point additions, with a cost
  of~$ℓ^2 \mathsf{M}(ℓ^{i-\beta+1})$.  Finally, all other steps are
  negligible.

  Since the cost of each loop grows geometrically, the last loop
  dominates all others, and gives the stated complexity.
\end{proof}

\subsection{Computation of a horizontal basis}
\label{ss:horizontal}

Using the previous algorithm
we can compute a diagonal basis of~$E[ℓ^{h+1}]$.
By Proposition~\ref{prop:diagonal-horizontal},
this gives us a horizontal basis of~$E[ℓ]$.
Thanks to Proposition~\ref{prop:push-horizontal},
we can use this information to improve horizontal points of~$E[ℓ^i]$
into horizontal points of~$E[ℓ^{i+1}]$, as illustrated in
Algorithm~\ref{alg:horizontal}.

\begin{algorithm}
\caption{\label{alg:horizontal}Computing a horizontal point of order~$ℓ^k$}
\begin{algorithmic}[1]
\REQUIRE $(P_0, Q_0)$: a diagonal basis of~$E[ℓ^{h+1}]$; $k$: an integer.\\
\ENSURE $R$: a horizontal point of~$E[ℓ^k]$ with direction~$λ$.
\FOR{$i = 1$ to~$k-1$}
\STATE $ϕ_i \gets $ isogeny with kernel~$\chev{ℓ^{h} P_{i-1}}$
\STATE $Q_{i} \gets ϕ_i(Q_{i-1})$
\STATE\label{alg:horizontal:divide} $P' \gets \sfdiv(\ell, ϕ_i(P_{i-1}))$.
\STATE\label{alg:horizontal:frob} Write~$π(P') = λ P' + b Q_i$ for~$b ∈ ℤ/ℓℤ$ and
let $P_{i} \gets P' - (b/μ) Q_i$.
\ENDFOR
\RETURN\label{alg:horizontal:final} $R = \widehat{ϕ}_1 ∘ … ∘ \widehat{ϕ}_{k-1}
  (\sfdiv( ℓ^{k-(h+1)}, P_{k-1}) )$. 
\end{algorithmic}
\end{algorithm}
\begin{prop}\label{th:horizontal}
  Algorithm~\ref{alg:horizontal} is correct and computes its output
  using $O(\RR(k-\beta) + k\RR(h-β+1) + kℓ^2\MM(ℓ^{h-β+1}))$
  operations in $\F_q$.
\end{prop}
\begin{proof}
We check that at step~$i$ of the loop,
the points~$(P_i, Q_i)$ form a diagonal basis of~$E_i[ℓ^{h+1}]$,
and $ϕ_i$~has direction~$λ$.
The fact that $R$~is horizontal is then a consequence
of Proposition~\ref{prop:push-horizontal}.

The two most expensive operations in the loop are
Steps~\ref{alg:horizontal:divide} and~\ref{alg:horizontal:frob},
costing respectively $O(\RR(h-β+1))$ and $O(ℓ^2\MM(ℓ^{h-β+1}))$, as
discussed in the proof of Proposition~\ref{th:diagonal}. They are
repeated $k$ times. Finally, Step~\ref{alg:horizontal:final} is
dominated by the last $\sfdiv$ operation, which costs $O(\RR(k-β))$.
\end{proof}

One application of Algorithm~\ref{alg:diagonal} (with input~$k ← h+1$)
and two applications of Algorithm~\ref{alg:horizontal} allow us
to compute a horizontal basis of~$E[ℓ^k]$.
This could be done directly with Algorithm~\ref{alg:diagonal} instead,
but that would require computing in an extension $F_{k+h-\beta}$.


%%%%%%%%%%%%%%%

\section{Interpolation step}
\label{sec:interpolation}

After constructing bases $(P,Q)$ of $E[ℓ^k]$ and $(P',Q')$ of
$E'[ℓ^k]$ using the algorithms of the previous section, our algorithm
computes the polynomial with coefficients in $\F_q$ sending
$x(P)↦x(P')$, $x(Q)↦x(Q')$, and the other abscissas accordingly.  In
this section we give an efficient algorithm for this specific
interpolation problem. The algorithm has already appeared
in~\cite{df10} and~\cite{enge+morain03}; we recall it here, and adapt
the complexity analysis to our setting.

\subsection{Rational interpolation}

We start by tackling a simpler problem. We suppose we have constructed
a tower of Kummer extensions $\F_q=F_0⊂F_1⊂\cdots⊂F_n$, with
$[F₁:F₀]\mid(ℓ-1)$, and $[F_{i+1}:F_i]=ℓ$ for any $i>0$. Given two
elements $v,w∈F_n\setminus F_{n-1}$, we want to compute polynomials
$T$ and $L$ such that:
\begin{itemize}
\item $T \in \F_q[x]$ is the minimal polynomial of $v$, of degree
  $d=\deg T<ℓ^n$;
\item $L$ is in $\F_q[x]$, of degree less than $d$, and $L(v)=w$.
\end{itemize}
Observe that, since $v,w∉F_{n-1}$, we necessarily have $v_ℓ(d)=n-1$,
so that $ℓ^{n-1}≤d<ℓ^n$.

Using a fast interpolation algorithm~\cite[Chapter~10.2]{vzGG}, the
polynomials $T$ and $L$ could be computed in
$O\bigl(n\MM(ℓ^{2n})\log(ℓ)\bigr)$ operations in $\F_q$. We can do
much better by exploiting the form of the Kummer tower, and the
Frobenius algorithm given in Lemma~\ref{lemma:frob-ell}.

Following~\cite{df10}, we first compute $T$, starting from
$T^{(0)}=x-v$.  We let $\sigma_i$ be the map that takes all the
coefficients of a polynomial in $F_{n-i}[x]$ to the power
$\#F_{n-i-1}$. For $i=0,\dots,n-2$, suppose we know a polynomial
$T^{(i)}$ of degree $\ell^i$ in $F_{n-i}[x]$. Then, compute the
polynomials $T^{(i,j)}$ given by
\begin{equation*}
  T^{(i,j)}= \sigma_i^j\bigl (T^{(i)} \bigr)
  \quad\text{for $0 \le j \le \ell-1$},
\end{equation*}
and define
$$T^{(i+1)}=\prod_{j=0}^{\ell-1} T^{(i,j)};$$ one easily sees that
$T^{(i+1)}$ is the minimal polynomial of $v$ over $F_{n-i+1}$. For the
last step $i=n-1$, we proceed in a similar way by defining
\begin{equation*}
  T = T^{(n)}=\prod_{j=0}^{d/ℓ^{n-1}} T^{(n-1,j)}.
\end{equation*}

\begin{lem}\label{lemma:interpolation:minpoly}
  The cost of computing $T^{(n)}=T$ is bounded by $O(nℓ\MM(ℓ^{n+1}))$
  operations in $\F_q$.
\end{lem}

\begin{proof}
  At each step $i$, from the knowledge of $T^{(i)}$ we compute all
  $T^{(i,j)}$ using Lemma~\ref{lemma:frob-ell}. The cost for a single
  polynomial $T^{(i,j)}$ is of
  $O\bigl(ℓ^i\bigl(ℓ^{n-i}\MM(ℓ)+\MM(ℓ^{n-i})\bigr)\bigr)$ operations,
  which we simplify to a total of $O(ℓ\MM(ℓ^{n+1}))$ for all $O(\ell)$
  of them.

  From the $T^{(i,j)}$'s we compute $T^{(i+1)}$ using a subproduct
  tree, as in~\cite[Lemma~10.4]{vzGG}. The result has degree
  $O(ℓ^{i+1})$ and coefficients in $F_{n-i}$, thus the overall cost is
  $O(\MM(ℓ^{n+1})\log(ℓ))$. After $T^{(i+1)}$ is computed this way, we
  can convert its coefficients to $F_{n-i-1}$ at no algebraic cost.

  Summing over all $i$, we obtain the stated complexity.
\end{proof}

We can finally proceed with the interpolation itself. First, compute
$w' = w/T'(v)$ and let $L^{(0)}=w'$.  Next, for $i=0,\dots,n-2$,
suppose we know a polynomial $L^{(i)}$ in $F_{n-i}[x]$ of degree less
than $\ell^i$. We compute the polynomials $L^{(i,j)}$ given by
$$L^{(i,j)}= \sigma_i^j\bigl(L^{(i)}\bigr),$$
for $0 \le j \le \ell-1$, and
$$L^{(i+1)} = \sum_{j=0}^{\ell-1} L^{(i,j)}\frac{T^{(i+1)}}{T^{(i,j)}}.$$ The last step $i=n-1$
is done analogously.  As shown in~\cite{df10}, $L^{(n)}$ is the
polynomial $L$ we are looking for.

\begin{prop}
  Given elements $v,w∈F_n\setminus F_{n-1}$, the cost of computing the
  minimal polynomial $T∈\F_q[x]$ of $v$, and the interpolating
  polynomial $L∈\F_q[x]$ such that $L(v)=w$, is of $O(nℓ\MM(ℓ^{n+1}))$
  operations in $\F_q$.
\end{prop}
\begin{proof}
  After the polynomials $T^{(i)}$ have been computed, we need to
  compute $T'(v)$. This is done by means of successive Euclidean
  remainders, since
  $T'(v) = (((T' \bmod T^{(1)}) \bmod T^{(2)}) \cdots \bmod T^{(n)})$.
  At stage $i$, we have to compute the Euclidean division of a
  polynomial of degree $O(ℓ^{n-i+1})$ by one of degree $O(ℓ^{n-i})$ in
  $F_i[x]$. Using the complexities from Section~\ref{sub:towers} we
  deduce that each division can be done in time $O(\MM(ℓ^{n+1}))$, for
  a total of $O(n\MM(ℓ^{n+1}))$ operations. Then, computing
  $w' = w/T'(v)$ takes $O(\MM(\ell^n)\log(\ell^n))$ operations.

  Finally, at each step $i$, the polynomials $L^{(i,j)}$ are computed
  at a cost of $O(ℓ\MM(ℓ^{n+1}))$, as in the proof of
  Lemma~\ref{lemma:interpolation:minpoly}.  The computation of
  $T^{(i)}$ requires $O(ℓ)$ additions, multiplications and divisions
  of polynomials of degree $O(ℓ^{i+1})$ with coefficients in
  $F_{n-i}$, again at a cost of $O(ℓ\MM(ℓ^{n+1}))$. Summing over all
  $i$, the complexity statement follows readily.
\end{proof}

We finally go to the general problem of interpolating a polynomial in
$\F_q[x]$ at many points of $F_n$.

\begin{prop}\label{prop:interpol}
  Let $(v_1,w_1),\dots,(v_s,w_s)$ be pairs of elements of $F_n$, let
  $t_i$ be the degree of the minimal polynomial of $v_i$, and let
  $t=\sum t_i$. The polynomials
  \begin{itemize}
  \item $T∈\F_q[x]$ of degree $t$ such that $T(v_i)=0$ for all $i$,
    and
  \item $L∈\F_q[x]$ of degree less than $t$ such that $L(v_i)=w_i$ for
    all $i$
  \end{itemize}
  can be computed using
  $O\bigl(\MM(t)\log(s) + nℓ\MM(ℓ^2t)\bigr)$ operations in $\F_q$.
\end{prop}
\begin{proof}
  The polynomial $T$ is simply the product of all the minimal
  polynomials $T_i$. Let $n_i=v_ℓ(t_i)$, so that
  $v_i,w_i∈F_{n_i+1}\setminus F_{n_i}$, and $ℓ^{n_i}≤t_i<ℓ^{n_i+1}$.
  We convert $(v_i,w_i)$ to a pair of elements of $F_{n_i+1}$ at no
  algebraic cost, then we compute $T_i$ as explained previously at a
  cost of $O(nℓ\MM(ℓ^{n_i+2}))$ operations. Bounding $ℓ^{n_i}$ by
  $t_i$, summing over all $i$, and using the superlinearity of $\MM$,
  we obtain a total cost of $O(nℓ\MM(ℓ^2t))$ operations.
  Simultaneously, we compute all the polynomials $L_i$ such that
  $L_i(v_i)=w_i$, at the same cost.

  Then we arrange the $T_i$'s into a binary subproduct tree and
  multiply them together. A balanced binary
  tree, though not necessarily optimal, has a depth of
  $O(\log (s))$, and requires $O(\MM(t))$ operations per level. Thus
  we can bound the cost of computing $T$ by $O(\MM(t)\log(s))$.

  Finally, using the same subproduct tree structure, we apply the
  Chinese remainder algorithm of~\cite[Chapter~10]{vzGG} to compute
  the polynomial $L$ at the same cost $O(\MM(t)\log(s))$.
\end{proof}


\section{The complete algorithm}
\label{sec:complete-algorithm}

We finally come the description of the full algorithm. As stated in
the introduction, we are given two elliptic curves $E$ and $E'$, and
an integer $r$, and we want to compute an isogeny $ψ:E→E'$ of degree
$r$.

Since the algorithms of Section~\ref{sec:acti-frob-endm} only apply to
curves on top of volcanoes with cyclic crater, we first need to
determine a small Elkies prime $ℓ$ for $E$ and $E'$, and then reduce
to an explicit isogeny problem on the crater of the
$ℓ$-volcanoes. These steps are discussed and analyzed next.

%%%%%%%%%%%%%%%
\subsection{Finding a suitable $ℓ$-volcano}
\label{sub:shape-volcano}

Our algorithm uses an Elkies prime~$ℓ$.  According to Chebotarev's
density theorem, the density of primes~$ℓ$ such that~$(d_K/ℓ) = +1$ is
asymptotically~$1/2$, so that we need only try a $O(1)$ number of
primes~$ℓ$.  Since~$d_K$ is not assumed to be known yet, we need to be
able to compute the height $h$ of the volcano, the shape of its
crater, as well as the shortest $ℓ$-isogeny chain from~$E$ to the
crater.

The algorithms of Fouquet and Morain~\cite{volcano} compute the height
$h$ and find a curve $E_{\max}$ on the crater at the cost of $O(ℓh^2)$
factorizations of the $ℓ$-th modular polynomial $Φ_ℓ$. The polynomial
$Φ_ℓ$ is computed using $\tildO(ℓ³\logℓ)$ binary operations and
$O(\MM(ℓ^2)\log q)$ operations in $\F_q$, then each factorization costs
$O(\MM(ℓ)\log(ℓ)\log(ℓq))$ operations
using the Cantor-Zassenhaus algorithm. More efficient methods for special
instances of volcanoes are presented in~\cite{MiretMRV05} and
in~\cite{IonicaJ10}, but we ignore them .

Once we know a curve on the crater,
we still have to determine the shape of the crater.
Since the height~$h$ of the volcano is known,
using Algorithm~\ref{alg:diagonal} we can compute a matrix of $π|E[ℓ^{h+1}]$.
If this matrix has two distinct eigenvalues then the crater is cyclic,
otherwise it is not.

By Proposition~\ref{prop:parallel},
the depth of~$E$ and $E'$ below their respective craters is the same.
Using the methods of Section~\ref{sub:shape-volcano},
we can compute the shortest path of $ℓ$-isogenies~$α: E → E_{\max}$,
$α': E' → E'_{\max}$ linking the curves~$E, E'$ to the craters.
By Proposition~\ref{prop:parallel}~\ref{prop:parallel:ascent},
the curves~$E_{\max}$ and~$E'_{\max}$ are again $r$-isogenous;
we can use our algorithm to compute such an isogeny~$ψ_{\max}$.
Then $ψ = (α')^{-1} ∘ ψ_{\max} ∘ α$ is the required $r$-isogeny; its
kernel can be computed in $O(h\MM(r)\log(r))$ operations by evaluating
$α^{-1}$ on the kernel of $ψ_{\max}$ via a sequence of resultants.


\subsection{Interpolating the isogeny}
\label{sub:final-interp}

We now assume that both curves~$E, E'$
have $ℓ$-maximal endomorphism rings.
We fix bases of~$E[ℓ^k]$, $E'[ℓ^k]$ and write~$π, π'$ for the matrices
of the Frobenius.
Since $ψ$~is rational, its matrix satisfies the relation~$π' · ψ = ψ · π$
in $ℤ_ℓ^{2×2}$ and hence in~$(ℤ/ℓ^k ℤ)^{2 × 2}$.

If diagonal bases of~$E[ℓ^k], E'[ℓ^k]$ are used, then,
since $π$~is a cyclic endomorphism of~$ℤ_ℓ^2$,
this condition seems to ensure that $ψ$~is a diagonal matrix;
however, $ℤ/ℓ^k ℤ$~is not an integral domain
and $π$~is congruent, modulo~$ℓ^h$, to the scalar matrix~$λ$,
so we can only say that~$ψ\pmod{ℓ^{k-h}}$ is diagonal.
If on the other hand we choose \emph{horizontal} bases
of~$E[ℓ^k], E'[ℓ^k]$ then, by Proposition~\ref{prop:parallel}~\ref{prop:parallel:func},
we know that $ψ$~is a diagonal matrix.

We then enumerate all the $ℓ^{2k-2}$ invertible diagonal matrices; for
each matrix~$M$, we interpolate the action of~$M$ on~$E[ℓ^k]$ as a
rational fraction, and verify that it is an $r$-isogeny. The
successful interpolation will be our explicit isogeny~$ψ$.

We interpolate using the abscissas of non-zero points of $E[ℓ^k]$;
there are $(ℓ^{2k}-1)/2$ distinct such abscissas (or $2^{2k-1}+1$ when
$ℓ=2$).  Since the isogeny~$ψ$ is a rational fraction of
degrees~$(r, r-1)$, it is defined by $2r$ coefficients.  For this
method to work, we therefore select the smallest~$k ≥ h+1$ such
that~$ℓ^{2k}-1 > 4r$.

Summarizing, our algorithm for two curves having $ℓ$-maximal
endomorphism ring proceeds as follows:
\begin{enumerate}
\item\label{alg:ours:horizontal} Use Algorithms~\ref{alg:diagonal}
  and~\ref{alg:horizontal} to compute horizontal bases $(P,Q)$ of
  $E[ℓ^k]$ and $(P',Q')$ of $E'[ℓ^k]$;
\item\label{alg:ours:T} Compute the polynomial~$T$ vanishing
  on the abscissas of $\langle P,Q\rangle$ using the method of
  Section~\ref{sec:interpolation};
\item\label{alg:ours:for} For each invertible diagonal matrix
  $\smat{a&0\\0&b}$ in $(ℤ/ℓ^k ℤ)^{2×2}$:
  \begin{enumerate}
  \item\label{alg:ours:interp} compute the interpolation polynomial
    $L_{a,b}$ such that
    $L_{a,b} (x (u P + v Q)) = x(a\, u\,P' + b\,v\, Q')$ for all
    $u, v ∈ ℤ/ℓ^k ℤ$;
  \item\label{alg:ours:cauchy} Use the \emph{Cauchy interpolation
      algorithm} of~\cite[Chapter~5.8]{vzGG} to compute a rational
    fraction $F_{a,b}≡L_{a,b}\pmod{T}$ of degrees~$(r, r-1)$;
  \item If $F_{a,b}$ defines an isogeny of degree $r$, return it and
    stop.
  \end{enumerate}
\end{enumerate}

\begin{thm*}
It is possible to solve the “Explicit Isogeny Problem”
using at most
\[ O(r · \MM(r) · \log^2(r) + r · \log(r) · \log(q))\]
operations in~$\F_q$.
\end{thm*}
\begin{proof}
  We expect to find a small Elkies prime $ℓ$ after $O(1)$ trials. Also
  note that the height $h$~of the volcano of $ℓ$-isogenies is small
  with very high probability; in case it is not, we can simply
  discard~$ℓ$ and choose another one. Hence, we treat both $h$ and $ℓ$
  as constants for the purpose of this theorem.

  First we reduce the problem to one on $ℓ$-maximal curves. This costs
  $O(\log(q) + \MM(r)\log(r))$, as outlined in
  Section~\ref{sub:shape-volcano}.  Then we pick $k$ with
  $ℓ^{2k} ∈ O(r)$, and we apply the algorithm outlined above. By
  Proposition~\ref{prop:classes}, there is a $β<h$ such that $E[ℓ^k]$
  is contained in $E(F_n)$ with $n=k-β$. We thus construct the Kummer
  tower $F₀⊂\cdots⊂F_n$, and we do the precomputations required by
  Lemma~\ref{lemma:frob-ell} at a cost of $O(\log(r)\log(q))$.

  Step~\ref{alg:ours:horizontal} costs $O(k\RR(k-\beta))$ according to
  Propositions~\ref{th:diagonal} and~\ref{th:horizontal}. Using the
  estimates of Section~\ref{sub:towers}, we see that this cost is
  bounded by $O(\MM(\sqrt{r})\log(rq))$ if $ℓ=2$, or by
  $O(r +\MM(\sqrt{r}\log(q))\log(r))$ otherwise.

  By Proposition~\ref{prop:classes}~\ref{prop:classes:count}, there
  are at most~$O(k· ℓ^{k+β})$ Galois classes in~$E[ℓ^k]$. In order to
  apply the algorithms of Section~\ref{sec:interpolation}, we need to
  compute a representative for each class. Each representative is
  computed from the basis $(P,Q)$ using point multiplication by two
  scalars $≤ℓ^k$ in the field $F_n$, which costs
  $O(\MM(ℓ^n)\log(ℓ^k))$ operations. We thus have a total cost of
  $O(k\MM(ℓ^{2k})\log(ℓ^k)) ⊂ O(\MM(r)\log^2(r))$ to compute all such representatives.

  Then, using proposition~\ref{prop:interpol}, where the total degree
  is~$t = (ℓ^{2k}-1)/2∈O(r)$, and the number of interpolation points
  is $s∈O(k·ℓ^{k+β})$, we can compute the polynomials~$T$
  and~$L_{a,b}$ at a cost of~$O(\MM(r)\log^2(r))$.
  The cost of computing $F_{a,b}$, and identifying the isogeny is
  dominated by that of computing~$L_{a,b}$~\cite[§3.3]{df10}.
  Finally, in general approximately $ℓ^{2k} ≈ r$~candidate matrices
  must be tried before finding the isogeny.
\end{proof}


\section{Experimental results}
\label{sec:implem}
We implement the algorithm on SageMath, we run the test on SageMath v7.0 only for $\ell=2$. We place ourselves in the optimal case of \cite{DoSc12} with $p = 1 \bmod 4$. Since the main cost of the algorithm comes from the repetition of the interpolation, we observe some floor for the r-isogenies with the same $k$ such that $2^{2k-2}/3>4r$. Moreover with curves with a great rational torsion for $p=1033$ and $p=521$ we observes also some floor due to the fact that we need to repeat $O(h)$ times the interpolation. 
\todo{Describe implementation and show benchmarks. Possibly compare
  with Lercier-Sirvent (Luca has an implementation somewhere).}

\begin{acknowledgements}
  We thank many people.
\end{acknowledgements}

\bibliographystyle{alpha}
\bibliography{refs}


\appendix

\section{Galois classes in~$E[ℓ^k]$}
\label{ap:galois}

We give here the full decomposition of $E[ℓ^k]$ in Galois classes.
This is a more precise form of Proposition~\ref{prop:classes}~(v).
\begin{prop}\label{prop:orbites-l-torsion}
Let~$E$ be an elliptic curve with $ℓ$-maximal endomorphism ring.
Assume $ℓ ≠ 2$, $λ ≡ μ ≡ 1 \pmod{ℓ}$ and let~$α = v_ℓ(λ-1), β=v_ℓ(μ-1)$.
Write~$ν(x, y) = \min (x+y, x+β-1, y+α-1)$
and~$ρ(x, y) = x+y - ν(x, y) = \max (0, x-α+1, y-β+1)$.
The decomposition of the group~$E[ℓ^k]$ in Galois classes is as follows:
\begin{enumerate}
\item for~$i, j = 1, …, k-1$:
$(ℓ-1)^2 · ℓ^{ν(i,j)}$ classes of size~$ℓ^{ρ(i,j)}$;
\item for~$i = 1, …, k-1$:
$(ℓ-1) · ℓ^{\min (i, α-1)}$ classes of size~$ℓ^{\max (0, i-α+1)}$, and
$(ℓ-1) · ℓ^{\min (i, β-1)}$ classes of size~$ℓ^{\max (0, i-β+1)}$;
\item the $ℓ^2$ singleton classes of~$E[ℓ]$.
% \item \todo{le cas où $ℓ=2$}
\end{enumerate}
\end{prop}
\begin{proof}
Fix a basis~$(P, Q)$ of~$E[ℓ^k]$ such that~$π(P)=λP$, $π(Q)=μQ$.
Studying the Galois orbits of~$E[ℓ^k]$
means studying the map~$ℤ_ℓ^2 → ℤ_ℓ^2, (x, y) ↦ (λ x, μ y)$.
In other words, the orbits correspond to elements of~$ℤ_ℓ^2$
modulo the \emph{multiplicative} subgroup generated by~$(λ, μ)$.
An easy way to describe this
is to consider a \emph{multiplicative lattice} in~$(ℚ_ℓ^×)^2$.

Let~$ξ$ be a primitive $(ℓ-1)$-th root of unity in~$ℤ_ℓ$.
Then by~\cite[Théorème II.3.2]{Serre.Arith},
the map~$f(x, y, z) = ℓ^x· ξ^y· \exp (ℓ z)$
is a group isomorphism between~$ℤ × (ℤ/(ℓ-1) ℤ) × ℤ_ℓ$ and~$ℚ_ℓ^{×}$.
For~$i ∈ \bcro{0,k-1}$ and~$c ∈ ℤ/(ℓ-1)ℤ$,
let~$V(i,c)$ be the image in $ℤ/ℓ^k ℤ$ of the map~$f(k-1-i,c,–)$:
then the multiplicative structure of $V(i, c)$
is that of a principal homogeneous space under~$ℤ/ℓ^i ℤ$.
We also define~$W(i,j,c,d) = V(i, c) · P \,+\, V(j, d) · Q ⊂ E[ℓ^k]$.

Since~$λ ≡ 1 \pmod{ℓ}$, we may write~$λ = f(0,0,u\, ℓ^{α-1})$
and~$μ = f(0,0, v\, ℓ^{β-1})$
for some~$u, v ∈ ℤ_ℓ^{×}$.
This implies that the set~$W(i,j,c,d)$ is stable under Galois.
Moreover, the orbits of~$W(i,j,c,d)$ correspond bijectively to
points of a fundamental domain of the lattice~$Λ_{i,j}$ generated by
the columns of~$\smat{ℓ^i & 0 & u ℓ^{α-1}\\ 0 & ℓ^j & v ℓ^{β-1}}$,
whereas the size of each orbit is~$[(ℤ/ℓ^i ℤ)×(ℤ/ℓ^j ℤ)\::\: Λ_{i,j}]$.
By using elementary column manipulations,
we find that the covolume of~$Λ_{i,j}$ is~$ℓ^{ν(i,j)}$,
hence the point~(i) of the proposition.
(The case~$i = j = 0$ yields singleton classes in~$E[ℓ]$).

The reunion of all the sets~$W(j,i,c,d)$
is exactly the set of all~$x P + y Q$ for~$x, y ≠ 0$.
We obtain the classes of~(ii) by considering
the sets~$V(i, c) · P$ and~$V(j,d) · Q$.
\end{proof}

We now state the equivalent proposition when~$ℓ = 2$.
The proof is much the same as in the odd case.

\begin{prop}\label{prop:orbites-2-torsion}
Let~$E$ be an elliptic curve with $2$-maximal endomorphism ring.
Assume $λ ≡ μ ≡ 1 \pmod{4}$ and let~$α = v_2(λ-1), β=v_2(μ-1)$.
Write~$ν_2(x, y) = \min (x+y, x+β-2, y+α-2)$
and~$ρ_2(x, y) = x+y - ν_2(x, y) = \max (0, x-α+2, y-β+2)$.
The decomposition of the group~$E[2^k]$ in Galois classes is as follows:
\begin{enumerate}
\item for~$i, j = 1, …, k-2$:
$4 · 2^{ν_2(i,j)}$ classes of size~$2^{ρ_2(i,j)}$;
\item for~$i = 1, …, k-2$:
$4 · 2^{\min (i, α-2)}$ classes of size~$2^{\max (0, i-α+2)}$, and
$4 · 2^{\min (i, β-2)}$ classes of size~$2^{\max (0, i-β+2)}$.
\item the 16 singleton classes of~$E[4]$.
\end{enumerate}
\end{prop}
Note that if $λ$ or~$μ ≡ -1 \pmod{4}$ then
by replacing the base field by a quadratic extension,
we can always ensure that the condition $λ ≡ μ ≡ 1 \pmod{4}$ is
satisfied.
% \begin{proof}%<<<
% The proof is much the same as that of Prop.~\ref{prop:orbites-l-torsion}.
% The case~$ℓ = 2$ of~\cite[Théorème II.3.2]{Serre.Arith}
% states that $f(x,y,z) = 2^x · (-1)^y · \exp (4z)$
% is an isomorphism between~$ℤ × (ℤ/2ℤ) × ℤ_2$ and~$ℚ_2^×$.
% Let~$V(i, c)$ be the image modulo~$2^k$ of~$f(k-2-i, c, -)$;
% then $V(i, c)$ has cardinal~$2^{i}$ for~$i ≥ 0$,
% while~$V(-1, 0) = V(-1, 1)$ is the singleton $\acco{2^{k-1}}$.
% 
% Let~$P, Q$ be diagonal generators of~$E[2^{k}]$
% and define~$W(i,j,c,d) = V(i,c) P + V(j,d) Q$;
% also write~$λ = f(0,0,2^{α-2} u)$, $μ = f(0,0,2^{β-2} v)$.
% Then the Galois orbits in~$W(i,j,c,d)$ correspond to $ℤ_2^2$ modulo
% the lattice~$Λ_{i,j} = \smat{2^{i}&0&2^{α-2} u\\0&2^{j}&2^{β-2} v}$.
% We find that $Λ_{i,j}$ has covolume~$2^{ν_2(i,j)}$.
% This gives the orbits of point~(i) of the proposition,
% while the case~$i = j = 0$ gives singleton orbits.
% 
% By considering the orbits of~$V(i,c) P + ε Q$ and~$ε P + V(i,c) Q$
% for~$ε ∈ \acco{0, 2^{k-1}}$, we obtain the orbits
% of point~(ii).
% \end{proof}%>>>
%>>>1
\affiliationone{\todo{Luca et Cyril}}
\affiliationtwo{Jérôme Plût\\
ANSSI\\
51, boulevard de La Tour-Maubourg\\
75007 Paris\\
France}
\affiliationthree{\todo{Éric}}
\end{document}
%  LocalWords:  isogeny morphisms Isogenies isogenies isogenous
%  LocalWords:  cardinality bijection Couveignes automorphism

% vim: ts=2:
%  LocalWords:  Frobenius endomorphism precomputation morphism
%  LocalWords:  subproduct factorizations
