\documentclass{article}
\usepackage[utf8]{inputenc}
\usepackage[american]{babel}
\usepackage[T1]{fontenc}
\usepackage{amssymb,amsthm,amsmath,amsfonts}
\usepackage{algorithmic}
\usepackage{algorithm}
\usepackage{tikz}
\usepackage{nicefrac}

\theoremstyle{plain}
\newtheorem{thm}{Theorem}
\newtheorem{lem}[thm]{Lemma}
\newtheorem{cor}[thm]{Corollary}
\newtheorem{prop}[thm]{Proposition}
\theoremstyle{definition}
\newtheorem{defi}[thm]{Definition}
\theoremstyle{remark}
\newtheorem{rem}[thm]{Remark}
\newtheorem{exe}[thm]{Example}

\title{Explicit isogenies in quadratic time in any characteristic}
\author{Cyril Hugounenq}

\begin{document}

\begin{abstract}
The problem we will consider here is the computation of an isogeny between two elliptics curves with the knowledge of the domain and the codomain of the isogeny and it's degree $r$. Couveignes's algorithm is an algorithm which solves this problem in $O(r^2)$ operations using the $p$-torsion. We want to extend the method used by Couveignes  We try to adapt his method here to the case of the $2$ torsion and more generally to the $\ell$ torsion, thus we propose an alternative for medium characteristic with this algorithm.
\end{abstract}

\section*{Proposed notation}

This section is for internal reference only: erase after the paper has
stabilized.

\begin{itemize}
\item $\mathbb{F}_q$ is the field we are working on
\item $\ell$ is for the $\ell$ torsion we are working on
\item $r$ is the degree of the isogeny we want to compute
\item $k$ is the integer such that $\ell^{2k}>4r+1$
\item we thus work with a tower which has for top level $F_{q^{\ell^k}}$
\item $E$ is for ordinary elliptic curves defined over the finite field $\mathbb{F}_q$
\item $\mathcal{O}$ (resp. $\mathcal{O}_x$) is the notation for the endomorphism ring associated (up to isomorphism) to $E$ (resp. $E_x$)
\item $K$ is the notation for the imaginary quadratic field in which $\mathcal{O}$ is defined
\item $d_K$ is the negative integer such that $K=\mathbb{Z}[d_K]$  
\end{itemize}
\section{Introduction}
%Rappel sur l'algorithme de Couveignes, dire que basiquement il interpole un groupe de points sur un autre, puis fait de la reconstruction rationnelle et teste alors si son résultat est correct, si ce n'est pas le cas il change les groupes et recommence. 

Couveignes's algorithm \cite{couveignes96} is an algorithm which computes an isogeny using the $p$-torsion. This algorithm takes advantage of the cyclic group structure of the $p$ torsion to interpolate $p^k$ torsion points of one curve on another, with $k$ chosen such that $p^k>4r$. Couveignes's algorithm also take into account that if an isogeny send a point on an other then it also must respect the same property for the multiple points. With the interpolation polynomial Couveignes do a rational reconstruction of the isogeny, if the rational reconstruction don't work he tried with another couple of generators to do another interpolation. The complexity of this algorithm is of $O(r^2)$
\newline
We try to adapt his method here to the case of the $2$ torsion for ordinary elliptic curves and more generally to the $\ell$ torsion with $2 \wedge r ( \ell \wedge r)$.

Our goal is to reduce the number of points to interpolate with operations which are not so much costly. Thus we consider the Frobenius endomorphism to give us knowledge on the points we want to interpolate. We have considered the graph of isogeny (called volcano) which is linked to the group structure (see \cite{MiretMRV05}, \cite{IonicaJ10} ) of the curve for our study. First we have to determined which knowledge we obtain thanks to the Frobenius. The Frobenius allow us to determine horizontal isogenies according to the "notations" from \cite{Kohel} and \cite{volcano}, and thus we can associate to those horizontal isogenies their kernel and then one generator of their kernel. It will be those generators that we will use for the interpolation. To speed up the interpolation computation we will use techniques such as one used in \cite{enge+morain03} where we take into account the action of the Frobenius. In the end we will obtain a computational complexity of $O(r^2)$.

\section{Implantation}
%Il faut aborder le fait que l on travaille sur des constructions de tours 2 adique de la meme maniere que le papier de Javad et Eric mais cela ne devrait peut etre pas etre precise lors d une section

\section{Study of the Frobenius on the crater}
%Dans cette section on montre les résultats que nous permettent d'obtenir le frobenius sur le cratère
\subsection{Horizontal basis}
%Dans cette section on montre comment on construit une base horizontale de la 2(\ell) torsion
\subsection{Proof of the algorithm}
%Dans cette section on montre que une base horizontale est envoyee sur une base horizontale

\section{Interpolation}
%Dans cette section on montre les outils employes pour diminuer le cout de l interpolation 

\section{Reduction to the crater}
%Dans cette section on montre que l on peut reduire le probleme a l etude de deux courbes sur le cratere

\bibliographystyle{plain}
\bibliography{refs}

\end{document}
