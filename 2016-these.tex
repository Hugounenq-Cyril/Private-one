\documentclass[10pt,a4paper]{book}
\usepackage[utf8]{inputenc}
\usepackage[T1]{fontenc}
\usepackage{hyperref}
\usepackage[francais]{babel}
\usepackage{hyperref}
\usepackage{amssymb,amsthm,amsmath,amsfonts}
\usepackage{units}
\usepackage{lmodern}
\usepackage{graphicx}
\bibliographystyle{plain}
\usepackage{epsfig}
\usepackage{epstopdf}
\usepackage{makeidx}
\usepackage{gloss}
\usepackage{xcolor}
\def\glossname{Glossaire}
\usepackage{ragged2e}
\usepackage{changepage}
\usepackage{tikz}
\usetikzlibrary{calc}
\usetikzlibrary{fixedpointarithmetic}

\usepackage{fancyhdr}
\pagestyle{fancy}




\theoremstyle{plain}
\newtheorem{thm}{Théorème}
\theoremstyle{definition} 
\newtheorem{lem}[thm]{Lemme}
\theoremstyle{definition} 
\newtheorem{cor}[thm]{Corollaire}
\theoremstyle{definition} 
\newtheorem{prop}[thm]{Proposition}
\theoremstyle{definition} 
\newtheorem{defi}[thm]{Définition}
\theoremstyle{remark} 
\newtheorem{rem}[thm]{Remarque}
\theoremstyle{remark} 
\newtheorem{exe}[thm]{Exemple}

\newcommand{\defeq}{\mathrel{\mathop:}=}

\author{Hugounenq Cyril}
\title{Arithmétique Rapide appliqué à la Géométrie et à la Cryptologie}
\makeindex
\makegloss


\makeatletter
\def\footrule{{
\vskip-\footruleskip\vskip-\footrulewidth
\color{\footrulecolor}
\hrule\@width\headwidth\@height
\footrulewidth\vskip\footruleskip
}}
\makeatother

%%% BEGIN DOCUMENT
\begin{document}
\fancyhf{}%␣efface␣tout␣ce␣qu'il␣y␣avait␣avant
\fancyhead[L]{\nouppercase\leftmark}%␣LO␣=␣gauche/impair␣;␣RE␣=␣droite/pair
\fancyhead[R]{}
\fancyfoot[C]{\thepage}%␣C␣=␣centré
\fancyfoot[L]{}




\renewcommand{\headrulewidth}{1pt}% 1pt header rule
\renewcommand{\headrule}{\hbox to\headwidth{%
  \color{orange}\leaders\hrule height \headrulewidth\hfill}}
  
\renewcommand{\footrulewidth}{1pt}% 1pt header rule
\newcommand{\footrulecolor}{blue}


%\small
\setlength{\parskip}{-1pt plus 1pt}

\renewcommand{\abstracttextfont}{\normalfont}
%\abstractintoc
%\begin{abstract} 
%Text 
%\end{abstract}
%\abstractintoc
%\renewcommand\abstractname{R\'esum\'e}
%\begin{abstract} \selectlanguage{french}
%Texte 
%\end{abstract}
\selectlanguage{french}% french si rapport en français



%\cleardoublepage

%\tableofcontents* % the asterisk means that the table of contents itself isn't put into the ToC
\normalsize

\mainmatter
% \SingleSpace

\chapter{Rappels théoriques}



\section{Théorie des ordres dans un corps quadratique imaginaire}

\section{Courbes Elliptiques}

Dans ce chaptire nous allons faire quelques rappels sur les courbes elliptiques. Pour approfondir le sujet le lecteur pourra consulter \cite{Silv1} ou encore \cite{Washington:2008}

\subsection{\'Equations de Weierstrass}

Soit $\mathbb{K}$ un corps, on dénote par $\overline{\mathbb{K}}$ sa cloture algébrique. On rappelle tout d'abord  la définition d'un espace projectif sur $\mathbb{K}$:

\begin{defi}
On appelle espace projectif de dimension $2$ sur un corps $\mathbb{K}$ et on note $\mathbb{P}^2(\mathbb{K})$, l'ensemble des classes $(X:Y:Z)$ de la relation d'équivalence définie comme suit:
\begin{equation*}
\forall (X,Y,Z) \in \mathbb{K}^3 \setminus (0,0,0), (X,Y,Z) \equiv (X',Y',Z') \Leftrightarrow \exists c \in \mathbb{K}^*, X'=cX, Y'=cY, Z'=cZ.
\end{equation*}
\end{defi}



On définit sur $\mathbb{K}$ une équation de Weierstrass comme une équation de la forme:

\begin{defi}
Une courbe elliptique $E$ peut être définie comme la variété projective associée à l'équation :
\begin{equation}
\label{eq:weierstrass-proj}
E:Y^2Z+a_1XYZ+a_3YZ^2=X^3+a_2X^2Z+a_4XZ^2+a_6Z^3
\end{equation}
avec $a_1,a_2,a_3,a_4,a_6$ des éléments de $\overline{\mathbb{K}}$. Le point $[0:1:0]$ est appelé le \emph{point à l'infini} et sera noté $0_E$. Dès lors que l'on a $a_1,a_2,a_3,a_4,a_6$ des éléments de $\mathbb{K}$ alors on dit que la courbe est définie sur $\mathbb{K}$.
\end{defi}

L'équation \ref{eq:weierstrass-proj} est dite forme homogène de l'équation de Weierstrass. En appliquant le changement de variables suivant: $x=\frac{X}{Z}$, $y=\frac{Y}{Z}$, on passe alors en coordonnées affines, on obtient l'équation de Weierstrass suivante:

\begin{equation}
\label{eq:weierstrass-red}
E:y^2+a_1xy+a_3y=x^3+a_2x^2+a_4x+a_6
\end{equation}
appelée équation affine de Weierstrass. La courbe $E$ est alors définie comme étant l'ensemble des points solutions de \ref{eq:weierstrass-red} avec le point à l'infini $O_E$.

 On définit aussi:
\begin{equation*}
b_2=a_1^2+4a_2, b_4=a_1a_3+2a_4, b_6=a_3^2+4a_6, b_8=a_1^2a_6-a_1a_3a_4+4a_2a_6+a_2a_3^2-a_2^4
\end{equation*}
On définit alors le discriminant $\Delta$ de l'équation de Werierstrass \ref{eq:weierstrass-red}:
\begin{equation*}
\Delta = -b_2^2b_8-8b_4-27b_6^2+9b_2b_4b_6
\end{equation*}

La courbe $E$ admet un unique point singulier si et seulement si $\Delta$ est nul, la courbe est non singulière si et seulement si $\Delta$ est non nul (\cite[Prop. III.1.4]{Silv1}).

%L'équation de Weierstrass \ref{eq:weierstrass} est dite non-singulière si et seulement si $\Delta$ est non nul.
On peut alors énoncer une autre définition de courbe elliptique.

\begin{defi}
Une courbe elliptique $E$ sur $\mathbb{K}$ est une courbe définie par une équation de Weierstrass \ref{eq:weierstrass-red} non singulière.
\end{defi}

On définit: 
\begin{equation*}
j=\frac{(b_2^2-24b_4)^3}{\Delta}
\end{equation*}
La constante $j$ est appelée \emph{$j$-invariant} de la courbe.

Une question naturelle est de se demander à quel point l'équation de Weierstrass définit de façon unique une courbe elliptique. En supposant que l'on considère les changements de variable qui préservent le point à l'infini $O_E$, il est montré \cite[III.3.1b]{Silv1} que le seul changement de variable préservant la forme de l'équation de Weierstrass et $O_E$ est:

\begin{equation*}
x=u^2x'+r    \quad  y=u^3y'+u^2sx'+t
\end{equation*}
avec $u,r,s,t$ des éléments de $\overline{\mathbb{K}}$ et $u$ différent de $0$. En actualisant alors le calcul des $a'_i$ et des calculs subséquents on justifie alors la dénomination de \emph{$j$-invariant}.
\newline


On définit alors l'ensemble des points d'une courbe elliptique $E$ définie sur $\mathbb{K}$  noté $E(\mathbb{K})$.

\begin{defi}
Soit une courbe $E$ définie sur $\mathbb{K}$ on définit $E(\mathbb{K})$ comme étant l'ensemble:
\begin{equation*}
E(\mathbb{K})=\{(X:Y:Z)\in \mathbb{P}^3(\mathbb{K}) \setminus (0,0,0) : Y^2Z+a_1XYZ+a_3YZ^2=X^3+a_2X^2Z+a_4XZ^2+a_6Z^3 \}
\end{equation*}
Le seul point correspondant à $Z=0$ est le point à l'infini $O_E=(0:1:0)$, chaque classe différente de $O_E$ a un unique représentant $(X:Y:1)$ ce qui nous donne la notation équivalente suivante:
\begin{equation*}
E(\mathbb{K})=\{(x,y)\in \mathbb{K}^2  : y^2+a_1xy+a_3y=x^3+a_2x^2+a_4x+a_6 \} \cup {O_E}
\end{equation*}
\end{defi}

On rappelle que si l'on note $f(x,y)=0$ l'équation affine de Weierstrass d'une courbe elliptique $E$ définie sur $\mathbb{K}$, alors son corps de fonction $\mathbb{K}(E)$ est le corps des fractions suivant:
\begin{equation*}
\mathbb{K}(E) \defeq \mathbb{K}[x,y]/(f(x,y)).
\end{equation*}

\subsection{Loi de groupe}
%\begin{figure}
%\begin{tikzpicture}[domain=-10/3:4.5,samples=100,yscale=1/2]
%\begin{scope}[fixed point arithmetic]
%\coordinate (Q) at (2, -4/3) {};
%\coordinate (-Q) at (2, 4/3) {};
%\coordinate (-2Q) at (-1079/576, 59653/13824);
%\coordinate (3Q) at (2319138/4977361, 27920423524/11104492391);
%\draw plot (\x,{sqrt(\x*\x*\x+(3-100/9)*\x+10)});
%\draw plot (\x,{-sqrt(\x*\x*\x+(3-100/9)*\x+10)});
%\draw (-2Q) node[/dot,label=$P$] {}
%              (3Q) node[/dot,label=$Q$] (G) {}
%              (-Q) node[/dot,label=$R$] {};
%\draw[label position=right,label distance=5pt] (Q) node[/dot,label=$P+Q$] {};
%\end{scope}
%\end{tikzpicture}
%\end{figure}
%
%\begin{figure}
%
%\begin{tikzpicture}[domain=-10/3:4.5,samples=100,yscale=1/2]
%          \begin{scope}[fixed point arithmetic]
%            \begin{scope}[fixed point arithmetic]
%              \coordinate (Q) at (2, -4/3) {};
%              \coordinate (-Q) at (2, 4/3) {};
%              \coordinate (-2Q) at (-1079/576, 59653/13824);
%              \coordinate (3Q) at (2319138/4977361, 27920423524/11104492391);
%            \end{scope}
%
%            \draw plot (\x,{sqrt(\x*\x*\x+(3-100/9)*\x+10)});
%            \draw plot (\x,{-sqrt(\x*\x*\x+(3-100/9)*\x+10)});
%
%            \draw[shorten >=-2cm, shorten <=-2cm] (-2Q) -- (-Q);
%            \draw[dashed] (-Q) -- (Q);
%
%            \begin{scope}[/dot/.style={draw,circle,inner sep=1pt,fill},
%              every label/.style={inner sep=5pt,font=\scriptsize}]
%              \draw (-2Q) node[/dot,label=$P$] {}
%              (3Q) node[/dot,label=$Q$] (G) {}
%              (-Q) node[/dot,label=$R$] {};
%              \draw[label position=right,label distance=5pt] (Q) node[/dot,label=$P+Q$] {};
%            \end{scope}
%          \end{scope}
%
%          \begin{scope}[xshift=10cm]
%            \draw plot (\x,{sqrt(\x*\x*\x+(3-100/9)*\x+10)});
%            \draw plot (\x,{-sqrt(\x*\x*\x+(3-100/9)*\x+10)});
%
%            \begin{scope}[fixed point arithmetic]
%              \coordinate (-Q) at (2, 4/3) {};
%              \coordinate (-2Q) at (-1079/576, 59653/13824);
%              \path let \p1 = (-2Q) in coordinate (2Q) at (\x1,-\y1);
%            \end{scope}
%
%            \draw[shorten >=-2cm, shorten <=-2cm] (-Q) -- (2Q);
%            \draw[dashed] (-2Q) -- (2Q);
%
%            \begin{scope}[/dot/.style={draw,circle,inner sep=1pt,fill},
%              every label/.style={inner sep=5pt,font=\scriptsize}]
%              \draw (-Q) node[/dot,label=$P$] {}
%              (-2Q) node[/dot,label=$2P$] (G) {};
%              \draw[label position=below] (2Q) node[/dot,label=$R$] {};
%            \end{scope}
%          \end{scope}
%        \end{tikzpicture}
%
%\end{figure}

On peut définir sur $E(\mathbb{K})$ une loi de groupe abélienne à l'aide de la méthode dite de la \emph{corde et de la tangente}

\begin{defi}[Loi d'addition]
Soient $P,Q \in E(\mathbb{K})$, $L$ la droite reliant $P$ à $Q$ (ou tangente si $P=Q$) et $R$ le troisième point d'intersection de $L$ avec $E$. Soit $L'$ la droite reliant $R$ à $O_E$ cette droite intersecte alors la courbe en un troisième point $P+Q$.
\end{defi}

Une preuve que cette loi définit bien une loi de groupe abélienne est montrée dans \cite[III.2.2]{Silv1}.

En utilisant la notation $(x_P,y_P)$ pour les coordonnées affines d'un point $P$ de la courbe, on obtient alors pour les points distincts $P,Q$ de $E(\mathbb{K})$ les formules suivantes:
\begin{itemize}
\item $P+O_E=O_E+P=(x_P,y_P)$
\item $-P=(x_P,-y_P-a_1x_P-a_3)$
\item En notant $\lambda= \begin{cases}
\frac{y_{Q}-y_{P}}{x_{Q}-x_{P}} & \textit{si } P\neq Q\\
\frac{3x_{P}^{2}+2a_{2}x_{P}+a_{4}-a_{1}y_{P}}{2y_{P}+a_{1}x_{P}+a_{3}} & \textit{si } P=Q
\end{cases}\ $,

on peut alors exprimer les coordonnées de $P+Q$:
\begin{equation*}
\begin{alignedat}{1}
& x_{P+Q}=\lambda^2 +a_1\lambda-a_2-x_P-x_Q,\\
& y_{P+Q}=-(\lambda +a_1)x_{P+Q}-y_{P}-\lambda x_P-a_3
\end{alignedat}
\end{equation*}
\end{itemize}




\paragraph*{Multiplication par $m$}\   {
\newline


\'A l'aide de la loi additive on peut définir la multiplication scalaire par l'entier naturel $m$ comme suit:

\begin{equation}
\begin{alignedat}{1}
[m]: E(\mathbb{K}) & \rightarrow E(\mathbb{K}) \\
P & \mapsto \underbrace{P+ \cdots +P}_{m fois} 
\end{alignedat}
\end{equation}

\begin{defi}
On appelle points de \emph{$m$-torsion} l'ensemble des points de $E$ qui sont envoyés sur l'élément neutre $0_E$ par $[m]$. Cet ensemble est noté $E[m]$.
\end{defi}
%On a ici un exemple d'endomorphismes de la courbe $E$.
}

\subsection{Isogénies}

\begin{defi}
Soient $E_1$ et $E_2$ deux courbes elliptiques, une isogénie de $E_1$ dans $E_2$ est un morphisme de groupes:
\begin{equation*}
\varphi:E_1 \rightarrow E_2
\end{equation*}
tel que $\varphi(O_{E_1})=O_{E_2}$.
\end{defi}
On dira que $E_1$ et $E_2$ sont isogénes si il existe une isogénie $\varphi$ non nulle qui va de $E_1$ dans $E_2$. Une question naturelle est de se demander comment ces isogénies agissent sur les groupes de points d'une courbe, le théorème suivant répond à cette question.

\begin{thm}
Soient $\varphi:E_1 \rightarrow E_2$ une isogénie, $P,Q$ deux points de $E_1$. Alors:
\begin{equation*}
\varphi(P+Q)=\varphi(P)+\varphi(Q)
\end{equation*}
\end{thm}

\begin{proof}
Voir \cite[III.4.8]{Silv1}
\end{proof}

\begin{exe}
La multiplication par $m$ est par exemple un homomorphisme:
\begin{equation*}
[m](P+Q)=[m]P+[m]Q
\end{equation*}
\end{exe}

\begin{thm}
Soit $\varphi$ une isogénie non nulle définie sur un corps $\mathbb{K}$ qui va de $E_1$ dans $E_2$, alors:
\begin{itemize}
\item $\ker(\varphi)$ est un groupe fini,
\item $\varphi$ est surjective si $\mathbb{K}$ est un corps algébriquement clos.
\end{itemize}
\end{thm}

\begin{proof}
Voir \cite[II.2]{Silv1}
\end{proof}

Lorsque l'on a $\varphi$ surjective alors on peut définir sur les corps de fonctions associés à $E_1$ et $E_2$ l'injection suivante:
\begin{equation*}
\begin{alignedat}{1}
\varphi^*: \mathbb{K}(E_2) & \rightarrow \mathbb{K}(E_1) \\
f & \mapsto f  \circ \varphi
\end{alignedat}
\end{equation*}

\begin{defi}
Soit $\mathbb{K}$ un corps algébriquement clos, $\varphi$ une isogénie définie sur $\mathbb{K}$ qui va de $E_1$ dans $E_2$, alors $\varphi$ est dite \emph{séparable} ou \emph{inséparable} selon les propriétés de $[K(E_1):\varphi^*(\mathbb{K}(E_2))]$. On note alors \emph{$\deg_s{\varphi}$}, \emph{$\deg_i{\varphi}$} les degrés correspondants et l'on a $\deg{\varphi}=\deg_s{\varphi}\deg_i{\varphi}$
\end{defi}

Lorsque $\mathbb{K}$ n'est pas algébriquement clos on définit le degré de l'isogénie comme étant le degré de l'isogénie définie sur la cloture algébrique de $\mathbb{K}$.


On désignera alors pour un entier $\ell$, une isogénie de degré $\ell$ par: $\ell$-isogénie.

\begin{thm}
Soit $\varphi$ une isogénie non constante de $E_1$ dans $E_2$, alors:
\begin{itemize}
\item pour tout point $Q$ de $E_2$ on a $|\varphi^{-1}(Q)|=\deg(\varphi)$,
\item si $\varphi$ est séparable alors $\varphi$ est non ramifiée et on a $\deg_s(\varphi)=\ker(\varphi)$.
\end{itemize}
\end{thm}

\begin{proof}
Voir \cite[III.4.10]{Silv1}
\end{proof}

Une conséquence directe de ce théorème est le résultat suivant:

\begin{cor}
Soit $E_1$ une courbe elliptique, alors il y a une correspondance bijective entre les isogénies séparables qui ont pour domaine de définition $E_1$ et les sous-groupes de $E_1$:
\begin{equation*}
\begin{alignedat}{1}
(\varphi :E_1\rightarrow \varphi(E_1)) &\mapsto  \ker(\varphi)  \\
 (E_1\rightarrow\nicefrac{E_1}{C})  &\  \reflectbox{$\mapsto$} \ C 
\end{alignedat}
\end{equation*}
\end{cor}

L'ensemble des isogénies de $E_1$ vers $E_2$ est appelé homomorphismes et est noté \emph{$\mathrm{Hom}(E_1,E_2)$}. L'ensemble des isogénies de $E_1$ vers $E_1$ est appelé endomorphismes et est noté \emph{$\mathrm{End}(E_1)=\mathrm{Hom}(E_1,E_1)$}, les éléments de $\mathrm{End}(E_1)$ inversibles sont appelés automorphismes et sont notés \emph{$\mathrm{Aut}(E_1)$}. Une isogénie définie sur $\mathbb{K}$ sera dite \emph{$\mathbb{K}$-rationnelle} ou simplement \emph{rationnelle} quand il n'y a pas d'ambiguïtés.

\begin{defi}[Isogénie duale]
Soit $\varphi$ une isogénie définie sur $\mathbb{K}$ d'une courbe elliptique $E_1$ vers une courbe elliptique $E_2$, alors l'unique isogénie $\widehat{\varphi}$ qui va de $E_2$ ver $E_1$ telle que $\widehat{\varphi}\circ\varphi=[\deg(\varphi)]$ est appelée \emph{isogénie duale}.
\end{defi}

\begin{proof}
Voir \cite[III.6.1]{Silv1}
\end{proof}

\begin{thm}
Soit $\varphi:E_1 \rightarrow E_2$ une isogénie.
\begin{enumerate}
\item Soit $m=\deg(\varphi)$, alors:
\begin{equation*}
\widehat{\varphi}\circ \varphi=[m]_{E_1} \textit{ et }  \varphi \circ \widehat{\varphi}=[m]_{E_2}.
\end{equation*}
\item Soit $\lambda:E_2 \rightarrow E_3$ une autre isogénie, alors
\begin{equation*}
\widehat{\lambda \circ \varphi}=\widehat{\lambda} \circ \widehat{\varphi}.
\end{equation*} 
\item Soit $\phi:E_1 \rightarrow E_2$ une autre isogénie, alors
\begin{equation*}
\widehat{\varphi+\phi}=\widehat{\varphi}+\widehat{\phi}.
\end{equation*}
\item Pour tout $m$ entier relatif,
\begin{equation*}
[\widehat{m}]=[m] \textit{ et } \deg[m]=m^2.
\end{equation*}
\item $\deg( \varphi)=\deg(\widehat{\varphi})$.
\item $\widehat{\widehat{\varphi}}=\varphi$
\end{enumerate}
 
\end{thm}

\begin{proof}
Voir \cite[Theorem III.6.2]{Silv1}
\end{proof}

\begin{cor}
Soit $E$ une courbe elliptique définie sur $\mathbb{K}$ et $m \in \mathbb{Z}$ avec $m \neq 0$,
\begin{enumerate}
\item Si $m \neq 0$ dans  $\mathbb{K}$ alors
\begin{equation*}
E[m]=\nicefrac{\mathbb{Z}}{m\mathbb{Z}} \times \nicefrac{\mathbb{Z}}{m\mathbb{Z}}. 
\end{equation*}
\item Si $\mathrm{car}(\mathbb{K})=p>0$ alors:
\begin{equation*}
\begin{alignedat}{1}
&\textit{Soit } E[p^e]=\{O_E\} \textit{ pour tout } e \geqslant 1 \\
&\textit{Soit } E[p^e]=\nicefrac{\mathbb{Z}}{p^e\mathbb{Z}} \textit{ pour tout } e \geqslant 1
\end{alignedat}
\end{equation*}
\end{enumerate}
\end{cor}

\begin{proof}
Voir \cite[Corrolary III.6.4]{Silv1}.
\end{proof}

\begin{defi}
Une courbe elliptique $E$ définie sur un corps $\mathbb{K}$ de caractéristique $p$ différente de $0$ est dite \emph{supersingulière} si $E[p^e]=\{O_E\}$ pour tout $e$ entier naturel non nul, elle est dite \emph{ordinaire} sinon. 
\end{defi}

\begin{defi}[Module de Tate]
Soit $E$ une courbe elliptique et $\ell$ un nombre premier. Le \emph{$\ell$-adique module de Tate} de $E$ est le groupe:
\begin{equation*}
T_{\ell}(E)=\varprojlim_nE[\ell^n],
\end{equation*}
la limite projective étant définie par rapport à l'application naturelle
\begin{equation*}
E[\ell^{n+1}] \overset{[\ell]}{\rightarrow} E[\ell^n].
\end{equation*}
\end{defi}
Comme chacun des $E[\ell^n]$ est un $\nicefrac{\mathbb{Z}}{\ell^n\mathbb{Z}}$-module, on observe alors que le module de Tate a une structure naturelle de $\mathbb{Z}_{\ell}$-module.
\begin{prop}
En tant que $\mathbb{Z}_{\ell}$ module, le module de Tate a la structure suivante:
\begin{equation*}
T_{\ell} \cong 
\begin{cases} 
\mathbb{Z}_{\ell} \times \mathbb{Z}_{\ell} &\textit{ si } \ell \neq \mathrm{car}(\mathbb{K}). \\
\mathbb{Z}_{\ell} &\textit{ si } \ell=\mathrm{car}(\mathbb{K})  \textit{et E est ordinaire}. \\
\{O_E\} &\textit{ si } \ell=\mathrm{car}(\mathbb{K})  \textit{et E est supersingulière}.
\end{cases}
\end{equation*}
\end{prop}

%\begin{alignedat}{2}
%& T_{\ell} \cong \mathbb{Z}_{\ell} \times \mathbb{Z}_{\ell} &\textit{ si } \ell \neq \mathrm{car}(\mathbb{K}).& \\
%& T_{p} \cong \{O_E\} \textit{ or } \mathbb{Z}_{p} \times \mathbb{Z}_{\ell} &\textit{ si }  p =\mathrm{car}(\mathbb{K})&>0.
%\end{alignedat}


\subsection{Endomorphismes}

Soit $E$ une courbe elliptique, alors si l'on adjoint à $\mathrm{End}(E)=\mathcal{O}$, la composition et l'addition d'endomorphismes cela forme un anneau. De plus la multiplication scalaire fournit un homomorphisme de $\mathbb{Z}$ dans $\mathrm{End}(E)$. Cependant $\mathrm{End}(E)$ n'est pas nécessairement réduit à $\mathbb{Z}$.

\begin{defi}
Une courbe elliptique $E$ est dite à multiplication complexe si $\mathrm{End}(E)$ n'est pas réduit à $\mathbb{Z}$.
\end{defi}
On a le résultat suivant sur la structure de $\mathrm{End}(E)$.

\begin{thm}
$\mathrm{End}(E)$ est soit isomorphe à $\mathbb{Z}$, soit isomorphe à un ordre dans un corps quadratique imaginaire, soit isomorphe à un ordre dans une algèbre de quaternions 
\end{thm}

\begin{proof}
Voir \cite[Corollary III.9.4]{Silv1}
\end{proof}

\section{Courbes elliptiques définies sur un corps fini}

Soit $p$ un nombre premier, on notera $\mathbb{F}_q=\mathbb{F}_{p^e}$ le corps fini à $q$ éléments défini à isomorphisme près. Travailler sur un corps fini entraîne le fait que le nombre de points d'une courbe elliptiqe est fini, ainsi nous travaillons avec un groupe abélien fini. Afin d'étudier ce groupe et en particulier d'établir sa cardinalité nous allons introduire l'endomorphisme de Frobenius. Dans la suite de ce document nous considérerons que les courbes elliptiques mentionnées sont définies sur un corps fini. 

\subsection{L'endomorphisme de Frobenius}
\begin{defi}
Soit $E$ une courbe définie sur $\mathbb{F}_q$, l'application: 
\begin{equation*}
\begin{alignedat}{1}
\pi :E &\mapsto  E  \\
 (x,y)  &\mapsto (x^q,y^q)  
\end{alignedat}
\end{equation*}
est un homomorphisme de la courbe et est appelée \emph{endomorphisme de Frobenius}. 
\end{defi}

\begin{prop}
Soit $E$ une courbe elliptique définie sur $\mathbb{F}_q$, soit $\pi$ l'endomorphisme de Frobenius relatif à $\mathbb{F}_q$. Les conditions suivantes sont équivantes:
\begin{enumerate}
\item $E$ est supersingulière,
\item $E[p^e]=\{O_E\}$ pour tout $e > 0$,
\item la duale $\widehat{\pi}$ de l'endomorphisme de Frobenius est purement inséparable,
\item la trace de l'endomorphisme de Frobenius est divisible par $p$,
\item l'anneau d'endomorphisme $\mathrm{End}(E)$ est isomorphe à un ordre dans une algèbre de quaternions.
\end{enumerate}
si ces conditions ne sont pas vérifiées alors on a les conditions équivalentes suivantes:
\begin{enumerate}
\item $E$ est ordinaire,
\item $E[p^e]=\mathbb{Z}/p^e\mathbb{Z}$ pour tout $e > 0$,
\item la duale $\widehat{\pi}$ de l'endomorphisme de Frobenius est séparable,
\item la trace de l'endomorphisme de Frobenius est première avec $p$,
\item l'anneau d'endomorphisme $\mathrm{End}(E)$ est isomorphe à un ordre dans un corps de nombre quadratique imaginaire.
\end{enumerate}
\end{prop}

\begin{proof}
Voir \cite[Theorem V.3.1]{Silv1}
\end{proof}

L'endomorphisme de Frobenius relatif à $\mathbb{F}_q$ admet pour polynôme caractéristique:
\begin{equation}
\pi^2-t\pi+q=0
\end{equation}
avec $t$ la trace de l'endomorphisme de Frobenius. Cette équation a un rôle important dans le calcul de cardinalité de courbes elliptiques, en particulier dans l'algorithme de Schoof.

\begin{prop}
Soit $E$ une courbe elliptique définie sur $\mathbb{F}_q$, soit $t$ la trace de l'endomorphisme de Frobenius relatif à $ \mathbb{F}_q$ alors on a:
\begin{equation*}
|E(\mathbb{F}_q)|=q+1-t.
\end{equation*}
\end{prop}

\begin{proof}
Voir \cite[Theorem V.1.1]{Silv1}
\end{proof}

\begin{thm}[Hasse]
Soit $E$ une courbe définie sur $\mathbb{F}_q$, alors 
\begin{equation}
|t| \leqslant 2 \sqrt{q}
\end{equation}
\end{thm}

\begin{proof}
Voir \cite[Theorem V.1.1]{Silv1}
\end{proof}

On a alors le résultat suivant sur la structure des points d'une courbe elliptique:

\begin{prop}
Le groupe abélien $E(\mathbb{F}_q$) est isomorphe à $\nicefrac{\mathbb{Z}}{n_1 \mathbb{Z}} \times \nicefrac{\mathbb{Z}}{n_2 \mathbb{Z}}$ avec $n_2 \mid n_1$ et $n_2 \mid q-1$
\end{prop}

\begin{proof}
Voir \cite[Theorem 4.1]{Washington:2008}
\end{proof}



\subsection{Courbes définies sur $\mathbb{C}$}
\subsection{Multiplication complexe}
\subsection{Polynôme modulaire}
\subsection{Bases (loi additive, isogénie, frobenius, module de Tate}
\subsection{S.E.A}
\subsection{Lien entre isogénie et son action sur le module de Tate}

\chapter{Construction efficaces de tours $\ell$-adiques}
\section{$\ell$= 2 [Doliskani-Schost]}
\section{$\ell \neq 2$ [De Feo-Doliskani-Schost]}


\chapter{Isogénies sur les corps finis}

\section{Calculs d'isogénie}
\subsection{Algorithme de Lercier Sirvent (+ resultat de Tristan)}
\subsection{Algorithme de Couveignes (parler de ses variantes)}

\subsection{Volcans d'isogénies}
\subsection{Travaux de Kohel, Fouquet-Morain}
Mettre l'algorithme de Fouquet Morain
\subsection{Travaux de Miret & Al. et Ionica-Joux}


\section{Frobenius et ses applications sur un volcan d'isogénies}
\subsection{Base diagonale, horizontale}
\subsection{Orbites selon le Frobenius}
\subsection{Amélioration de l'algorithme de Couveignes}

\section{Améliorations éventuelles à l'aide du pairing}

\bibliographystyle{plain}
\bibliography{Biblio}
\end{document}
