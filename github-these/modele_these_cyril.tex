\documentclass[a4paper, titlepage, 11pt]{book}

\usepackage{graphicx}
\usepackage{multicol}
\usepackage[utf8]{inputenc} 
\usepackage[T1]{fontenc}
\usepackage{lmodern}
\usepackage[francais]{babel} 
\usepackage[top = 3.5 cm , bottom = 3 cm , left = 3cm , right = 3 cm]{geometry}
\usepackage{fancyhdr}
\usepackage{amsmath}
\pagestyle{fancy} 

\fancyhead[RO]{\itshape\nouppercase\rightmark}
\fancyhead[RE,LO]{}
\fancyhead[C]{}
\fancyhead[LE]{\itshape\nouppercase\leftmark}
\fancyfoot[C]{\thepage}
%\renewcommand{\headrulewidth}{0pt}
\renewcommand{\footrulewidth}{0pt}

\setlength{\headheight}{15pt}

\makeatletter
\newcommand*{\cleartoleftpage}{%
  \clearpage
    \if@twoside
    \ifodd\c@page
      \hbox{}\newpage
      \if@twocolumn
        \hbox{}\newpage
      \fi
    \fi
  \fi
}
\makeatother

\graphicspath{{./}}
\DeclareGraphicsExtensions{.jpg,.png,.pdf}
\pdfcompresslevel=9
\usepackage[pdftex,bookmarks,bookmarksopen,bookmarksdepth=3]{hyperref}

\hypersetup{
  colorlinks=true, citecolor=black, linkcolor=black, urlcolor=black
}

\newcommand{\titre}{Volcans et calcul d'isogénies}
\newcommand{\titreeng}{Volcanoes and isogeny computation}
\newcommand{\hd}[1]{
\vspace{2eX}\noindent\textbf{#1}
}
% ----SPECIFIC COMMANDS-----------------
\allowdisplaybreaks

\usepackage{boites}

\begin{document}
\frontmatter
\pagestyle{empty}

\newgeometry{top = 1.5 cm , bottom = 1.5 cm , left = 1.5
cm , right = 1.5 cm}
%\begin{figure}[t]
 \includegraphics[width=5.11cm]{image_saclay2.jpg}\hfill\includegraphics[width=5.11cm]{image_uvsq.jpg}
%\end{figure}
\begin{breakbox}
\textbf{NNT : 2016SACLV064}

\vspace{3cm}

\begin{center}
{\Large\textsc{Thèse de Doctorat}

\textsc{de}

\textsc{l’Université Paris-Saclay}

\textsc{préparée à}

\textsc{l’Université de Versailles Saint-Quentin en Yvelines}

}

\vspace{6ex}

{\large\textsc{École doctorale n$^\textrm{o}$580}

Sciences et technologies de l'information et de la communication

\vspace{1ex}

Spécialité de doctorat : Informatique

}

\vspace{6ex}

Par


\vspace{1ex}

{\large\textbf{M. Cyril Hugounenq}

\vspace{1ex}

\titre

}
\end{center}

\vspace{3ex}

\noindent
\textbf{Thèse présentée et soutenue à Versailles, le 18 septembre 2017}

\vspace{2ex}

\noindent
\textbf{Composition du Jury :}

\vspace{2ex}

\begin{tabular}{lr}
\noindent
Luca De Feo, Maître de conférences, Université de Versailles& Co-encadrant\\
Mireille Fouquet, Maître de conférences, Université Paris 7& Examinatrice\\
Louis Goubin, Professeur, Université de Versailles& Directeur\\
David Kohel, Professeur, Université de Aix-Marseille& Rapporteur\\
Pierre Lairez, Chargé de recherches, Inria Saclay& Examinateur\\ 
Arianne Mézard, Professeur, Université Paris 7& Examinatrice\\
Josep M. Miret, Professeur associté, Université de Lleida& Rapporteur\\
François Morain, Professeur, \'Ecole Polytechnique& Directeur\\

\end{tabular}

\vspace{6ex}

\end{breakbox}
\restoregeometry

%\chapter*{Remerciements}
%
%blablabla
%
%\clearpage
%\tableofcontents
%
%\clearpage
%\listoffigures

% \clearpage
% \listoftables


\mainmatter
\pagestyle{fancy}

%\chapter*{Introduction}
%\addcontentsline{toc}{chapter}{Introduction}
%\markboth{Introduction}{Introduction}
%
%blalba


\chapter{1}


\cleartoleftpage
\pagestyle{empty}
\newgeometry{top = 1.5 cm , bottom = 2 cm , left = 2
cm , right = 2 cm}
\phantomsection
\includegraphics[scale = 0.7]{image_stic.jpg}

\vspace{1ex}

\begin{breakbox}

\noindent\textbf{Titre : }\titre

\vspace{2ex}

\noindent\textbf{Mots clés : }cryptographie , courbes elliptiques, isogénies, 
calcul formel

\begin{multicols}{2}
\noindent\textbf{Résumé : }
Le problème du calcul d'isogénies est apparu dans l'algorithme SEA de comptage
de points de courbes elliptiques définies sur des corps finis. L'apparition de 
nouvelles applications du calcul d'isogénies (crypto système à trappe, fonction 
de hachage, accélération de la multiplication scalaire, crypto système post 
quantique) ont motivé par ailleurs la recherche d'algorithmes plus rapides en 
dehors du contexte SEA. L'algorithme de Couveignes (1996), malgré ses 
améliorations par De Feo (2011), présente la meilleure complexité en le 
degré de l'isogénie mais ne peut s'appliquer dans le cas de grande 
caractéristique.


L'objectif de cette thèse est donc de présenter une modification de 
l'algorithme de Couveignes (1996) utilisable en toute caractéristique avec une 
complexité en le degré de l'isogénie similaire à celui de Couveignes (1996).


L'amélioration de l'algorithme de Couveignes (1996) se fait à 
travers deux axes: la construction de tours d'extensions
de degré $\ell$ efficaces pour rendre les opérations plus rapides, à l'image 
des travaux de De Feo (2011), et la détermination d'ensemble de points d'ordre 
$\ell^k$ stables sous l'action d'isogénies.

L'apport majeur de cette thèse est fait sur le second axe pour lequel nous 
étudions les graphes d'isogénies dans lesquels les points représentent les 
courbes elliptiques et les arrêtes représentent les isogénies. Nous utilisons
pour notre travail les résultats précédents de Kohel (1996), Fouquet et Morain
(2001), Miret \emph{et al.} (2005,2006,2008), Ionica et Joux (2001). Nous 
présentons donc dans cette thèse, à l'aide d'une étude de l'action du Frobenius
sur les points d'ordre $\ell^k$, un nouveau moyen de déterminer les directions 
dans le graphe (volcan) d'isogénies.


% après avoir abordé les différents algorithmes dans la lignée
%de Elkies (1998), 
%
%Le problème du calcul d'isogénies est apparu dans l'algorithme SEA de comptage
%de points de courbes elliptiques. Des travaux dans la lignée de ceux de 
%Elkies98 ont eu des résultats satisfaisants dans ce contexte. L'apparition de 
%nouvelles applications du calcul d'isogénie (crypto système à trappe, fonction 
%de hachage, accélération de la multiplication scalaire, crypto système post 
%quantique) ont motivé l’intérêt d'autres approches dont l'algorithme de 
%Couveignes 96.  L'algorithme de Couveignes96, a donc une meilleure complexité 
%que les travaux dans la lignée de Elkies, mais malgré les améliorations de 
%De Feo 11 a une dépendance polynomiale en la caractéristique, à cause de son 
%utilisation des points de $p^k$-torsions. Ainsi, afin de généraliser l'utilisation 
%de Couveignes 96, nous allons travailler à améliorer cet algorithme sur deux 
%axes: la construction de tours d'extensions de degré $\ell$ efficaces pour 
%rendre les opérations plus rapides et la détermination d'ensemble de points 
%d'ordre $\ell^k$ stables sous l'action d'isogénies. Ainsi, dans cette thèse 
%nous présentons une construction de tours d'extensions $\ell$-adique qui 
%généralise les travaux de Doliskani-Schost15 et DeFeo-Doliskani-Schost13. Pour 
%une détermination d'ensemble de points d'ordre $\ell^k$ stables sous l'action 
%d'isogénies, nous étudions notamment les graphes d'isogénies après les travaux 
%de Kohel96 Fouquet-Moran01 Miret et al.05,06,08 et Ionica-Joux10. En 
%particulier (un des apports majeur de la thèse) nous donnons dans cette thèse, 
%à l'aide d'étude de l'action du Frobenius, un nouveau moyen de déterminer des 
%directions dans le graphe d'isogénies ce qui n'a pas été fait auparavant dans 
%la totalité des cas. Nous utilisons ensuite les améliorations de DeFeo11 pour 
%parvenir à donner une nouvelle version de Couveignes96 avec une complexité 
%affranchie de la dépendance polynomiale en la caractéristique. Enfin nous 
%présentons quelques pistes de généralisations de notre algorithme et étudions 
%leur intérêt.
\end{multicols}
\end{breakbox}

%\vspace{1ex}

\begin{breakbox}

\noindent\textbf{Title : }\titreeng

\vspace{2ex}

\noindent\textbf{Keywords : }cryptography , ellitpic curves, isogeny, 
symbolic computation

\begin{multicols}{2}
\noindent\textbf{Abstract : }
Isogeny computation problem appeared in the SEA algorithm to count the number of points on an elliptic curve defined over a finite field. Algorithms using ideas of Elkies (1998) solved this problem with satisfying results in this context. The appearance of new applications of the isogeny computation problem (trapdoor crypto system, hash function, scalar multiplication acceleration, post quantic crypto system) motivated the search for a faster algorithm outside the SEA context. Couveignes's algorithm (1996) offers the best complexity in the degree of the isogeny but, despite improvements by DeFeo (2011), it proves being unpractical with great characteristic.

The aim of this work is to present a modified version of Couveignes's 
algorithm (1996) that maintains the same complexity in the degree of the 
isogeny but is practical with any characteristic.

Two approaches contribute to the improvement of Couveignes's algorithm
(1996) : firstly, the construction of towers of degree $\ell$ extensions which 
are efficient for faster arithmetic operations, as used in the work of De Feo 
(2011), and secondly, the specification of sets of points of order $\ell^k$ 
that are stable under the action of isogenies.

The main contribution of this document is done following the second approach. 
Our work uses the graph of isogeny where the vertices are 
elliptic curves and the edges are isogenies. We based our work on the previous 
results of David Kohel (1996), Fouquet and Morain (2001), Miret \emph{\& al.}
(2005,2006,2008), Ionica and Joux (2001). We therefore present in this 
document, through the study of the action of the Frobenius endomorphism on 
points of order $\ell^k$, a new way to specify directions in the isogeny graph (volcano). 

\end{multicols}
\end{breakbox}

%\vspace{1ex}

{\tiny
\noindent
\textbf{Université Paris-Saclay}

\noindent
Espace Technologique / Immeuble Discovery

\noindent
Route de l’Orme aux Merisiers RD 128 / 91190 Saint-Aubin, France 

}

\end{document}
