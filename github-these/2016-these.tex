\documentclass[10pt,a4paper]{book}
\usepackage[utf8]{inputenc}
\usepackage[T1]{fontenc}
\usepackage{hyperref}
\usepackage[francais]{babel}
\usepackage{hyperref}
\usepackage{amssymb,amsthm,amsmath,amsfonts}
\usepackage{algorithmic}
\usepackage{algorithm}
\usepackage{units}
\usepackage{lmodern}
\usepackage{fp}
\usepackage{graphicx}
\bibliographystyle{plain}
\usepackage{epsfig}
\usepackage{epstopdf}
\usepackage{makeidx}
\usepackage{gloss}
\usepackage{xcolor}
\def\glossname{Glossaire}
\usepackage{ragged2e}
\usepackage{changepage}
\usepackage{tikz}
\usetikzlibrary{calc}
\usetikzlibrary{fixedpointarithmetic}

\usepackage{fancyhdr}
\pagestyle{fancy}




\theoremstyle{plain}
\newtheorem{thm}{Théorème}
\theoremstyle{definition} 
\newtheorem{lem}[thm]{Lemme}
\theoremstyle{definition} 
\newtheorem{cor}[thm]{Corollaire}
\theoremstyle{definition} 
\newtheorem{prop}[thm]{Proposition}
\theoremstyle{definition} 
\newtheorem{defi}[thm]{Définition}
\theoremstyle{remark} 
\newtheorem{rem}[thm]{Remarque}
\theoremstyle{remark} 
\newtheorem{exe}[thm]{Exemple}

\def\algorithmicrequire{\textbf{Entrée:}}
\def\algorithmicensure{\textbf{Sortie:}}

\newcommand{\defeq}{\mathrel{\mathop:}=}

\author{Hugounenq Cyril}
\title{Arithmétique Rapide appliqué à la Géométrie et à la Cryptologie}
\makeindex
\makegloss


\makeatletter
\def\footrule{{
\vskip-\footruleskip\vskip-\footrulewidth
\color{\footrulecolor}
\hrule\@width\headwidth\@height
\footrulewidth\vskip\footruleskip
}}
\makeatother

%%% BEGIN DOCUMENT
\begin{document}
\fancyhf{}%␣efface␣tout␣ce␣qu'il␣y␣avait␣avant
\fancyhead[L]{\nouppercase\leftmark}%␣LO␣=␣gauche/impair␣;␣RE␣=␣droite/pair
\fancyhead[R]{}
\fancyfoot[C]{\thepage}%␣C␣=␣centré
\fancyfoot[L]{}




\renewcommand{\headrulewidth}{1pt}% 1pt header rule
\renewcommand{\headrule}{\hbox to\headwidth{%
  \color{orange}\leaders\hrule height \headrulewidth\hfill}}
  
\renewcommand{\footrulewidth}{1pt}% 1pt header rule
\newcommand{\footrulecolor}{blue}


%\small
\setlength{\parskip}{-1pt plus 1pt}

\newcommand{\abstracttextfont}{\normalfont}
%\abstractintoc
%\begin{abstract} 
%Text 
%\end{abstract}
%\abstractintoc
%\renewcommand\abstractname{R\'esum\'e}
%\begin{abstract} \selectlanguage{french}
%Texte 
%\end{abstract}
\selectlanguage{french}% french si rapport en français



%\cleardoublepage

%\tableofcontents* % the asterisk means that the table of contents itself isn't put into the ToC
\normalsize

\mainmatter
% \SingleSpace

\chapter{Rappels théoriques}



\section{Théorie des ordres dans un corps quadratique imaginaire}

\section{Courbes Elliptiques}

Dans ce chaptire nous allons faire quelques rappels sur les courbes elliptiques. Pour approfondir le sujet le lecteur pourra consulter \cite{Silv1} ou encore \cite{Washington2008}

\subsection{\'Equations de Weierstrass}

Soit $\mathbb{K}$ un corps, on dénote par $\overline{\mathbb{K}}$ sa cloture algébrique. On rappelle tout d'abord  la définition d'un espace projectif sur $\mathbb{K}$:

\begin{defi}
On appelle espace projectif de dimension $2$ sur un corps $\mathbb{K}$ et on note $\mathbb{P}^2(\mathbb{K})$, l'ensemble des classes $(X:Y:Z)$ de la relation d'équivalence définie comme suit:
\begin{equation*}
\forall (X,Y,Z) \in \mathbb{K}^3 \setminus (0,0,0), (X,Y,Z) \equiv (X',Y',Z') \Leftrightarrow \exists c \in \mathbb{K}^*, X'=cX, Y'=cY, Z'=cZ.
\end{equation*}
\end{defi}



On définit sur $\mathbb{K}$ une équation de Weierstrass comme une équation de la forme:

\begin{defi}
Une courbe elliptique $E$ peut être définie comme la variété projective associée à l'équation :
\begin{equation}
\label{eq:weierstrass-proj}
E:Y^2Z+a_1XYZ+a_3YZ^2=X^3+a_2X^2Z+a_4XZ^2+a_6Z^3
\end{equation}
avec $a_1,a_2,a_3,a_4,a_6$ des éléments de $\overline{\mathbb{K}}$. Le point $[0:1:0]$ est appelé le \emph{point à l'infini} et sera noté $0_E$. Dès lors que l'on a $a_1,a_2,a_3,a_4,a_6$ des éléments de $\mathbb{K}$ alors on dit que la courbe est définie sur $\mathbb{K}$.
\end{defi}

L'équation \ref{eq:weierstrass-proj} est dite forme homogène de l'équation de Weierstrass. En appliquant le changement de variables suivant: $x=\frac{X}{Z}$, $y=\frac{Y}{Z}$, on passe alors en coordonnées affines, on obtient l'équation de Weierstrass suivante:

\begin{equation}
\label{eq:weierstrass-red}
E:y^2+a_1xy+a_3y=x^3+a_2x^2+a_4x+a_6
\end{equation}
appelée équation affine de Weierstrass. La courbe $E$ est alors définie comme étant l'ensemble des points solutions de \ref{eq:weierstrass-red} avec le point à l'infini $O_E$.

 On définit aussi:
\begin{equation*}
b_2=a_1^2+4a_2, b_4=a_1a_3+2a_4, b_6=a_3^2+4a_6, b_8=a_1^2a_6-a_1a_3a_4+4a_2a_6+a_2a_3^2-a_2^4
\end{equation*}
On définit alors le discriminant $\Delta$ de l'équation de Werierstrass \ref{eq:weierstrass-red}:
\begin{equation*}
\Delta = -b_2^2b_8-8b_4-27b_6^2+9b_2b_4b_6
\end{equation*}

La courbe $E$ admet un unique point singulier si et seulement si $\Delta$ est nul, la courbe est non singulière si et seulement si $\Delta$ est non nul (\cite[Prop. III.1.4]{Silv1}).

%L'équation de Weierstrass \ref{eq:weierstrass} est dite non-singulière si et seulement si $\Delta$ est non nul.
On peut alors énoncer une autre définition de courbe elliptique.

\begin{defi}
Une courbe elliptique $E$ sur $\mathbb{K}$ est une courbe définie par une équation de Weierstrass \ref{eq:weierstrass-red} non singulière.
\end{defi}

On définit: 
\begin{equation*}
j=\frac{(b_2^2-24b_4)^3}{\Delta}
\end{equation*}
La constante $j$ est appelée \emph{$j$-invariant} de la courbe.

Une question naturelle est de se demander à quel point l'équation de Weierstrass définit de façon unique une courbe elliptique. En supposant que l'on considère les changements de variable qui préservent le point à l'infini $O_E$, il est montré \cite[III.3.1b]{Silv1} que le seul changement de variable préservant la forme de l'équation de Weierstrass et $O_E$ est:

\begin{equation*}
x=u^2x'+r    \quad  y=u^3y'+u^2sx'+t
\end{equation*}
avec $u,r,s,t$ des éléments de $\overline{\mathbb{K}}$ et $u$ différent de $0$. En actualisant alors le calcul des $a'_i$ et des calculs subséquents on justifie alors la dénomination de \emph{$j$-invariant}.
\newline


On définit alors l'ensemble des points d'une courbe elliptique $E$ définie sur $\mathbb{K}$  noté $E(\mathbb{K})$.

\begin{defi}
Soit une courbe $E$ définie sur $\mathbb{K}$ on définit $E(\mathbb{K})$ comme étant l'ensemble:
\begin{equation*}
E(\mathbb{K})=\{(X:Y:Z)\in \mathbb{P}^3(\mathbb{K}) \setminus (0,0,0) : Y^2Z+a_1XYZ+a_3YZ^2=X^3+a_2X^2Z+a_4XZ^2+a_6Z^3 \}
\end{equation*}
Le seul point correspondant à $Z=0$ est le point à l'infini $O_E=(0:1:0)$, chaque classe différente de $O_E$ a un unique représentant $(X:Y:1)$ ce qui nous donne la notation équivalente suivante:
\begin{equation*}
E(\mathbb{K})=\{(x,y)\in \mathbb{K}^2  : y^2+a_1xy+a_3y=x^3+a_2x^2+a_4x+a_6 \} \cup {O_E}
\end{equation*}
\end{defi}

On rappelle que si l'on note $f(x,y)=0$ l'équation affine de Weierstrass d'une courbe elliptique $E$ définie sur $\mathbb{K}$, alors son corps de fonction $\mathbb{K}(E)$ est le corps des fractions suivant:
\begin{equation*}
\mathbb{K}(E) \defeq \mathbb{K}[x,y]/(f(x,y)).
\end{equation*}

\subsection{Loi de groupe}
%\begin{figure}
%\begin{tikzpicture}[domain=-10/3:4.5,samples=100,yscale=1/2]
%\begin{scope}[fixed point arithmetic]
%\coordinate (Q) at (2, -4/3) {};
%\coordinate (-Q) at (2, 4/3) {};
%\coordinate (-2Q) at (-1079/576, 59653/13824);
%\coordinate (3Q) at (2319138/4977361, 27920423524/11104492391);
%\draw plot (\x,{sqrt(\x*\x*\x+(3-100/9)*\x+10)});
%\draw plot (\x,{-sqrt(\x*\x*\x+(3-100/9)*\x+10)});
%\draw (-2Q) node[/dot,label=$P$] {}
%              (3Q) node[/dot,label=$Q$] (G) {}
%              (-Q) node[/dot,label=$R$] {};
%\draw[label position=right,label distance=5pt] (Q) node[/dot,label=$P+Q$] {};
%\end{scope}
%\end{tikzpicture}
%\end{figure}
%
\begin{figure}

\begin{tikzpicture}[scale=0.60,domain=-10/3:4.5,samples=100,yscale=1/2]
          \begin{scope}[fixed point arithmetic]
            \begin{scope}[fixed point arithmetic]
              \coordinate (Q) at (2, -4/3) {};
              \coordinate (-Q) at (2, 4/3) {};
              \coordinate (-2Q) at (-1079/576, 59653/13824);
              \coordinate (3Q) at (2319138/4977361, 27920423524/11104492391);
            \end{scope}

            \draw plot (\x,{sqrt(\x*\x*\x+(3-100/9)*\x+10)});
            \draw plot (\x,{-sqrt(\x*\x*\x+(3-100/9)*\x+10)});

            \draw[shorten >=-2cm, shorten <=-2cm] (-2Q) -- (-Q);
            \draw[dashed] (-Q) -- (Q);

            \begin{scope}[/dot/.style={draw,circle,inner sep=1pt,fill},
              every label/.style={inner sep=5pt,font=\scriptsize}]
              \draw (-2Q) node[/dot,label=$P$] {}
               (3Q) node[/dot,label=$Q$] (G) {}
              (-Q) node[/dot,label=$R$] {};
              \draw[label position=right,label distance=5pt] (Q) node[/dot,label=$P+Q$] {};
            \end{scope}
          \end{scope}

          \begin{scope}[xshift=10cm]
            \draw plot (\x,{sqrt(\x*\x*\x+(3-100/9)*\x+10)});
            \draw plot (\x,{-sqrt(\x*\x*\x+(3-100/9)*\x+10)});

            \begin{scope}[fixed point arithmetic]
              \coordinate (-Q) at (2, 4/3) {};
              \coordinate (-2Q) at (-1079/576, 59653/13824);
              \path let \p1 = (-2Q) in coordinate (2Q) at (\x1,-\y1);
            \end{scope}

            \draw[shorten >=-2cm, shorten <=-2cm] (-Q) -- (2Q);
            \draw[dashed] (-2Q) -- (2Q);

            \begin{scope}[/dot/.style={draw,circle,inner sep=1pt,fill},
              every label/.style={inner sep=5pt,font=\scriptsize}]
              \draw (-Q) node[/dot,label=$P$] {}
              (-2Q) node[/dot,label=$2P$] (G) {};
              \draw[label position=below] (2Q) node[/dot,label=$R$] {};
            \end{scope}
          \end{scope}
        \end{tikzpicture}

\end{figure}

On peut définir sur $E(\mathbb{K})$ une loi de groupe abélienne à l'aide de la méthode dite de la \emph{corde et de la tangente}

\begin{defi}[Loi d'addition]
Soient $P,Q \in E(\mathbb{K})$, $L$ la droite reliant $P$ à $Q$ (ou tangente si $P=Q$) et $R$ le troisième point d'intersection de $L$ avec $E$. Soit $L'$ la droite reliant $R$ à $O_E$ cette droite intersecte alors la courbe en un troisième point $P+Q$.
\end{defi}

Une preuve que cette loi définit bien une loi de groupe abélienne est montrée dans \cite[III.2.2]{Silv1}.

En utilisant la notation $(x_P,y_P)$ pour les coordonnées affines d'un point $P$ de la courbe, on obtient alors pour les points distincts $P,Q$ de $E(\mathbb{K})$ les formules suivantes:
\begin{itemize}
\item $P+O_E=O_E+P=(x_P,y_P)$
\item $-P=(x_P,-y_P-a_1x_P-a_3)$
\item En notant $\lambda= \begin{cases}
\frac{y_{Q}-y_{P}}{x_{Q}-x_{P}} & \textit{si } P\neq Q\\
\frac{3x_{P}^{2}+2a_{2}x_{P}+a_{4}-a_{1}y_{P}}{2y_{P}+a_{1}x_{P}+a_{3}} & \textit{si } P=Q
\end{cases}\ $,

on peut alors exprimer les coordonnées de $P+Q$:
\begin{equation*}
\begin{alignedat}{1}
& x_{P+Q}=\lambda^2 +a_1\lambda-a_2-x_P-x_Q,\\
& y_{P+Q}=-(\lambda +a_1)x_{P+Q}-y_{P}-\lambda x_P-a_3
\end{alignedat}
\end{equation*}
\end{itemize}




\paragraph*{Multiplication par $m$}\   {
\newline


\'A l'aide de la loi additive on peut définir la multiplication scalaire par l'entier naturel $m$ comme suit:

\begin{equation}
\begin{alignedat}{1}
[m]: E(\mathbb{K}) & \rightarrow E(\mathbb{K}) \\
P & \mapsto \underbrace{P+ \cdots +P}_{m fois} 
\end{alignedat}
\end{equation}

\begin{defi}
On appelle points de \emph{$m$-torsion} l'ensemble des points de $E$ qui sont envoyés sur l'élément neutre $0_E$ par $[m]$. Cet ensemble est noté $E[m]$.
\end{defi}
%On a ici un exemple d'endomorphismes de la courbe $E$.
}

\subsection{Isogénies}

\begin{defi}
Soient $E_1$ et $E_2$ deux courbes elliptiques, une isogénie de $E_1$ dans $E_2$ est un morphisme de groupes:
\begin{equation*}
\varphi:E_1 \rightarrow E_2
\end{equation*}
tel que $\varphi(O_{E_1})=O_{E_2}$.
\end{defi}
On dira que $E_1$ et $E_2$ sont isogénes si il existe une isogénie $\varphi$ non nulle qui va de $E_1$ dans $E_2$. Une question naturelle est de se demander comment ces isogénies agissent sur les groupes de points d'une courbe, le théorème suivant répond à cette question.

\begin{thm}
Soient $\varphi:E_1 \rightarrow E_2$ une isogénie, $P,Q$ deux points de $E_1$. Alors:
\begin{equation*}
\varphi(P+Q)=\varphi(P)+\varphi(Q)
\end{equation*}
\end{thm}

\begin{proof}
Voir \cite[III.4.8]{Silv1}
\end{proof}

\begin{exe}
La multiplication par $m$ est par exemple un homomorphisme:
\begin{equation*}
[m](P+Q)=[m]P+[m]Q
\end{equation*}
\end{exe}

\begin{thm}
Soit $\varphi$ une isogénie non nulle définie sur un corps $\mathbb{K}$ qui va de $E_1$ dans $E_2$, alors:
\begin{itemize}
\item $\ker(\varphi)$ est un groupe fini,
\item $\varphi$ est surjective si $\mathbb{K}$ est un corps algébriquement clos.
\end{itemize}
\end{thm}

\begin{proof}
Voir \cite[II.2]{Silv1}
\end{proof}

Lorsque l'on a $\varphi$ surjective alors on peut définir sur les corps de fonctions associés à $E_1$ et $E_2$ l'injection suivante:
\begin{equation*}
\begin{alignedat}{1}
\varphi^*: \mathbb{K}(E_2) & \rightarrow \mathbb{K}(E_1) \\
f & \mapsto f  \circ \varphi
\end{alignedat}
\end{equation*}

\begin{defi}
Soit $\mathbb{K}$ un corps algébriquement clos, $\varphi$ une isogénie définie sur $\mathbb{K}$ qui va de $E_1$ dans $E_2$, alors $\varphi$ est dite \emph{séparable} ou \emph{inséparable} selon les propriétés de $[K(E_1):\varphi^*(\mathbb{K}(E_2))]$. On note alors \emph{$\deg_s{\varphi}$}, \emph{$\deg_i{\varphi}$} les degrés correspondants et l'on a $\deg{\varphi}=\deg_s{\varphi}\deg_i{\varphi}$
\end{defi}

Lorsque $\mathbb{K}$ n'est pas algébriquement clos on définit le degré de l'isogénie comme étant le degré de l'isogénie définie sur la cloture algébrique de $\mathbb{K}$.


On désignera alors pour un entier $\ell$, une isogénie de degré $\ell$ par: $\ell$-isogénie.

\begin{thm}
Soit $\varphi$ une isogénie non constante de $E_1$ dans $E_2$, alors:
\begin{itemize}
\item pour tout point $Q$ de $E_2$ on a $|\varphi^{-1}(Q)|=\deg(\varphi)$,
\item si $\varphi$ est séparable alors $\varphi$ est non ramifiée et on a $\deg_s(\varphi)=\ker(\varphi)$.
\end{itemize}
\end{thm}

\begin{proof}
Voir \cite[III.4.10]{Silv1}
\end{proof}

Une conséquence directe de ce théorème est le résultat suivant:

\begin{cor}
Soit $E_1$ une courbe elliptique, alors il y a une correspondance bijective entre les isogénies séparables qui ont pour domaine de définition $E_1$ et les sous-groupes de $E_1$:
\begin{equation*}
\begin{alignedat}{1}
(\varphi :E_1\rightarrow \varphi(E_1)) &\mapsto  \ker(\varphi)  \\
 (E_1\rightarrow\nicefrac{E_1}{C})  &\  \reflectbox{$\mapsto$} \ C 
\end{alignedat}
\end{equation*}
\end{cor}

L'ensemble des isogénies de $E_1$ vers $E_2$ est appelé homomorphismes et est noté \emph{$\mathrm{Hom}(E_1,E_2)$}. L'ensemble des isogénies de $E_1$ vers $E_1$ est appelé endomorphismes et est noté \emph{$\mathrm{End}(E_1)=\mathrm{Hom}(E_1,E_1)$}, les éléments de $\mathrm{End}(E_1)$ inversibles sont appelés automorphismes et sont notés \emph{$\mathrm{Aut}(E_1)$}. Une isogénie définie sur $\mathbb{K}$ sera dite \emph{$\mathbb{K}$-rationnelle} ou simplement \emph{rationnelle} quand il n'y a pas d'ambiguïtés.

\begin{defi}[Isogénie duale]
Soit $\varphi$ une isogénie définie sur $\mathbb{K}$ d'une courbe elliptique $E_1$ vers une courbe elliptique $E_2$, alors l'unique isogénie $\widehat{\varphi}$ qui va de $E_2$ ver $E_1$ telle que $\widehat{\varphi}\circ\varphi=[\deg(\varphi)]$ est appelée \emph{isogénie duale}.
\end{defi}

\begin{proof}
Voir \cite[III.6.1]{Silv1}
\end{proof}

\begin{thm}
Soit $\varphi:E_1 \rightarrow E_2$ une isogénie.
\begin{enumerate}
\item Soit $m=\deg(\varphi)$, alors:
\begin{equation*}
\widehat{\varphi}\circ \varphi=[m]_{E_1} \textit{ et }  \varphi \circ \widehat{\varphi}=[m]_{E_2}.
\end{equation*}
\item Soit $\lambda:E_2 \rightarrow E_3$ une autre isogénie, alors
\begin{equation*}
\widehat{\lambda \circ \varphi}=\widehat{\lambda} \circ \widehat{\varphi}.
\end{equation*} 
\item Soit $\phi:E_1 \rightarrow E_2$ une autre isogénie, alors
\begin{equation*}
\widehat{\varphi+\phi}=\widehat{\varphi}+\widehat{\phi}.
\end{equation*}
\item Pour tout $m$ entier relatif,
\begin{equation*}
[\widehat{m}]=[m] \textit{ et } \deg[m]=m^2.
\end{equation*}
\item $\deg( \varphi)=\deg(\widehat{\varphi})$,
\item $\widehat{\widehat{\varphi}}=\varphi$.
\end{enumerate}
 
\end{thm}

\begin{proof}
Voir \cite[Theorem III.6.2]{Silv1}.
\end{proof}

\begin{cor}
Soit $E$ une courbe elliptique définie sur $\mathbb{K}$ et $m \in \mathbb{Z}$ avec $m \neq 0$,
\begin{enumerate}
\item Si $m \neq 0$ dans  $\mathbb{K}$ alors
\begin{equation*}
E[m]=\nicefrac{\mathbb{Z}}{m\mathbb{Z}} \times \nicefrac{\mathbb{Z}}{m\mathbb{Z}}. 
\end{equation*}
\item Si $\mathrm{car}(\mathbb{K})=p>0$ alors:
\begin{equation*}
\begin{alignedat}{1}
&\textit{Soit } E[p^e]=\{O_E\} \textit{ pour tout } e \geqslant 1 \\
&\textit{Soit } E[p^e]=\nicefrac{\mathbb{Z}}{p^e\mathbb{Z}} \textit{ pour tout } e \geqslant 1
\end{alignedat}
\end{equation*}
\end{enumerate}
\end{cor}

\begin{proof}
Voir \cite[Corrolary III.6.4]{Silv1}.
\end{proof}

\begin{defi}
Une courbe elliptique $E$ définie sur un corps $\mathbb{K}$ de caractéristique $p$ différente de $0$ est dite \emph{supersingulière} si $E[p^e]=\{O_E\}$ pour tout $e$ entier naturel non nul, elle est dite \emph{ordinaire} sinon. 
\end{defi}

\begin{defi}[Module de Tate]
Soit $E$ une courbe elliptique et $\ell$ un nombre premier. Le \emph{$\ell$-adique module de Tate} de $E$ est le groupe:
\begin{equation*}
T_{\ell}(E)=\varprojlim_nE[\ell^n],
\end{equation*}
la limite projective étant définie par rapport à l'application naturelle
\begin{equation*}
E[\ell^{n+1}] \overset{[\ell]}{\rightarrow} E[\ell^n].
\end{equation*}
\end{defi}
Comme chacun des $E[\ell^n]$ est un $\nicefrac{\mathbb{Z}}{\ell^n\mathbb{Z}}$-module, on observe alors que le module de Tate a une structure naturelle de $\mathbb{Z}_{\ell}$-module.
\begin{prop}
En tant que $\mathbb{Z}_{\ell}$ module, le module de Tate a la structure suivante:
\begin{equation*}
T_{\ell} \cong 
\begin{cases} 
\mathbb{Z}_{\ell} \times \mathbb{Z}_{\ell} &\textit{ si } \ell \neq \mathrm{car}(\mathbb{K}). \\
\mathbb{Z}_{\ell} &\textit{ si } \ell=\mathrm{car}(\mathbb{K})  \textit{et E est ordinaire}. \\
\{O_E\} &\textit{ si } \ell=\mathrm{car}(\mathbb{K})  \textit{et E est supersingulière}.
\end{cases}
\end{equation*}
\end{prop}

%\begin{alignedat}{2}
%& T_{\ell} \cong \mathbb{Z}_{\ell} \times \mathbb{Z}_{\ell} &\textit{ si } \ell \neq \mathrm{car}(\mathbb{K}).& \\
%& T_{p} \cong \{O_E\} \textit{ or } \mathbb{Z}_{p} \times \mathbb{Z}_{\ell} &\textit{ si }  p =\mathrm{car}(\mathbb{K})&>0.
%\end{alignedat}


\subsection{Endomorphismes}

Soit $E$ une courbe elliptique, alors si l'on adjoint à $\mathrm{End}(E)=\mathcal{O}$, la composition et l'addition d'endomorphismes cela forme un anneau. De plus la multiplication scalaire fournit un homomorphisme de $\mathbb{Z}$ dans $\mathrm{End}(E)$. Cependant $\mathrm{End}(E)$ n'est pas nécessairement réduit à $\mathbb{Z}$.

\begin{defi}
Une courbe elliptique $E$ est dite à multiplication complexe si $\mathrm{End}(E)$ n'est pas réduit à $\mathbb{Z}$.
\end{defi}
On a le résultat suivant sur la structure de $\mathrm{End}(E)$.

\begin{thm}
$\mathrm{End}(E)$ est soit isomorphe à $\mathbb{Z}$, soit isomorphe à un ordre dans un corps quadratique imaginaire, soit isomorphe à un ordre dans une algèbre de quaternions 
\end{thm}

\begin{proof}
Voir \cite[Corollary III.9.4]{Silv1}
\end{proof}

\section{Courbes elliptiques définies sur un corps fini}

Soit $p$ un nombre premier, on notera $\mathbb{F}_q=\mathbb{F}_{p^e}$ le corps fini à $q$ éléments défini à isomorphisme près. Travailler sur un corps fini entraîne le fait que le nombre de points d'une courbe elliptiqe est fini, ainsi nous travaillons avec un groupe abélien fini. Afin d'étudier ce groupe et en particulier d'établir sa cardinalité nous allons introduire l'endomorphisme de Frobenius. Dans la suite de ce document nous considérerons que les courbes elliptiques mentionnées sont définies sur un corps fini. 

\subsection{L'endomorphisme de Frobenius}
\begin{defi}
Soit $E$ une courbe définie sur $\mathbb{F}_q$, l'application: 
\begin{equation*}
\begin{alignedat}{1}
\pi :E &\mapsto  E  \\
 (x,y)  &\mapsto (x^q,y^q)  
\end{alignedat}
\end{equation*}
est un homomorphisme de la courbe et est appelée \emph{endomorphisme de Frobenius}. 
\end{defi}

\begin{prop}
Soit $E$ une courbe elliptique définie sur $\mathbb{F}_q$, soit $\pi$ l'endomorphisme de Frobenius relatif à $\mathbb{F}_q$. Les conditions suivantes sont équivantes:
\begin{enumerate}
\item $E$ est supersingulière,
\item $E[p^e]=\{O_E\}$ pour tout $e > 0$,
\item la duale $\widehat{\pi}$ de l'endomorphisme de Frobenius est purement inséparable,
\item la trace de l'endomorphisme de Frobenius est divisible par $p$,
\item l'anneau d'endomorphisme $\mathrm{End}(E)$ est isomorphe à un ordre dans une algèbre de quaternions.
\end{enumerate}
si ces conditions ne sont pas vérifiées alors on a les conditions équivalentes suivantes:
\begin{enumerate}
\item $E$ est ordinaire,
\item $E[p^e]=\mathbb{Z}/p^e\mathbb{Z}$ pour tout $e > 0$,
\item la duale $\widehat{\pi}$ de l'endomorphisme de Frobenius est séparable,
\item la trace de l'endomorphisme de Frobenius est première avec $p$,
\item l'anneau d'endomorphisme $\mathrm{End}(E)$ est isomorphe à un ordre dans un corps de nombre quadratique imaginaire.
\end{enumerate}
\end{prop}

\begin{proof}
Voir \cite[Theorem V.3.1]{Silv1}
\end{proof}

L'endomorphisme de Frobenius relatif à $\mathbb{F}_q$ admet pour polynôme caractéristique:
\begin{equation}
\pi^2-t\pi+q=0
\end{equation}
avec $t$ la trace de l'endomorphisme de Frobenius. Cette équation a un rôle important dans le calcul de cardinalité de courbes elliptiques, en particulier dans l'algorithme de Schoof.

\begin{prop}
Soit $E$ une courbe elliptique définie sur $\mathbb{F}_q$, soit $t$ la trace de l'endomorphisme de Frobenius relatif à $ \mathbb{F}_q$ alors on a:
\begin{equation*}
|E(\mathbb{F}_q)|=q+1-t.
\end{equation*}
\end{prop}

\begin{proof}
Voir \cite[Theorem V.1.1]{Silv1}
\end{proof}

\begin{thm}[Hasse]
Soit $E$ une courbe définie sur $\mathbb{F}_q$, alors 
\begin{equation}
|t| \leqslant 2 \sqrt{q}
\end{equation}
\end{thm}

\begin{proof}
Voir \cite[Theorem V.1.1]{Silv1}
\end{proof}

On a alors le résultat suivant sur la structure des points d'une courbe elliptique:

\begin{prop}
Le groupe abélien $E(\mathbb{F}_q$) est isomorphe à $\nicefrac{\mathbb{Z}}{n_1 \mathbb{Z}} \times \nicefrac{\mathbb{Z}}{n_2 \mathbb{Z}}$ avec $n_2 \mid n_1$ et $n_2 \mid q-1$
\end{prop}

\begin{proof}
Voir \cite[Theorem 4.1]{Washington2008}
\end{proof}



\subsection{Courbes définies sur $\mathbb{C}$}
\subsection{Multiplication complexe}
\subsection{Polynôme modulaire}
\subsection{Bases (loi additive, isogénie, frobenius, module de Tate}
\subsection{S.E.A}
Ici vont être présentés les bases de S.E.A., aucun apport personnel n'a été fait dans ce cadre là, mais il est intéressant pour le lecteur de voir les concepts et  le contexte de celui-ci qui seront évoqués plus tard.

L'algorithme de Schoof \cite{Schoof85} consistait à calculer le cardinal d'une courbe elliptique $E$ définie sur $\mathbb{F}_q$ à l'aide du polynôme charactéristique du Frobenius $\pi$:
\begin{equation*}
X^2-t_{\pi}X+q=0
\end{equation*}
en déterminant la valeur de la trace du Frobenius $t_{\pi}$, on a alors: $|E(\mathbb{F}_q)|=q+1-t_{\pi}$. 
Cette équation est résolue modulo la $\ell_i$-torsion, avec $\ell_i$ un nombre premier. Cette résolution est répétée pour un ensemble de nombres premiers $L$ tel que: $\Pi_{\ell_i \in L}\ell_i>4 \sqrt{q}$, le théorème des restes chinois est alors utilisé pour déterminer $t_{\pi}$ modulo $4\sqrt{q}$. Ainsi avec le théorème de Hasse on peut conclure sur la cardinalité de la courbe sur $\mathbb{F}_q$.

L'un des principals problèmes avec cette approche étant que l'on travaille avec $\ell$ torsion qui lorsque l'on a $\ell \neq p$ est de la forme $E[\ell] \simeq \mathbb{Z}/\ell \mathbb{Z} \times \mathbb{Z}/\ell \mathbb{Z}$ et l'on doit alors travailler modulo $f_{\ell}$ le polynôme de $\ell$ division (qui s'annule sur les abscisses des points de $\ell$ torsion) qui est de degré $\ell^2-1/2$. L'idée est alors de considérer des ensembles plus restrictifs de la $\ell$-torsion afin de diminuer le coût de l'algorithme.

Des améliorations ont donc été apporté à l'algorithme de Schoof par Atkin et Elkies en analysant plus précisément l'action du Frobenius sur la $\ell$-torsion. En se rappelant que l'on a  $E[\ell] \simeq \mathbb{Z}/\ell \mathbb{Z} \times \mathbb{Z}/\ell \mathbb{Z}$ alors on peut représenter l'action de $\pi$ sur la $\ell$-torsion sous forme matricielle. Ainsi en analysant les solutions du polynôme charactéristique de $\pi$:
\begin{equation*}
X^2-t_{\pi}X+q = 0 \bmod \ell
\end{equation*} 
On a plusieurs cas possibles selon que le discriminant $d_{\pi}=t_{\pi}^2-4q$ soit un résidu quadratique ou non quadratique dans $\mathbb{Z}/\ell\mathbb{Z}$. Si $d_{\pi}$ est un résidu quadratique dans $\mathbb{Z}/\ell\mathbb{Z}$ alors $\ell$ est dit nombre de Elkies, sinon il est dit nombre de Atkin. Comme on ne peut connaître à priori ce résultat, car il suppose la connaissance de $t_{\pi}$ et donc la cardinalité de la courbe. On doit alors travailler avec un autre objet le $\ell$-ème polynôme modulaire $\Phi_{\ell}(X,Y) \in \mathbb{Z}[X,Y] $ de degré $\ell+1$ et symétrique. Ce polynôme a la propriété suivante soit $j_0 \in \mathbb{F}_q$ le $j$-invariant d'une courbe elliptique $E_0$ définie sur $\mathbb{F}_q$ alors les $\ell+1$ racines du polynôme $\Phi_{\ell}(X,j_0)$ sont les $j$-invariants de courbes $\ell$-isogénes à $E_0$. Ces $j$ invariants racines du polynôme modulaire évalué en $j_0$ correspondent exactement aux $j$-invariants des  courbes générées par les $\ell$ isogénies $E \to E/C$ avec $C$(noyau de l'isogénie) un des $\ell+1$ sous groupe cyclique de la $\ell$-torsion.  Dés lors on voit bien que avec cette remarque on voudrait travailler sur ces groupes cycliques de la $\ell$ torsion au lieu de la $\ell$ torsion toute entière, cependant il faudrait que celle-ci soit définie sur le corps sur lequel on travaille. On a alors le résultat suivant qui nous permet d'énoncer dans quel corps $C$ est défini en fonction des cas.

%On a un premier résultat qui nous permet de spécifier certains sous ensembles de la $\ell$-torsion.

\begin{prop}
Soit $E$ une courbe ordinaire définie sur $\mathbb{F}_q$ de $j$-invariant $j(E) \neq 0, 1728$. Alors:
\begin{enumerate}
\item le polynôme $\Phi(X,\ell)$ a une racine $j_0 \in \mathbb{F}_{q}^r$ si et seulement si le noyau $C$ de l'isogénie $E \to E/C$ avec $j(E/C)=j_0$ est un espace propre de dimension $1$ pour $\pi^r$ appliqué à $E[\ell]$
\end{enumerate}
\end{prop}

On peut alors énoncer le théorème suivant de Atkin qui permet de déterminer si le nombre est de Elkies ou de Atkin à l'aide du $\ell$-ième polynôme modulaire:

\begin{thm}[Atkin]
Soit $E$ une courbe ordinaire définie sur $\mathbb{F}_q$ de $j$-invariant $j(E) \neq 0, 1728$. Soit $\Phi_{\ell}(X,j(E))=f_1f_2\cdots f_s$ la factorisation de $\Phi_{\ell}(X,j(E))$ dans $\mathbb{F}_q[X]$ en éléments irréductibles, alors les degrés possibles de $f_1,f_2, \cdots , f_s$ sont:
\begin{enumerate}
\item $(1,\ell)$ ou $(1,1, \cdots, 1)$. Dans les deux cas on a $t_{\pi}^2-4q=0 \bmod \mathbb{Z}/\ell \mathbb{Z}$, dans le premier cas on pose $r=\ell$ et dans le second $r=1$;
\item $(1,1,r,r, \cdots,r)$. Dans ce cas on a $t_{\pi}^2-4q=0$ qui est un résidu quadratique modulo $\ell$. On a alors $r$ qui divise $\ell-1$, et l'on a alors $\pi$ qui agit comme une matrice scalaire sur $E[\ell]$;
\item $(r,r,\cdots,r)$ avec $r>1$. Dans ce cas $t_{\pi}^2-4q$ n'est pas un résidu quadratique, $r$ divise $\ell+1$ et la restriction de $\pi$ à $E[\ell]$ admet un polynôme charactéristique irréductible sur $\mathbb{Z}/\ell \mathbb{Z}$. 
\end{enumerate}
\end{thm}
Dans tous les cas on a $r$ qui correspond à l'ordre de la matrice $\pi$ dans $PGL_2(\mathbb{Z}/\ell \mathbb{Z})$

\subsection{Lien entre isogénie et son action sur le module de Tate}

\chapter{Construction efficaces de tours $\ell$-adiques}
La construction d'extension de corps finis est un problème qui devient naturel dès lors que l'on travaille sur des corps finis avec des algorithmes pour lesquels la précision nécessaire au cours du calcul n'est pas fixée mais est un paramètre de l'entrée. Dés lors on aura besoin de  construire des extensions de corps de taille arbitraire, pour lesquelles on voudra une construction dite compatible de telle sorte que l'on puisse injecter efficacement, lorsque cela est possible, les résultats dans les corps les plus petits au sens de l'inclusion.

\begin{defi}[Tour \textit{$\ell$}-adique]
\label{def:tour-ell}
On appelle tour d'extension $\ell$-adique du corps fini $\mathbb{F}_q$ la suite des extensions: $\mathbb{F}_q, \mathbb{F}_{q^{\ell}}, \mathbb{F}_{q^{\ell^2}}, ...$ pour lesquels chaque élément de la suite est une extension de degré $\ell$ du précédent.
\end{defi}


Ainsi les travaux de \cite{DeFeo-Shost'12} sur les tours $p$-adiques ont ouvert la voie à d'autres travaux:  sur les tours $2$-adiques\cite{Doliskani-Schost15}, les tours $\ell$-adiques\cite{DeFeo-Doliskani-Schost13} ; pour lesquels les idées de constructions de tours de corps finis compatibles et le passage de représentation univariée à bivariée ont permis d'obtenir des résultats satisfaisants en terme de complexité.
Les travaux de  \cite{DeFeo-Shost'12} ne seront pas étudiés ici, ces travaux sont en effet complémentaires de ceux étudiés et utilisés ici \cite{Doliskani-Schost15}, \cite{DeFeo-Doliskani-Schost13}.

La présentation des travaux \cite{Doliskani-Schost15}, \cite{DeFeo-Doliskani-Schost13} va se faire dans un ordre anti-chronologique car les travaux de \cite{Doliskani-Schost15} sur les tours $2$-adiques sont plus simples du fait du sujet traité et contiennent des idées similaires reprises de ses antécédents \cite{DeFeo-Doliskani-Schost13}, \cite{DeFeo-Shost'12}.

Une façon naturelle de  construire une extension de corps fini est de trouver un polynôme irréductible défini sur celui-ci et de degré égal à celui de l'extension désirée et d'ensuite quotienter le corps par celui-ci.
Ainsi les éléments des tours d'extension seront toujours représentés comme des polynômes à coefficients dans $\mathbb{F}_q$, la taille des entrées et sorties d'éléments dans $\mathbb{F}_{q^n}$ sera donc $O(n)$.

Les complexités sont calculées en nombre d'opération ($+, \times,\div$) à coût identique dans $\mathbb{F}_q$, indépendamment de la construction de $\mathbb{F}_q$. On notera par $M: \mathbb{N} \rightarrow \mathbb{N}$ l'application telle que la multiplication d'un polynôme de degré $n$ à coefficients dans $R$ coûtera $M(n)$ opérations dans $R$ et telle que $M$ est super-linéaire, en utilisant les résultats de \cite{Cantor-Kaltofen91} on a $M(n)=O(n\log(n) \log \log(n))$.

\section{$\ell$= 2 [Doliskani-Schost]}
La \emph{totalité} des travaux présentés ici proviennent du travail de \cite{Doliskani-Schost15}.

L'idée de base derrière le travail de \cite{Doliskani-Schost15} est de permettre un passage efficace d'une représentation bivariée à une représentation monovariée afin de rajouter à la volée des extensions de la tour compatibles avec ce qui a été fait auparavant.


L'objectif est de calculer une extension du corps fini $\mathbb{F}_q$, avec $q$ impair, de taille $\mathbb{F}_{q^n}$ avec $n=2^k$ et $k \geqslant 1$. 
Comme vu dans \cite{lang2002algebra}{Theorem VI.9.1} et mis en pratique dans \cite{Shoup93} pour calculer une extension quadratique d'un corps fini $\mathbb{F}_q$ on a deux cas à considérer:
\begin{itemize}
\item[$q=1 \bmod 4$] alors tout résidu non quadratique $\alpha$ de $\mathbb{F}_q$ permet de construire le polynôme irréductible $X^n-\alpha$ duquel on déduit alors une construction de $\mathbb{F}_{q^n}$,
\item[$q=3 \bmod 4$] on se ramène au cas $q=1\bmod 4$ en construisant une extension quadratique $\mathbb{F}_{q'}$ de $\mathbb{F}_q$, on peut faire cela en considérant le polynôme irréductible $X^2+1$.
\end{itemize} 

Ainsi pour construire une extension de degré 2 d'un corps fini $\mathbb{F}_q$ un résidu non quadratique $\alpha$ défini sur $\mathbb{F}_q$ est nécessaire. On trouve celui-ci après $O(1)$ essais, on ne va donc pas considérer le coût de calcul de celui-ci. 

De plus il sera considéré pour le corps fini  $\mathbb{F}_q$ que $q=1 \bmod 4$ car le calcul d'une extension quadratique $\mathbb{F}_{q'}$ lorsque $q=3 \bmod 4$ ne rajoute que un facteur $O(1)$ pour le nombre d'opérations sur $\mathbb{F}_q$, cela se justifie par la discussion sur les coûts d'opérations $+,\times, \div$ sur des polynômes de degré $n$ sur $\mathbb{F}_q$.

Dans ce document on ne va pas considérer toutes les opérations sur une tour $2$-adique, seront évidemment considérées les opérations de base telles que la multiplication et l'inversion, mais aussi le Frobenius (crucial pour les applications que l'on va en faire), le calcul de racine carrée, d'autres opérations peuvent être trouvées dans \cite{Doliskani-Schost15}.
\begin{prop}[Doliskani-Schost'15]
Soit $q=1 \bmod 4$, étant donné un résidu non quadratique $\alpha$ de $\mathbb{F}_q$, alors pour $k\geqslant 0$ et $n=2^k$, les temps de calculs pour les opérations dans $\mathbb{F}_q$ énoncés dans le tableau \ref{tab:complexite-degre2} sont vérifiés.
\end{prop}
\begin{figure}
\label{tab:complexite-degre2}
\begin{tabular}{|l|r|}
  \hline
  \textit{Opérations}  & \textit{Coûts} \\
  \hline
  \textit{addition/soustraction}  & $O(n)$ \\
  \textit{multiplication}  & $M(n)+ O(n)$ \\
  \textit{division} & $O(M(n))$ \\
  \textit{Frobenius} & $O(n + \log(q))$ \\
  \textit{Square-root} & $ O(M(n)\log(nq)) (\textit{attendu}) $ \\
  \hline
\end{tabular}
\caption{Coûts des opérations dans $\mathbb{F}_{q^n}$ avec $n=2^k$}
\end{figure}
Il est à noter que pour le calcul d'une racine carrée le coût est attendu, cela est dû à une partie de l'algorithme qui est de type non déterministe (Las Vegas).

Certains résultats du tableau \ref{tab:complexite-degre2} sont connus tels que l'addition et la soustraction, l'obtention des autres sera expliqué par la suite.

\subsection{Représentation des éléments de la tour $\ell$-adique}%titre a retravailler
Dans cette sous-section la tour $2$-adique: $\mathbb{F}_q, \mathbb{F}_{q^2}, \mathbb{F}_{q^{2^2}}, \cdots$ sera écrite:
\[
\mathbb{L}_{0} \subset \mathbb{L}_{1} \subset \mathbb{L}_{2}, ... 
\]
afin d'alléger la notation, avec $\mathbb{L}_{k}=\mathbb{F}_{q^{2^{k}}}$ pour tout entier $k$.


Les éléments de $\mathbb{L}_k$ peuvent être représentés de deux manières différentes: monovariée et bivariée, de la même manière que dans \cite{DeFeo-Shost'12}. 

Soit $\alpha$ un résidu non quadratique de $ \mathbb{L}_0$, à l'aide de celui-ci sont définis les suites de polynômes $T_1,T_2,\cdots, T_k$ en les indéterminées $X_1,X_2, \cdots X_k$ de la façon suivante:
\[
T_1=X_1-\alpha \quad \textit{et} \quad T_k=X_k^2-X_{k-1} \quad \textit{avec $k$ entier naturel non nul}
\]  de la même manière la suite de polynômes $P_1,P_2, \cdots P_k$ est définie à l'aide de $\alpha$ de la façon suivante:
\[
P_k=X_k^{2^k}-\alpha \quad \textit{avec $k$ entier naturel non nul} 
\]

Comme abordé dans \cite{lang2002algebra}, l'égalité entre les idéaux suivants est vérifiée:
\begin{equation}
\label{eq:corres-repre-2}
\langle T_1, T_2, \cdots, T_k \rangle = \langle P_k, X_{k-1}-X_k^2, ..., X_1-X_{k}^{2^{k-1}} \rangle,
\end{equation}
Cet idéal premier est alors appelé $A_k$ et l'on a:
\[
\mathbb{L}_k \cong \mathbb{F}_q[X_1,X_2,\cdots,X_k]/A_k
\]
Soit $k \geqslant 1$  on note $x_k$ l'image de l'indéterminée $X_k$ dans l'anneau résiduel $\mathbb{F}_q[X_1,X_2, ...,X_k]/A_k$, alors par inclusion naturelle de $\mathbb{F}_q[X_1,X_2, ...,X_k]/A_k$ dans $\mathbb{F}_q[X_1,X_2, ...,X_{k+1}]/A_{k+1}$, $x_k$ peut aussi être vu de façon indifférenciée comme un élément de $\mathbb{F}_{q^{2^j}}$ pour $j \geqslant k$.

De plus grâce à l'équation \ref{eq:corres-repre-2}, représenter un élément de $\mathbb{L}_k$ peut être fait soit comme un polynôme en $x_1, ..., x_k$ de degré au plus un en chaque variable grâce au côté gauche de l'équation \ref{eq:corres-repre-2}, soit comme un polynôme en $x_k$ de degré au plus $2^k$ grâce au côté droit de l'équation \ref{eq:corres-repre-2}.

L'écriture par défaut des éléments de $\mathbb{L}_k$ sera donc la représentation univariée. Ainsi un élément de $\mathbb{L}_k$ sera représenté sous la forme $g(x_k)$ avec $g$ un polynôme de $\mathbb{F}_q[X_k]$ de degré strictement inférieur à $2^k$. 

Cependant lors des représentations d'éléments de $\mathbb{L}_k$ il ne va pas être fait des changements de base univariée à base multivariée mais seulement à une base bivarié pour permettre des passages un niveau plus bas dans la tour $2$ adique. On représentera par exemple alors $\mathbb{L}_k$ pour $k\geqslant1$ des deux façons suivantes:
\begin{equation*}
\mathbb{L}_k \cong \mathbb{F}_q[X_k]/\langle P_k(X_k) \rangle \cong \mathbb{F}_q[X_k,X_{k-1}]/\langle P_{k-1}(X_{k_1}), X_k^2-X_{k-1}  \rangle
\end{equation*}  
La $\mathbb{F}_q$ base monomiale de $\mathbb{L}_k$ est donc:
\begin{equation*}
1,x_k,x_k^2,...,x_k^{2^k-1}
\end{equation*}
la $\mathbb{F}_q$ base bivariée de $\mathbb{L}_k$ est alors:
\begin{equation*}
1,x_{k-1},x_{k-1}^2,...,x_{k-1}^{2^{k-1}-1},x_k,x_kx_{k-1},x_kx_{k-1}^2,...,x_kx_{k-1}^{2^{k-1}-1}
\end{equation*}
En pratique ce changement de base se fait de la façon suivante à partir d'une expression $G(x_k)$ d'un élément de $\mathbb{L}_k$ avec $G$ un polynôme de degré inférieur à $2^k-1$:
\begin{equation*}
G(x_k)=G_0(x_k)+x_kG_1(x_k^2)=G_0(x_k)+x_kG(x_{k-1})
\end{equation*}
avec $G_0$ et $G_1$ des polynômes de degré inférieur à $2^{k-1}-1$. Ce changement de base ne requiert aucun calcul arithmétique, ainsi celui-ci pourra être utilisé sans coût supplémentaire et en particulier un élément pourra être considéré indifféremment dans une base bivariée ou monovariée.%a completer

L'inversion d'un élément pourrait être alors considérée dans une base univariée en utilisant l'algorithme étendu d'Euclide étendu. Ainsi le coût d'inversion d'un élément de $\mathbb{L}_k$ serait de $O(M(n)log(n))$ opérations sur $\mathbb{F}_q$ en utilisant les résultats de \cite{vzGJG03}[ch.11]. Une autre technique plus avantageuse est préférée en remarquant un élément $\gamma=G(x_k)$ inversible de $\mathbb{L}_k$ alors en écrivant $G(x_k)=G_0(x_{k-1})+x_kG_1(x_{k-1})$ on obtient:
\begin{align*}
\frac{1}{G(x_k)} &=\frac{1}{G_0(x_{k-1})+x_kG_1(x_{k-1})} \\
 				&= \frac{G_0(x_{k-1})-x_kG_1(x_{k-1})}{G_0(x_{k-1})^2-x_{k-1}G_1(x_{k-1})^2}
\end{align*}
Ainsi le coût d'une inversion d'un élément de $\mathbb{L}_k$ est le coût d'une addition de la multiplication d'éléments de $\mathbb{L}_k$ ainsi que le coût d'une inversion d'éléments dans $\mathbb{L}_{k-1}$, on a donc affaire à un algorithme récursif de coût  $T(k)=T(k-1)+O(M(2^k))$; avec la super linéarité de $M$ on obtient un coût de calcul de $O(M(n))=O(M(2^k))$ opérations en $\mathbb{F}_q$.

\subsubsection*{Calcul du Frobenius}
On aborde ici une partie essentielle à la réussite de l'algorithme de calcul d'isogénies expliqué plus tard, le calcul du Frobenius. Rappelons ici l'objectif d'une telle opération: soit $\gamma $ un élément de $\mathbb{L}_k$ et soit $s=q^d$ avec $d$ un entier quelconque, on veut alors calculer $\gamma^s$ dans $\mathbb{L}_k$. Pour plus de lisibilité on pose $2^k=n$.


On commence par le cas le plus simple: $\gamma=x_k$, ainsi en réécrivant $s=nu+v$, le fait que l'on ait $x_k^n=\alpha $ nous permet d'obtenir $x_k^s=\alpha^ux_k^v$. De plus comme $\alpha$ est un élément de $\mathbb{F}_q$ on obtient $\alpha^u=\alpha^{u \bmod q-1}$, cette opération est donc effectuée en $O(\log(q))$ opérations dans $\mathbb{F}_q$ à l'aide de l'exponentiation rapide. Une méthode rapide est préconisée pour le calcul de $u \bmod q-1$ et $v$, dans un premier temps il faut calculer $\rho=r \bmod n(q-1)$ à l'aide de l'exponentiation rapide puis $u \bmod q-1$ et $v$ sont obtenus comme le quotient et le reste de la division euclidienne de $\rho$ par $n$. Le coût booléen d'une telle opération est de $O(\log(d)\log(nq))$.

Soit  $\gamma=G(x_k)$, alors en écrivant $G(x_k)=g_{n-1}x_k^{n-1}+ \cdots + g_1 x_k + g_0$ on obtient alors:

\begin{align*}
\gamma^r &= g_{n-1}(x_k^r)^{n-1}+ \cdots + g_1 (x_k^r) + g_0 \\
         &= g_{n-1}(\alpha^ux_k^v)^{n-1}+ \cdots + g_1 (\alpha^ux_k^v) + g_0
\end{align*}
Ainsi avec la connaissance de $\alpha^u,v$ le calcul de $\gamma^r$ revient au calcul des $n$ premières puissances de $\alpha^ux_k^v$ et de les remplacer dans l'écriture de $G$. En effet comme $\alpha=x_k^n$ ces puissances sont juste des monômes en $x_k$ et on peut donc les calculer en $O(n)$ multiplications sur $\mathbb{F}_q$. Le coût total pour le Frobenius obtenu est donc de $O(n+ \log(q))$ opérations sur $\mathbb{F}_q$.

Maintenant que le coût du Frobenius a été établi nous allons maintenant voir une autre opération nécessaire pour le calcul d'isogénies, le calcul de racines carrées.

\subsubsection*{Calcul d'une racine carrée}
Les résultats de cette problématique sont issus d'idées développées dans \cite{Doliskani-Schost14} et sont adaptés à cette construction de tour $2$ adique où le coût du Frobenius est très faible. 

Soit $\delta$ un carré non nul de $\mathbb{L}_k$ ayant pour racine carrée $\gamma$ (non connue a priori). Soit $\beta$ dans $\mathbb{F}_q$ défini de la manière suivante:
\begin{align}
\label{eq:trace-square-root}
\beta=T_{\mathbb{L}_k/\mathbb{F}_q}(\gamma)=\sum_{i=0}^{n-1}\gamma^{q^i}&=\gamma(1+\gamma^{q-1}+\cdots+ \gamma^{q^{n-1}-1} ) \\
&=\gamma(1+\delta^{(q-1)/2}+\cdots+ \gamma^{(q^{n-1}-1)/2} \\
&=\gamma \eta
\end{align}
en posant 
\begin{align*}
\eta=1+\delta^{(q-1)/2}+\cdots+ \gamma^{(q^{n-1}-1)/2}
\end{align*}
Sans perte de généralité on peut supposer que l'on a $\eta \neq 0$, si ce n'est pas le cas alors on peut remplacer $\gamma$ par $\gamma c^2$ avec $c$ choisi aléatoirement parmi les inversibles de $\mathbb{F}_q$.
$O(1)$ essais sont nécessaires en moyenne pour obtenir $T_{\mathbb{L}_k/\mathbb{F}_q}(\gamma c^2) \neq 0$. En effet il y a $q^{2^k}/q$ valeurs de $c$ pour lesquelles on a $T_{\mathbb{L}_k/\mathbb{F}_q}(\gamma c^2) = 0$ ainsi la probabilité d'avoir $T_{\mathbb{L}_k/\mathbb{F}_q}(\gamma c^2) \neq 0$ est de $1-(q^{2^k}/q)/(q^{2^k})=1-1/q>1/2$.
Maintenant si l'on met au carré l'équation \ref{eq:trace-square-root} on obtient alors l'équation suivante dans $\mathbb{F}_q$: $\beta^2=\delta \eta^2$. Dés lors si $\eta$ est connu, $\beta$ peut être déduit depuis cette équation et $\gamma$ calculé car on a alors $\gamma=\beta \eta^{-1}$.
Le coût de calcul de $\beta$ est un nombre attendu de $O(\log(q))$ opérations sur $\mathbb{F}_q$ \cite{vzGJG03}[ch 14.5], ainsi seul le coût du calcul de $\eta$ est crucial pour obtenir une complexité satisfaisante.

Soit $\lambda$ un élément de $\mathbb{L}_k$ défini par $\lambda = \delta^{(q-1)/2} $ on a alors la relation suivante avec $\eta$:
\begin{equation*}
\eta = 1 + \lambda + \lambda^{1+q} + \lambda^{1+q+q^2} + \cdots + \lambda^{1+q+q^2+\cdots+q^{n-2}} 
\end{equation*}
Pour $m \geqslant 0$ on définit:
\begin{equation}
\zeta_m=\lambda^{q+q^2+\cdots+q^{m}} 
\end{equation}
que l'on peut construire par récurrence:
\begin{equation*}
\zeta_1=\gamma^q \quad \textit{et} \quad
\zeta_m=
\begin{cases} 
\zeta_{m/2}  \zeta_{m/2}\zeta_{m/2}^{q^{m/2}} \textit{si m est pair }\\
\zeta_{1}  \zeta_{m-1}^{q} \textit{si m est impair}
\end{cases}
\end{equation*}

On définit aussi
\begin{equation*}
\varepsilon_m=\lambda^{q} + \lambda^{q+q^2} + \cdots + \lambda^{q+q^2+\cdots+q^{m}}
\end{equation*} 
on a alors une relation de récurrence pour le calcul de $\varepsilon$:

\begin{equation}
\label{eq:varepsilon}
\varepsilon_1=\lambda^q \quad \textit{et} \quad
\varepsilon_m=
\begin{cases} 
\varepsilon_{m/2} + \zeta_{m/2}\varepsilon_{m/2}^{q^{m/2}} \textit{si m est pair }\\
\varepsilon_{m-1} + \zeta_{m} \textit{si m est impair}
\end{cases}
\end{equation}

Ainsi on obtient $\eta=1+\lambda+\lambda \varepsilon_{n-2}$. 


 
Le calcul de $\lambda$ a un coût de $O(M(n)log(q))$ opérations sur $\mathbb{F}_q$, le calcul de $\varepsilon_1$ et $\eta_1$ se fait quant à lui à l'aide de $O(1)$ fois le Frobenius. Calculons alors par récurrence le coût total, supposons que le calcul de $\varepsilon_m$ et $\zeta_m$ ait déjà été effectué. En utilisant les équations \ref{varepsilon} $\varepsilon_{2m}$ et $\zeta_{2m}$ ou $\varepsilon_{2m+1}$ et $\zeta_{2m+1}$ peuvent être calculés en utilisant $O(1)$ Frobenius et $O(1)$ multiplications sur $\mathbb{F}_{q^n}$. Ainsi $\varepsilon_n$ et $\eta=1+\lambda+\lambda\varepsilon_{n-2}$ sont calculés en $O(M(n)\log(n)+M(n)\log(q))=O(M(n)\log(nq))$ opérations sur $\mathbb{F}_q$.
$\beta$ peut donc être calculé en $O(\log(q))$ opérations sur $\mathbb{F}_q$. Le coût de calcul de $\gamma$ étant négligeable par rapport à celui de $\eta$, on a alors un coût total de $O(M(n)\log(nq))$ opérations sur $\mathbb{F}_q$.

Une implantation de cette construction de tours $2$-adique et des opérations du Frobenius et du calcul de racine carrée ont été réalisé sur SageMath le code est disponible à cette adresse: \url{https://github.com/Hugounenq-Cyril/Two_curves_on_a_volcano/blob/master/Code/extension_corps.sage}.

Il est à noter que en pratique pour le calcul de racine carrée de $\delta$ on peut parfois être amené à considérer un élément de $\mathbb{L}_k$ qui n'est pas un carré. Cela est testé en pratique sur $\beta$ qui est un résidu quadratique si et seulement si $\delta$ l'est. Une astuce consiste alors à multiplier $\delta$ par l'élément générateur $x_{k}$ du corps afin  de multiplier cet élément par un résidu non quadratique, on injecte alors $\delta x_k$ dans $\mathbb{L}_{k+1}$. On applique l'algorithme vu précédemment, le résultat obtenu est alors divisé par $x_{k+1}$ pour obtenir la racine carrée de $\delta$ souhaitée.


\section{$\ell \neq 2$ [De Feo-Doliskani-Schost]}

Le travail présenté ici est issu de \cite{DeFeo-Doliskani-Schost13}. Les différences avec le travail de \cite{Doliskani-Schost15} sont notamment des "push" et "pull" plus compliqués à calculer, ainsi que la discussion faite sur l'existence d'un résidu $\ell$ adique qui ne se résout pas de façon aussi simple que dans \cite{Doliskani-Schost15}. Le calcul du Frobenius n'avait pas été fait dans \cite{DeFeo-Doliskani-Schost13}, cela est fait ici lors d'un travail conjoint avec Luca De Feo, Jérome Plut et Eric Schost, les idées utilisées sont des applications directes de celles de \cite{Doliskani-Schost15}.

Dans cette section la tour $\ell$-adique sera écrite:
\[
\mathbb{F}_q, \mathbb{F}_{q^\ell}, \mathbb{F}_{q^{\ell^2}}, \cdots.
\]
L'objectif ici sera de représenter de façon univariée les éléments du corps $\mathbb{F}_{q^{\ell^i}}$, pour cela on va représenter les éléments de ce corps à l'aide d'éléments de $\mathbb{F}_{q}[X_i]/\langle Q_i \rangle$ pour un polynôme irréductible $Q_i \in \mathbb{F}_q[X_i]$. La principale diffculté par rapport au cas $\ell=2$ sera de construire de tels polynômes, pour cela on sera obligé de construire des extensions de corps d'indice $r$ où des résidus non $\ell$ adiques existent. On construira alors sur ces corps des tours d'extension $\ell$-adique, qui contiendront donc des corps de la taille $r\ell^i$. Il faudra alors utiliser une descente dans les corps pour alors avoir des corps de taille $\ell^i$.

\subsection{Notation des représentations}
Comme dit précédemment les éléments du corps $\mathbb{F}_{q^{\ell^i}}$ peuvent être représentés comme des éléments de $\mathbb{F}_q[X_i]/\langle Q_i \rangle $, avec $Q_i$ un polynôme irréductible. Un élément de $\mathbb{F}_{q^{\ell^i}}$ peut être donc représenté à l'aide de la base:

\[
U_i=(1,x_i,x_i^2,\cdots,x_i^{\ell^i-1})
\]

On peut alors vouloir construire une autre base pour représenter $\mathbb{F}_{q^{\ell^i}}$, ainsi à l'aide d'une représentation de $\mathbb{F}_{q^{\ell^{i-1}}}=\mathbb{F}_q[X_{i-1}]/\langle Q_{i-1} \rangle $, d'un polynôme irréductible $T_{i}$ de $\mathbb{F}_{q^{\ell^{i-1}}}$ de degré $\ell$, on a alors une construction de $\mathbb{F}_{q^{\ell^i}}$: $\mathbb{F}_{q^{\ell^{i}}}\simeq \mathbb{F}_q[X_{i-1},X_{i}]/\langle Q_{i-1}(X_{i-1}), T_i(X_{i-1},X_{i}) \rangle $. Ainsi un élément de $\mathbb{F}_{q^{\ell^{i}}}$ pourra être représenté à l'aide de la base bivariée suivante:

\[
B_i=(1,x_{i-1},x_{i-1}^2,\cdots,x_{i-1}^{\ell^{i-1}-1},x_i^{\ell-1},x_i^{\ell-1}x_{i-1}^{\ell^{i-1}-1})
\] 
Le relèvement correspond à un changement de base de $B_i$ vers $U_i$, la descente correspond à un changement de base de $U_i$ vers $B_i$.

Ces opérations de base servent à faire des plongements dans des corps situés à des niveaux plus bas dans la tour, mais sont aussi un moyen de faire des opérations de base plus rapides à l'aide notamment de l'utilisation de bases univariées comme vu précédemment dans le cas $\ell=2$.

De façon plus générale on notera pour $0 \leqslant j \leqslant i $ $Q_{i,j} \in \mathbb{F}_q(x_j)[X_i]$ le polynôme minimal de $x_i$ sur $\mathbb{F}_{q}(x_j)$, il a pour degré $\ell^{i-j}$. On a en particulier $Q_{i,0}=Q_i$ et $Q_{i,i-1}=T_{i}[X_i,X_{i-1}]$.

\subsection{Construction de tours quasi-cyclotomique}
La première étape consiste à construire une extension de corps $\mathbb{K}_0=\mathbb{F}_q[Y_0]/\langle P_0 \rangle$ telle que la classe résiduelle $y_0$ de $Y_0$ soit un résidu non $\ell$ adique de $\mathbb{K}_0$, on note $r$ le degré de $P_0$.

Par \cite[Th. VI.9.1]{Lang2002algebra} on a pour $i \geqslant 1$ le polynôme $Y_i^{\ell^i}-y_0 \in \mathbb{K}_0[Y_i]$ qui est irréductible, le corps $\mathbb{K}_i=\mathbb{K}_0[Y_i]/\langle Y_i^{\ell^i}-y_0 \rangle$ est ainsi un corps à $q^{r\ell^i}$ éléments. En notant $y_i$ les classes résiduelles de $Y_i$ dans $\mathbb{K}_i=\mathbb{K}_0[Y_i]/\langle Y_i^{\ell^i}-y_0 \rangle$, on a ces représentations de corps qui sont naturellement plongées les unes dans les autres à l'aide de l'isomorphisme $\mathbb{K}_{i+1} \simeq \mathbb{K}_i[Y_{i+1}]/\langle Y_{i+1}^{\ell}-y_i\rangle $, en particulier on a $y_{i+1}^{\ell}=y_i$.


 Pour pouvoir construire un corps de taille $\ell^i$ on utilise le processus de descente suivant qui provient d'un  travail de Shoup:

\begin{prop}[Shoup]
Pour $i \geqslant 0$, soit $x_i$ la trace de $y_i$ sur un sous corps d'indice $r$:
\begin{equation*}
x_i=\sum_{j=0}^{r-1}y_i^{q^{\ell^i}j}
\end{equation*}
On a alors par \cite[Th. 2.1]{Shoup88} $\mathbb{F}_q(x_i)=\mathbb{F}_{q^{\ell^i}}$.
\end{prop}

Montrons donc dans un premier temps comment calculer $\mathbb{K}_0 \simeq \mathbb{F}_{q^r}$.



\subsubsection{Calcul d'une extension contenant un résidu non $\ell$ adique}
Le calcul de $P_0$ se fait à l'aide du $\ell$-ème polynôme cyclotomique $\Phi_{\ell} \in \mathbb{Z}[X_0]$ et de sa factorisation sur $\mathbb{F}_q[X_0]$, par \cite[Th.9]{Shoup93} celle-ci a un coût de $O_e(M(\ell)\log(\ell q))$ opérations sur $\mathbb{F}_q$.

$\Phi_{\ell}$ se scinde sur $\mathbb{F}_q[X_0]$ en facteurs irréductibles de même degré $r$, avec $r$ l'ordre de $q$ dans $(\mathbb{Z}/\ell \mathbb{Z})^{\times}$, il est à noter que $r$ divise $\ell-1$. Soit $R_0$ un des facteurs irréductibles de $\Phi_{\ell}$, alors par construction il existe des résidus non $\ell$ adiques dans $\mathbb{F}_q[X]/\langle R_0 \rangle$. Une fois qu'un tel résidu $y_0$ est trouvé son polynôme minimal $P_0$ sur $\mathbb{F}_q$ de degré $r$ est calculé en $O(r^2)$ opérations sur $\mathbb{F}_q$ d'après \cite[Th.4]{Shoup93}. 
En pratique un tel résidu non quadratique $y_0$ est trouvé aléatoirement après $O(1)$ essais d'après \cite[Lemma 15]{Shoup93}, chaque essai a un coût de $O_e(M(\ell)\log(r)+M(r)\log(\ell)\log(r)+M(r)\log(q))$ opérations sur $\mathbb{F}_q$.
%peut etre affiner les sources

Selon la valeur de $r$ on a différents cas qui se présentent à nous pour le calcul de $Q_{i,j}$. Il y a alors deux cas particuliers qui sont ceux où $\ell \mid q-1$ et $\ell \nmid q-1$ et ensuite le cas général qui a une complexité moins intéressante mais toujours linéaire en la taille de l'extension. Le cas général va d'abord être présenté avant de voir les deux cas particuliers qui partagent une construction commune généralisée par l'idée de Couveignes et Lercier \cite{couveignesLercier2013}.

%\paragraph{$\ell \nmid q+1$}
%Dans ce cas $\Phi_{\ell}$ se scinde en facteurs quadratiques sur $\mathbb{F}_q$ (id est $r=2$). Il est alors nécessaire de prendre $y_0$ tel qu'il soit de norme $1$ sur $\mathbb{F}_q$. Le résultat suivant nous donne alors un moyen de calculer $Q_{i,j}$:
%Apparemment inutile car une methode plus simple est trouvée plus tard...
%\begin{prop}
%Pour $1\geqslant j<i$ $Q_{i,j}$ vérifie:
%\begin{equation*}
%Q_{i,j}(X_i)=Y^{\ell^{i-j}-Y^{-\ell^{i-j}}-x_j \bmod Y^2-X_iY+1
%\end{equation*}
%\end{prop}
%
%\begin{proof}
%Voir \cite{DeFeo-Doliskani-Schost13}
%\end{proof}

\subsubsection{Cas général}
Dans ce cas on ne fait aucune hypothèse sur la décomposition de $\phi_{\ell}$ dans $\mathbb{F}_q[X]$. De plus comme $r=[\mathbb{K}_0:\mathbb{F}_q]$ et qu'il divise $\ell-1$ alors $r$ est premier avec $\ell$. Ainsi $Q_i$ est bien le polynôme minimal de $x_i$ sur $\mathbb{F}_q$, de la même manière $Q_{i,j}$ est le polynôme minimal de $x_i$ sur $\mathbb{K}_j$. Ceci nous permet donc de remplacer le corps de base $\mathbb{F}_q$ par $\mathbb{K}_0$. Le calcul se fera sur la base d'opérations dans $\mathbb{K}_0$, un facteur $O(M(r)\log(r))$, qui correspond au coût d'opérations dans $\mathbb{K}_0$, sera ajouté pour avoir le coût exprimé en nombres d'opérations sur $\mathbb{F}_q$.


Pour $i \geqslant 0$, comme $\mathbb{K}_i=\mathbb{K}_0[Y_i]/\langle Y_i^{\ell^i} - y_0 \rangle$, chacun de ses éléments peut être représenté dans une base $ \{y_i^e | 0 \leqslant e < \ell^i \} $, ainsi l'élément $x_i$ s'écrit dans cette base:
\begin{equation*}
x_i= \sum_{0 \leqslant j \leqslant r-1}y_i^{q^{\ell^i}j \bmod \ell^i}y_0^{q^{\ell^i}j / \ell^i}
\end{equation*}
On peut alors représenter $\mathbb{K}_i$ à l'aide des deux bases univariées $\mathbb{K}_i$: $\{ y_i^e | 0 \leqslant e < \ell^i\}$ et $\{x_i^e | 0 \leqslant e < \ell^i\}$. On va alors montrer comment calculer l'isomorphisme entre les deux bases univariées de $\mathbb{K}_i$: $\{ y_i^e | 0 \leqslant e < \ell^i\}$ et $\{x_i^e | 0 \leqslant e < \ell^i\}$:
\begin{equation*}
\Psi_i: \mathbb{K}_i=\mathbb{K}_0[Y_i]/\langle Y_i^{\ell^i}-y_0 \rangle \to \mathbb{K}_0[X_i]/\langle Q_{i,0} \rangle,
\end{equation*}
 %on pourra alors employer la méthode de descente vue dans \cite{Shoup88} sur la représentation image de l'isomorphisme de corps calculée pour avoir une représentation de $\mathbb{F}_{q^{\ell^i}}$. 
 Lors du calcul de $\Psi_i$, le calcul nécessaire de $Q_{i,0}$ et $Q_{i,i-1}$ sera effectué. On se servira de $\Psi_i$ pour calculer la descente et le relèvement entre la base monomiale en $x_i$ et la base bivariée en $(x_{i-1},x_i)$.
 
 On va factoriser $\Psi_i$ en isomorphismes élémentaires:
 \begin{equation*}
 \Psi_{i,j}:\mathbb{K}_{j}[X_i]/\langle Q_{i,j} \rangle \to \mathbb{K}_{j-1}[X_i]/ \langle Q_{i,j-1} \rangle, \quad j=i, \cdots,1. 
 \end{equation*}
 
On commence donc avec $i=j$, dans ce cas-là on pose $Q_{i,i}=X_i-x_i \in \mathbb{K}_i[X_i]$, de telle sorte que $\mathbb{K}_i=\mathbb{K}_i[X_i] / \langle Q_{i,i} \rangle $.

Considérons maintenant le cas $j \leqslant i$ et supposons que $Q_{i,j}$ est connu. $\Psi_{i,j}$ va être alors factorisé en $\Phi_{i,j}'' \circ \Phi_{i,j}' \circ \Phi_{i,j}$, il est alors nécessaire d'introduire l'isomorphisme :
\begin{equation*}
\varphi_j:\mathbb{K}_j \to \mathbb{K}_{j-1}[Y_j]/\langle Y_j^{\ell}-y_{j-1} \rangle
\end{equation*}
Cet isomorphisme est dans son sens direct une descente de la base monomiale en $y_j$ vers la base bivariée en $(y_j,y_{j-1})$ et son inverse est donc un relèvement, aucun des deux ne nécessite des opérations arithmétiques, ces deux opérations sont exactement les mêmes que celles décrites dans le cas simple où $r=1$ \ref{par:easycase}. L'isomorphisme suivant peut alors être calculé:
\begin{equation*}
\Phi_{i,j}: \mathbb{K}_j[X_i]/\langle Q_{i,j} \rangle \to \mathbb{K}_{j-1}[Y_j,X_i]/\langle Y_j^{\ell}-y_{j-1},Q_{i,j}^* \rangle , 
\end{equation*} 
avec $Q_{i,j}^*$ obtenu en appliquant $\varphi_j$ à tous les coefficients de $Q_{i,j}$. Comme $\Phi_{i,j}$ consiste en une application coefficient par coefficient de $\phi_j$ alors appliquer $\Phi_{i,j}$ ou son inverse ne coûte aucune opération arithmétique.

On change ensuite l'ordre entre $Y_j$ et $X_i$, cela permet de déduire l'existence de $S_{i,j} \in \mathbb{K}_{j-1}[X_j]$ et d'un isomorphisme:

\begin{equation*}
\Phi_{i,j}': \mathbb{K}_{j-1}[Y_j,X_i]/ \langle Y_j^{\ell}-y_{j-1},Q_{i,j}^* \rangle \to \mathbb{K}_{j-1}[X_i,Y_j]/\langle Q_{i,j-1}, Y_j-S_{i,j}\rangle
\end{equation*}
avec $\deg(Q_{i,j}^*,X_i)=\ell^{i-j}$ et $\deg(Q_{i,j-1},X_i)=\ell^{i-j+1}$.

\begin{lem}
A partir de $Q_{i,j}^*$, $Q_{i,j-1}$ et $S_{i,j}$ sont calculés en $O(M(\ell^{i+1})\log(\ell^i))$ oéprations sur $\mathbb{K}_0$. Une fois  $Q_{i,j-1}$ et $S_{i,j}$ calculés on peut appliquer $\Phi_{i,j}'$ ou son inverse en $O(M(\ell^{i+1}))$ opérations sur $\mathbb{K}_0$
\end{lem}

\begin{proof}
Voir \cite{DeFeo-Doliskani-Schost13}.
\end{proof}

Le dernier isomorphisme $\Phi_{i,j}''$ s'écrit trivialement:
\begin{equation*}
\Phi_{i,j}'': \mathbb{K}_{j-1}[X_i,Y_j]/\langle Q_{i,j-1}, Y_j-S_{i,j}\rangle \to \mathbb{K}_{j-1}[X_i]/\langle Q_{i,j-1} \rangle 
\end{equation*}
il "oublie" la variable $Y_j$, il ne nécessite aucune opération arithémtique.

En faisant varier $j$ de $1$ à $i$ on peut alors calculer $Q_{i,i-1}$ jusqu'à $Q_{i,0}$ en $O(i^2M(\ell^{i+1})\log(\ell))$ opérations sur $\mathbb{K}_0$. En composant les isomorphismes $\Psi_{i,j}$, on en déduit que $\Psi_i$ ou son inverse peut être calculé en $O(iM(\ell^{i+1}))$ opérations dans $\mathbb{K}_0$.

Montrons alors comment est calculé un relèvement à partir des éléments sus-mentionnés. Soit $A$ un élément écrit dans la base bivariée $(x_i,x_{i-1})$, autrement dit $A$ est dans $\mathbb{K}_0[X_{i-1},X_i]/\langle Q_{i-1}, T_{i} \rangle$. On applique alors $\Psi_{i-1}$ à ses coefficients en $x_i^0,x_i,\cdots, x_i^{\ell-1}$ pour réécrire $A$ comme un élément de
\begin{equation*}
\mathbb{K}_0[Y_{i-1},X_i]/\langle Y^{\ell^{i-1}}-y_{i-2},T_i \rangle = \mathbb{K}_{i-1}[X_i]/\langle Q_{i,i-1} \rangle.
\end{equation*}
On applique alors $\Psi_{i,i}^{-1}:\mathbb{K}_{i-1}[X_i]/\langle Q_{i,i-1} \rangle \to \mathbb{K}_i[X_i]/\langle Q_{i,i} \rangle$ pour avoir l'image de $A$ dans $\mathbb{K}_i$. Enfin on applique $\Psi_{i}:\mathbb{K}_i[X_i]/\langle Q_{i,i} \rangle \to \mathbb{K}_0[X_i]/ \langle Q_i \rangle $ pour avoir $A$ exprimé dans une base monomiale en $x_i$. Faire le push revient à faire l'inverse de ses opérations.


\subsubsection{\boldmath $\ell \mid q-1$}\label{par:easycase}
Maintenant on étudie le cas où $r=1$ lorsque $\Phi_{\ell}$ se scinde en facteurs linéaires sur $\mathbb{F}_q$. Le polynôme $P_0$ s'écrit donc $Y_0-y_0$ avec $y_0$ un résidu non $\ell$-adique de $\mathbb{F}_q$. Comme le calcul de la factorisation de $\Phi_{\ell}$ n'est pas nécessaire, on obtient alors une complexité de $O_e(\log(q))$ opérations sur $\mathbb{F}_q$.

Ce cas-ci permet de ne pas utiliser de méthode de descente car on a $\mathbb{K}_0 \simeq \mathbb{F}_q$ et par conséquent on obtient pour $i \geqslant 0 $   $\mathbb{K}_i \simeq \mathbb{F}_{q^{\ell^i}}$, et donc $x_i=y_i$. Les polynômes obtenus alors sont: $Q_i=X^{\ell^i}-y_0$ et $T_i=X_i^{\ell}-X_{i-1}$. Le relèvement et la descente ne sont que des opérations arithmétiques. En effet pour $F=\sum_{0 \leqslant j \leqslant \ell^{i+1}}f_jx_{i+1}^j$ le relévement réécrit cet élément dans une base $x_i,x_{i+1}$ en utilisant la relation suivante:
\begin{equation*}
x_{i+1}^j=x_i^{j / \ell}x_{i+1}^{j \bmod \ell} 
\end{equation*}
tandis que la descente réécrit un élément éxprimé dans une base $x_i, x_{i+1}$ en un élément exprimé dans la base $x_{i+1}$ en utilisant la relation suivante:
\begin{equation*}
\quad x_i^ex_{i+1}^f=x_{i+1}^{e\ell + f} 
\end{equation*}
 

\subsubsection{Couveignes and Lercier's ideas to explicitely compute Q and T}
Ici vont être présentés des travaux de Couveignes et Lercier \cite{couveignesLercier2013} permettant de calculer des tours $\ell$ adiques.

Dans un premier temps nous allons voir les idées de Couveignes et Lercier, puis dans un second temps nous allons voir un exemple, celui où $\ell$ est un nombre premier différent de $p$ qui ne divise pas $q-1$, cet exemple a été choisi de par l'intérêt de l'objet qu'il manipule: un cycle d'isogénie. Ce sera donc l'occasion pour le lecteur de se familiariser avec les graphes d'isogénies. Un autre cas où le polynôme modulaire se scinde en facteurs linéaires et quadratiques sur $\mathbb{F}_q$ n'est pas traité ici mais l'est dans \cite{DeFeo-Doliskani-Schost13}.

Présentons donc l'idée directrice avant de voir un exemple. Soient $G,G'$ deux $\mathbb{F}_q$ groupes algébriques intègres de même dimension, $\phi: G' \to G$ un morphisme de groupe surjectif et séparable de degré $\ell$, alors l'ensemble des points $x$ de $G$ de fibré $G'_x$ de cardinal $\ell$ est un ensemble ouvert non vide $U \subset G$. 
L'homomorphisme induit de groupes finis $\phi_{\mid \mathbb{F}_q}: G'(\mathbb{F}_q) \to G(\mathbb{F}_q)$ permet lorsqu'il est non surjectif de définir des extensions algébriques de $\mathbb{F}_q$ à partir des fibrés de points de $G(\mathbb{F}_q)$ ne se trouvant pas dans l'image de $\phi_{\mid \mathbb{F}_q}$. 

Pour construire ces extensions algébriques on applique sur un fibré irréductible issu de points ne se trouvant pas dans l'image de $\phi_{\mid \mathbb{F}_q}$ une projection linéaire cela permet alors de  déterminer un polynôme irréductible de degré divisant $\ell$ (espéré égal à $\ell$). Si l'on peut itérer cette méthode avec un nouvel homomorphisme: $\phi':G'' \to G'$ qui vérifie les même conditions que vu précédemment, alors on peut déterminer des extensions $\ell$-adiques de $\mathbb{F}_q$.

On va donc voir une application de cette méthode appliquée sur un cycle d'isogénies de degré $\ell$.

\paragraph{Towers from elliptic curves}
%Dans cette partie on revoit les constructions de Couveignes et Lercier \cite{couveignesLercier2013} et l'on montre en particulier les passages pertinents pour une construction de tour.

Soit $\ell$ un nombre premier différent de $p$ qui ne divise pas $q-1$, soit $E_0$ une courbe elliptiqe de cardinal sur $\mathbb{F}_q$: $|E_0(\mathbb{F}_q)|=\ell^e a$ avec $e \geqslant 1$ et $a \wedge \ell =1$. Une telle courbe existe d'après le théorème de Hasse seulement si $\ell \leqslant q+2\sqrt{q} +1$.  
Pour mettre en pratique l'idée de Couvegines et Lercier \cite{couveignesLercier2013} expliqué plus haut on travaille alors avec des isogénies $\mathbb{F}_q$-rationnelles séparables de degré $\ell$. Il est donc naturel de s'intéresser à l'action du Frobenius sur la $\ell$ torsion afin de déterminer ces isogénies. On étudie donc le polynôme caractéristique du Frobenius:
%peut etre enoncer cela sous forme de proposition ??
\begin{equation*}
X^2-t_{\pi}X+q=(X-1)(X-q) \bmod \ell
\end{equation*}
qui dans ce cas précis se scinde modulo $\ell$.


En effet par hypothèse sur le cardinal de la courbe on a: $|E|=q+1-t_{\pi}=0 \bmod \ell$, on peut alors montrer que $d_{\pi}=t_{\pi}^2-4q=(q+1)^2-4q=(q-1)^2 \bmod \ell$. De plus le fait que $\ell \nmid q-1$ nous permet d'énoncer que le discrimant $d_{\pi}$ du polynôme charactéristique de $\pi$ est différent de $0$ modulo $\ell$, le Frobenius admet donc deux valeurs propres. Enfin on montre que les deux valeurs propres sont $1$ et $q \bmod \ell$ car on a $t_{\pi}=q+1 \bmod \ell$.

Ainsi à l'aide de ces deux valeurs propres on peut déterminer des sous-groupes cycliques de la $\ell$ torsion. 

Comme on ne peut pour le moment que travailler sur $\mathbb{F}_q$, le but de cette méthode étant de construire une extension $\ell$-adique de $\mathbb{F}_q$, on ne peut utiliser que le sous-groupe cyclique de taille $\ell^e$ associé à la valeur propre $1$. On construit donc à l'aide de ce sous groupe une $\ell$ isogénie qui a pour image $E_1$. 

De manière plus générale  on peut affirmer que la $\ell$ torsion de $E_0$ n'est pas entièrement rationnelle car on a $\ell \nmid q-1$, la $\ell$ torsion ne peut donc seulement être cyclique. Ainsi cet argument et la cardinalité de la courbe permettent d'affirmer que l'on a un sous groupe cyclique d'ordre $\ell^e$ défini sur $\mathbb{F}_q$, on peut donc aussi travailler directement avec des $\ell^i$ isogénies avec $1 \leqslant i \leqslant e$.

Le sous groupe cyclique de la $\ell$ torsion défini comme le noyau de $\pi -q$ génère une $\ell$ isogénie rationnelle, car l'ensemble des points de son noyau est stable sous l'action du Frobenius. Ainsi $E_0$ admet deux courbes $\ell$ isogènes, cet argument peut se répéter à ses courbes $\ell$isogènes car celles-ci ont la même cardinalité. On a donc défini un cycle de $\ell$isogénies. Ainsi à partir de $E_0$ on dénote $E_1$ la courbe codomaine de la $\ell$-isogénie qui a pour noyau le sous groupe cyclique de la $\ell$ torsion égal à $\ker (\pi_{|E_0[\ell]} - \mathrm{Id}_{|E_0[\ell]})$, de manière générale on notera $E_{i+1}$ le codomaine de l'isogénie engendré par $\ker (\pi_{|E_i[\ell]} - \mathrm{Id}_{|E_i[\ell]})$ . Ce cycle a une longueur finie comme l'énonce le lemme suivant issu de \cite{DeFeo-Doliskani-Schost13}:

\begin{lem}
Soient $E_0,E_1, \cdots$ les courbes elliptiques définies comme ci-dessus, alors il existe un $n \in O(\sqrt{q}\log(q))$ tel que $E_n$ est isomorphe à $E_0$.
\end{lem}

\begin{proof}
La preuve se sert du théorème de Minkowski qui nous donne une borne sur le nombre de groupe de classe de $\mathbb{Q}[X]/(X^2-t_{\pi}X+q)$. Les détails techniques de cette preuve ne sont pas le sujet de cette partie.
\end{proof}

Ainsi on travaillera avec les courbes $E_i$ avec $i$ modulo la longueur du cycle: $n$. En pratique pour faire cela il faut prendre en compte l'isomorphisme $E_n \simeq E_0$, cet isomorphisme peut être calculé grâce aux formules de Weierstrass et cela n'ajoute pas de grosses opérations arithmétiques à effectuer.

On va donc se servir de ce cycle pour calculer des extensions de façon incrémentale, ainsi on pourra travailler sur des tours dont on ne connaît pas à priori le degré d'extension. 

Montrons comment construire un polynôme irréductible $f(x) \in \mathbb{F}_q[x]$ de degré $\ell$. Soient  les deux entiers $\lambda_{e+1}, \mu_{e+1}$ tels que 
\begin{align*}
\lambda_{e+1}=1 \bmod \ell^{e} \quad , \quad \mu_{e+1}=q \bmod \ell^e, \\
X^2-tX+q = (X-\lambda_{e+1})(X-\mu_{e+1}) \bmod \ell^{e+1} 
\end{align*}
On note $\lambda_{e+1}=1+\ell^el'$ avec $l' \wedge \ell = 1$. Dans la suite on notera tout simplement $\lambda=\lambda_{e+1}$ et $\mu = \mu_{e+1}$. On pose maintenant:
\begin{equation*}
(\ell,\pi-\lambda)=\mathfrak{l}.
\end{equation*}
C'est un idéal inversible, noyau de l'isogénie $\phi_{i}:E_i \to E_{i+1}$. Notre but est de travailler avec un élément de $\ell$ torsion de $E_{i+1}$ pour lequel on est sûr de ne pas avoir d'antécédent par la $\ell$ isogénie défini dans $\mathbb{F}_q$. Le $\ell$ sous-groupe de Sylow de $E_{i+1}(\mathbb{F}_q)$ est le noyau de $\mathfrak{l}^e=(\ell^e,\pi - 1)$ et il est cyclique, on travaille donc avec $A$ un générateur de $\mathfrak{l}^e$ avec $\ell^e \neq 2$. En écrivant l'isogénie $\phi_i$ comme une fraction rationnelle : 
\begin{align*}
\phi_{i}:E_i \to E_{i+1}, \\
(x,y) \mapsto \left( \frac{f_{i}(x)}{g_{i}(x)} ,y \left( \frac{f_{i}(x)}{g_{i}(x)} \right) ^{'} \right)
\end{align*}
%^' \right)

avec $g_i$ le carré du polynôme de degré $\frac{\ell-1}{2}$  qui s'annule sur l'abscisse des points du noyau de $\phi_{i}$, on  définit alors $ T_{1}(X)=f_{1}(x)-g_{1}(x)x(A)$. Soit $B \in E_i(\overline{\mathbb{F}_q})$ tel que $\phi_i(B)=A$, alors  $B$ est un générateur de $\mathfrak{l}^{e+1}=(\ell^{e+1},\pi-\lambda)$. En particulier on a:
\begin{equation*}
\pi(B)=\lambda B
\end{equation*}
avec l'ordre de $\lambda=1+\ell^el'$ dans $(\mathbb{Z}/\ell^{e+1}\mathbb{Z})^*$ qui est $\ell$. Ainsi l'orbite de Galois de $B$ est de taille $\ell$ et le polynôme $T_{1}(X)$,dont les racines appartiennent au fibré $\phi_i^{-1}(A)$, est irréductible. 

Pour définir des extensions successives on prend alors les pré images par les isogénies $\phi_{i_1},\phi_{i_2}, \cdots$. On définit donc la suite de polynômes suivante pour construire une tour d'extension $\ell$-adique en notant $\eta =x(A)$ l'abscisse d'un générateur de la $\ell^e$-torsion de $E_0$:
\begin{align*}
T_0=f_{-1}(X_1)-\eta g_{-1}(X_1), \\
T_i=f_{-i-1}(X_{i+1})-X_{i+1}g_{-i-1}(X_{i+1})
\end{align*}

On aurait pu définir un polynôme irréductible de degré $\ell^{\delta}$ directement en considérant une isogénie $\psi:E_i \to E_{i+\delta}$ de degré $\ell^{\delta}$ et qui a pour noyau $\mathfrak{d}=(\ell,\pi-\lambda_{e+\delta})^{\delta}=\mathfrak{l}^{\delta}$ avec $\lambda_{e+\delta}, \mu_{e+\delta}$ tels que 
\begin{align*}
\lambda_{e+\delta}=1 \bmod \ell^{e} \quad , \quad \mu_{e+\delta}=q \bmod \ell^e, \\
X^2-tX+q = (X-\lambda_{e+\delta})(X-\mu_{e+\delta}) \bmod \ell^{e+\delta} .
\end{align*}

\paragraph{Existence de telles courbes}
Il est naturel de se demander sous quelles conditions on peut trouver une courbe elliptique $E$ définie sur $\mathbb{F}_q$ de cardinal multiple de $\ell$ avec $\ell \wedge q =1$ et $\ell \nmid q-1$. Par le théroème de Hasse si l'on a $\ell \leqslant 2\sqrt{q}$ alors il existe au moins deux entiers multiples de $\ell$ dans l'intervalle $[q+1-2\sqrt{q}, q+1+2\sqrt{q} ] $, on a alors au moins un des deux entiers qui n'est pas congru à $1$ modulo $q$. Il existe donc au moins une courbe de cardinal multiple de $\ell$. L'objectif est donc d'expliciter une borne inférieure sur ce nombre de courbes. Pour cela on utilise les résultats de Lenstra \cite{lenstra1987} qui ont été étendus à notre cas par Howe \cite{howe1993}.

\begin{cor}[Howe]
\label{cor:Howe:densite}
Soit $W(\mathbb{F}_q)$ l'ensemble des courbes elliptiques sous forme de Weierstrass définies sur $\mathbb{F}_q$. Soit $\mu_w$ la mesure uniforme sur cet ensemble et $\ell$  tel que $\ell \nmid q-1$ alors la densité $r(\ell,q)$ de courbes elliptiques sous forme de Weierstrass ayant un point $\mathbb{F}_q$-rationnel d'ordre $\ell$ vérifie:
\begin{equation*}
|r(\ell,q)-\frac{1}{\ell-1}| \leqslant \frac{4\ell(\ell+1)}{(\ell-1)\sqrt{q}}.
\end{equation*}
\end{cor}
Ainsi pour limiter le terme d'erreur pour l'évaluation de $r(\ell,q)$ on se place dans le cas où l'on a $4\ell \leqslant q^{\frac{1}{4}}$, on aura alors l'erreur: $\frac{4\ell(\ell+1)}{(\ell-1)\sqrt{q}} \leqslant \frac{1}{2 \ell} $. Dans ce cas-là on pourra donc considérer qu'il faut répéter $\ell$ fois le choix d'une courbe elliptique sous forme de Weierstrass avant de trouver une courbe de cardinal divisible par $\ell$.

On peut alors énoncer le résultat suivant issu de \cite{DeFeo-Doliskani-Schost13} qui donne le temps de calcul de construction d'une tour $\ell$-adique:

\begin{thm}
Pour $\ell$ premier avec $\ell \leqslant q^{1/4}$, $\ell \wedge q = 1$ et $\ell \nmid q-1$. L'initialisation de la tour requiert $O_e(\ell \log^5(q)+\ell^3)$ opérations binaires; la construction du $i$-ème étage de la tour requiert $O_e(\ell^2+M(\ell)\log(\ell q)+M(\ell^i)\log(\ell^i))$ opérations sur $\mathbb{F}_q$
\end{thm} 

\begin{proof}
Le coût de calcul de la cardinalité d'une courbe elliptique est de $O(\log q)^{5+\epsilon(q)}$ opérations élémentaires en utilisant l'algorithme de Schoof. On a cette opérations à répéter jusqu'à $\ell$ fois pour espérer trouver une courbe de cardinal multiple à $\ell$ d'après le corollaire \ref{cor:Howe:densite}. Le coût total de calcul d'une courbe de cardinalité multiple de $\ell$ est donc de  $O(\ell\log q^{5+\epsilon(q)})$. Pour calculer les $j$-invariants du graphe cyclique de $\ell$-isogénies on a besoin de calculer le $\ell$-ème polynôme modulaire $\Phi_{\ell}$ modulo $p$, il est d'abord calculé sur $\mathbb{Z}$ avec un coût de $\tilde{O}(\ell^3)$ opérations binaires d'après \cite{enge2008} puis réduit modulo $p$. On doit aussi calculer un point d'ordre $\ell^e$ pour cela on choisit au hasard un point de la courbe et on le multiplie par le cofacteur du cardinal de la courbe, on calcule cela en $O_{e}(\log(q))$ opérations sur $\mathbb{F}_q$.

Pour construire le $i$ ème niveau on doit tout d'abord calculer l'équation de $E_{-i}$, pour cela on évalue le $\ell$-polynôme modulaire $\phi_{\ell}$ en $j(E_{-(i-1)})$ en $O(\ell^2)$ opérations, on le factorise alors sur $\mathbb{F}_q$ avec $O(M(\ell)\log(\ell q))$ oéprations d'après \cite[Ch.14]{vzGJG03}. Comme on travaille dans le cas particulier d'un graphe de $\ell$-isogénies cyclique ce polynôme n'a que deux racines définies sur $\mathbb{F}_q$: $j(E_{-i})$ et $j(E_{-i+2})$. On sélectionne alors une courbe arbitraire de $j$-invariant: $j(E_{-i})$. Vu que l'on connaît la cardinalité de la courbe alors on calcule un point $P$ d'ordre $\ell$ en $O_{e}(\log(q))$ opérations sur $\mathbb{F}_q$. On calcule alors à l'aide des formules de Vélu l'isogénie $\phi_{-i}$ qui a pour noyau le groupe engendrée par $P$ de cette isogénie on déduit les polynômes $T_{i},Q_{i}$ en $O(M(\ell^i)\log(\ell^i))$ d'après \cite[Algorithme 1]{DeFeo-Doliskani-Schost13}.
\end{proof}

Comme remarqué dans \cite{DeFeo-Doliskani-Schost13} on peut éviter à chaque ajout d'un nouvel étage à la tour d'extension le coût du calcul du $j$-invariant de la courbe isogène en calculant la totalité du cycle juste après avoir trouve une courbe se situant dans un graphe de $\ell$-isogénies cyclique. Pour calculer ce cycle on commence par calculer sur $E_0$ l'isogénie qui a pour noyau la $\ell$ torsion cyclique rationnelle. Pour cela on calcule en $O_e(\log(q))$ opérations sur $\mathbb{F}_q$ un point de $\ell$-torsion rationnel, on calcule alors la courbe image et par conséquent son $j$-invariant en $O(M(\ell))$ opérations sur $\mathbb{F}_q$ en utilisant les formules de Vélu, on répète cette opération jusqu'à retomber sur le $j$-invariant $j(E_0)$, comme vu précédemment on a une borne théorique pour la longueur du cycle de $O(\sqrt{q}\log(q))$. Ainsi on a une complexité totale de $O((M(\ell) + \log(q)) \sqrt{q} \log(q))$, cependant en pratique la longueur du cycle est petite, on peut alors utiliser cette amélioration.


\paragraph{Relèvement et descente}
Les différentes constructions de tours $\ell$-adiques à l'aide de fibrés irréductibles s'appliquent aussi au cas où l'on a $\Phi_{\ell}$ qui se scinde en produits de facteurs irréductibles de degré $1$ et $2$ en plus du cas précédent comme cela est développé plus en détail dans \cite{DeFeo-Doliskani-Schost13}. Ainsi toutes ces constructions à l'aide de fibrés irréductibles ont une structure commune qui nous permet de calculer le relèvement et la descente de la même manière. En rennomant les variables $(X_i,X_{i-1})$ par $(X,Y)$ et les polynômes $(Q_{i-1},Q_{i},T_i)$ par $(R,S,T)$, alors l'extension au niveau $i$ s'écrit:
\begin{equation}
\mathbb{F}_q[Y]/\langle S(Y) \rangle  \quad \textit{et } \quad \mathbb{F}_q[X,Y]/ \langle R(X), T(X,Y) \rangle
\end{equation}
avec $R$ de degré $\ell^{i-1}$ et $S$ de degré $\ell^i$ et $T(X,Y)$ de la forme $f(Y) -Xg(Y)$ avec $\deg(f)=\ell$ et $deg(g)< \ell$ et $ \mathrm{pgcd}(g,f)=1$. En particulier on peut avoir $g=1$. Dans le reste de cette section on considèrera $f,g$ et le degré $\ell$ fixés.

Le relèvement est le passage d'une base bivariée vers une base monovariée, cela correspond donc au passage de la base de droite à celle de gauche. La descente est l'opération inverse. En utilisant la forme particulière de $T$ on montre alors qu'on peut réduire ces opérations à la composition et la décomposition de fonctions rationnelles.

\paragraph{Relèvement}
Soit $P$ dans $\mathbb{F}_q[X,Y]$ et $n$ dans $\mathbb{N}$ avec $\deg(P,X)<n$. On définit $P[f,g,n]$ comme étant:
\[
P[f,g,n]=g^{n-1}P(\frac{f}{g},Y)\in \mathbb{F}_q[X,Y].
\]
Si par exemple $P=\sum_{i=0}^{n-1}p_i(Y)X^i$ alors $P[f,g,n]=\sum_{i=0}^{n}p_if^ig^{n-1-i}$. On va donc montrer comment le calculer puis on va relier ce polynôme au relèvement d'éléments de la tour.

\begin{algorithm}
\caption{\label{alg:Compose}Compose}
\begin{algorithmic}[1]
\REQUIRE $P \in \mathbb{F}_q[X,Y], f,g \in \mathbb{F}_q[Y], n \in \mathbb{N}$
\ENSURE $P[f,g,n] \in \mathbb{F}_q[X,Y]$.
\IF{ $n=1$}
 \RETURN $P$
\ELSE
\STATE $m \gets \left\lceil n/2 \right\rceil   $
\STATE Soient $P_0,P_1$ tels que $P=P_0 + X^{m} P_1$
\STATE $Q_0 \gets \textrm{Compose}(P_0,f,g,m))$
\STATE $Q_1 \gets \textrm{Compose}(P_0,f,g,n-m))$
\STATE $Q \gets Q_0g^{n-m}+Q_1f^m$
\RETURN $Q$ 
\ENDIF
\end{algorithmic}
\end{algorithm}


\begin{thm}\label{thm:compose}
Avec les entrées $P,f,g,n$ tels que $\deg(P,X)<n$ et $\deg(P,Y)<\ell$, l'algorithme \ref{alg:Compose} calcule $Q=P[f,g,n]$ avec un coût de $O(M(\ell n)\log(n))$ opérations sur $\mathbb{F}_q$.
\end{thm}

\begin{proof}
Voir \cite{DeFeo-Doliskani-Schost13}.
\end{proof}

\begin{cor}
Au niveau $i$ de la tour d'extension $\ell$-adique, un relèvement coûte $O(M(\ell^i)\log(\ell^i))$ opérations sur $\mathbb{F}_q$
\end{cor}

\begin{proof}
Soit $\alpha$ un élément de $\mathbb{F}_{q^{\ell^i}}$ alors $\alpha$ peut être représenté dans une base bivariée, c'est à dire par un polynôme $A(X,Y)$ avec $\deg(A,X)<n=\ell^{i-1}$ et $\deg(A,Y)<\ell$. On calcule alors à l'aide de l'algorithme \ref{alg:Compose} le polynôme univarié en $Y$ $A^*=A[f,g,n]$ et $\gamma = g^{n-1}$ le tout à l'aide de $O(M(\ell^i)\log(\ell^i))$ opérations dans $\mathbb{F}_q$. Le relevé de $\alpha$ sera alors $A^*/\gamma$ modulo $S$. L'inversion de $\gamma$ coûte $O(M(\ell n)\log(\ell n))$ opérations, et la multiplication ajoute un cout de $O(M(\ell n))$.
\end{proof}

\paragraph{Descente}
On a donc le problème inverse de celui traité dans le théorème \ref{thm:compose} pour faire une descente. On a donc en entrée un polynôme $Q \in \mathbb{F}_q[y]$, on veut alors retrouver $P\in \mathbb{F}_q[X,Y]$ tel que $Q=P[f,g,n]$. 
\begin{algorithm}
\caption{\label{alg:Decompose}Decompose}
\begin{algorithmic}[1]
\REQUIRE $Q,f,g,h \in \mathbb{F}_q[Y], n \in \mathbb{N}$
\ENSURE $P[f,g,n] \in \mathbb{F}_q[X,Y]$.
\IF{ $n=1$}
 \RETURN $Q$
\ELSE
\STATE $m \gets \left\lceil n/2 \right\rceil   $
\STATE $u \gets h^{n-m}=1/g^{n-m} \bmod f^m$
\STATE $Q_0 \gets Qu \bmod f^m$
\STATE $Q_1 \gets (Q-Q_0g^{n-m}) \mathrm{div} f^m$
\STATE $P_0 \gets \textrm{Decompose}(Q_0,f,g,h,m))$
\STATE $P_1 \gets \textrm{Decompose}(Q_1,f,g,h,n-m))$
\RETURN $P_0 + X^mP_1$ 
\ENDIF
\end{algorithmic}
\end{algorithm}

\begin{thm}
Soient $Q,f,g,h,n$ avec $\deg(Q)<\ell n$ et $h=1/g \bmod f$ les entrées de l'algorithme \ref{alg:Decompose}, alors l'algorithme \ref{alg:Decompose} calcule un polynôme $P \in \mathbb{F}_q[X,Y]$ tel que $\deg(P,X)<n$, $\deg(P,Y)<\ell$ et $Q=P[f,g,n]$ avec une complexité de $O(M(\ell n) \log(n))$ opérations sur $\mathbb{F}_q$.
\end{thm}

\begin{proof}
Voir la preuve issue de \cite{DeFeo-Doliskani-Schost13}.
\end{proof}

\begin{cor}
Au $i$ème niveau de la tour, l'opération de descente coûte $O(M(\ell^i)\log(\ell^i))$ opérations sur $\mathbb{F}_q$.
\end{cor}

\begin{proof}
Soit $\alpha$ un élément de $\mathbb{F}_{q^{\ell^i}}$ alors $\alpha$ peut être représenté dans une base univariée sous la forme $A(Y)$ avec $A$ un polynôme de degré strictement inférieur à $\ell n $ avec $n=\ell^{i-1}$. On calcule tout d'abord le polynôme $g^{n-1}$ afin de pouvoir calculer $A^*=g^{n-1}A \bmod S$ le tout avec une complexité de $O(M(\ell^i))$ opérations. On calcule alors le polynôme $h=1/g \bmod f$ avec un coût de $O(M(\ell)\log(\ell))$ opérations sur $\mathbb{F}_q$ et l'on applique alors l'algorithme \ref{alg:Decompose} aux entrées: $A^*,f,g,h,n$. Celui-ci calcule alors un polynôme bivarié $B$ avec une complexité de $O(M(\ell^i\log(\ell^i))$ opérations sur $\mathbb{F}_q$, $B$ représente alors $\alpha$ dans la base bivariée.
\end{proof}


\subsubsection{Remarques sur les utilisations dans notre cas pratique}
On pourra remarquer que dans le cas qui nous intéresse en pratique on choisit comme corps de base pour la tour $\ell$-adique une extension $F_1$ de $\mathbb{F}_q$ dans laquelle on a $E[\ell] \subset E(F_1)$. Pour calculer une telle extension on calcule les ordres des valeurs propres $\lambda, \mu$ de $\pi$ modulo la $\ell$-torsion. On travaille alors avec l'extension $F_1$ de degré $d_1$ avec $d_1=ppcm(o(\lambda),o(\mu))=o(q)$ et $o$ l'ordre multiplicatif dans $\mathbb{Z}/\ell\mathbb{Z}^*$, $d_1$ est en particulier l'ordre de la matrice $\pi$ sur la $\ell$ torsion, il divise $\ell-1$. Une conséquence importante pour l'analyse des constructions possibles de tours $\ell$-adiques c'est que dans ce cas-là on aura $\ell \mid F_1-1$, qui est comme on a vu le cas le plus avantageux pour les opérations dans les tours d'extension. 

Ainsi pour le calcul d'opérations dans de telles tours $\ell$-adiques on devra ajouter un facteur $O(M(\ell)\log(\ell))$ aux résultats obtenus pour une tour $\ell$ adique construite sur un corps $F_1$ tel que $\ell \mid F_1-1$ pour avoir le coût des opérations dans $\mathbb{F}_q$.


\subsubsection{Opérations sur les tours $\ell$-adique}
Comme il va être expliqué plus tard on va construire une tour $\ell$-adique à partir d'une extension de corps  de$\mathbb{F}_q=F_0$ dénotée $F_1$ telle que l'on ait $d_1=[F_1:F_0]$ égal à l'ordre de $q$ dans $\mathbb{Z}/\ell \mathbb{Z}$, on aura en particulier $d_1 \mid \ell-1$. On fera remarquer qu'il peut très bien arriver que l'on ait $d_1=1$. De plus par définition de $d_1$, on a $\ell \mid q^{d_1}-1$ et donc on pourra utiliser les constructions de tour $\ell$-adique préconisées dans ce paragraphe \ref{par:easycase} .%cas le plus simple en l'occurence

\paragraph{Calcul du Frobenius}
Cette opération sera très importante par la suite pour déterminer des ensembles spécifiques de points d'une courbe elliptique (comme vu dans l'exmple sur les cycles d'isogénies), c'est pourquoi on s'attache ici à montrer que l'on peut effectuer cette opération à un coût raisonnable comparé au cas $\ell=2$ \cite{Doliskani-Schost15} dont le résultat ici, issu d'un travail conjoint avec Luca De Feo Jérome Plut et Eric Schost, s'inspire.

\begin{thm} \label{thm:frob-ell}
Soit $F_0 \subset F_1 \subset \cdots \subset F_n$ une tour de Kummer définie telle que $[F_{i+1}:F_i]=\ell$ pour $i \neq 0$ et $[F_1:F_0]=d_1$ avec $d_1$ égal à l'odre multiplicatif de $q=|F_0|$  dans $\mathbb{Z}/\ell \mathbb{Z}$ (voir \textcolor{red}{(réf à mettre plus tard)} pour plus de détail). Soit $a$ un élément de $F_i$ pour tout entier $j$ le calcul de la puissance $|F_j|$ de $a$ coûte $O(\ell^{i-1}M(\ell))$ opérations sur $\mathbb{F}_q$ après un pré calcul de $O(\ell M(\ell)\log(q))$
\end{thm} 

\begin{proof}
Sans perte de généralité on peut supposer que l'on a $j<i$, le résultat étant sinon $a$ lui-même. Notons $s= |F_j|$ et $d=[F_i:F_1]=\ell^{i-1}$. Soit $x_i$ l'image de $x$ dans $F_i=\mathbb{F}_q[x]/P_i(x)$, on a en particulier $x_i^d=x_1$.

La première étape est de calculer $y=x_i^s$. En écrivant $s=ud+r$ avec $0 \leqslant r < d$ on a alors $y=x_1^{u \bmod |F_1|-1}x_i^r$. Le calcul de $x_1^{u \bmod |F_1|-1}$ se fait en utilisant $O(\ell M(\ell) \log(q))$ opérations sur $\mathbb{F}_q$, on conserve ce résultat sous la forme d'un monôme dans $F_1[x_i]$. Par hypothèse on a $a$ qui est représenté sous la forme d'un polynôme en $x_i$ de degré inférieur ou égal à $[F_i:F_0]$: 
\begin{equation*}
a^*_0+a_1^*x_i+ \cdots + a^*_{d_1d-1}x_i^{d_1d-1}
\end{equation*}
avec les $a_i^* \in F_0$. On réécrit $a$ alors sous la forme:
\begin{equation*}
a_0+a_1x_i+ \cdots + a_{d-1}x_i^{d-1}
\end{equation*}
avec les $a_w=a_w^* + \sum_{k=1}^{d_1d-1/d}a_{i+dk}x_1^k \in F_1$ pour $w$ appartenant à $[0; d-1]$ en rappelant que pour $t$ appartenant à $[1; d_1d-1]$ on a $t=d u_t+v_t$ avec $0 \leqslant v_t<d$ par conséquent $x_w^t=x_1^{u \bmod |F_1|-1}x_w^{v}$ . Ainsi cette opération est juste un réarrangement des coefficients de $a$.

On a alors $a^s$ qui s'écrit sous la forme:
\begin{equation*}
a_0 + a_1 y + a_2 y^2 + \cdots + a_{d_1}y^{d-1}
\end{equation*}
On calcule alors $a(y)$ par la méthode de Horner. Les puissances $y^k$ étant des monômes dans $F_1[x_i]$, on calcule chacune des puissances successives à l'aide de la précédente en $O(M(\ell))$ opérations sur $\mathbb{F}_q$, on aura alors un coût total de $O(\ell^{i-1}M(\ell)))$. En réarrangeant les monômes $a_k y^k$ on obtient un polynôme en $(x_1,x_i)$ de degré $(d_1,d)$, on réécrit alors $a^s$ dans une base canonique de $F_i$ à l'aide du changement de représentation inverse de celui effectué précédemment. En effet on a $a^s$ représenté sous la forme:
\begin{equation*}
b_0 + b_1 x_i + b_2 x_i^2 + \cdots + b_{d-1}x_i^{d-1}
\end{equation*}
avec pour tout $k$, $b_k$ un élément de $F_1$. En remarquant que l'on a 
\begin{equation*}
 b_k=b_{k,0} + b_{k,1}x_1 + \cdots + b_{k,d_1-1} x_1^{d_1-1}=b_{k,0} + b_{k,1}x_i^d + b_{k,2} x_i^{2 \times d} + \cdots + b_{k,d_1-1} x_i^{(d_1-1)\times d} 
\end{equation*}
on aura alors $a^s$ représenté sous la forme:
\begin{equation*}
b_0^* + b_1^* x_i + b_2 x_i^2 + \cdots + b_{d_1d-1} x_i^{d_1d-1}
\end{equation*}
avec pour $k=ud+v$ avec $0 \leqslant v < d$ $b_k^*=b_{u,v}$. On a alors ici aussi un changement de représentation qui se fait à l'aide d'un réarrangement des coefficients.
\end{proof}

\chapter{Isogénies sur les corps finis}

\section{Calculs d'isogénie}
\subsection{Algorithme de Lercier Sirvent (+ resultat de Tristan)}
\subsection{Algorithme de Couveignes (parler de ses variantes)}

\subsection{Volcans d'isogénies}
\subsection{Travaux de Kohel, Fouquet-Morain}
Mettre l'algorithme de Fouquet Morain
\subsection{Travaux de Miret \& Al. et Ionica-Joux}


\section{Frobenius et ses applications sur un volcan d'isogénies}
\subsection{Base diagonale, horizontale}
\subsection{Orbites selon le Frobenius}
\subsection{Amélioration de l'algorithme de Couveignes}

\section{Améliorations éventuelles à l'aide du pairing}

\bibliographystyle{plain}
\bibliography{Biblio}
\end{document}
