\documentclass[10pt,a4paper]{article}
\usepackage[utf8]{inputenc}
\usepackage[T1]{fontenc}
\usepackage{hyperref}
%\usepackage{natbib}
\usepackage[francais]{babel}
\usepackage{hyperref}
\usepackage{amssymb,amsthm,amsmath,amsfonts}
\usepackage{algorithm}
\usepackage{algorithmic}
\usepackage{units}
\usepackage{lmodern}
\usepackage{fp}
\usepackage{graphicx}
\usepackage{epsfig}
\usepackage{epstopdf}
\usepackage{makeidx}
%\usepackage{gloss}
\usepackage{xcolor}
%\usepackage[toc]{glossaries}
\usepackage{changepage}
\usepackage{tikz}
\usepackage{array}
\usetikzlibrary{decorations}
\usetikzlibrary{decorations.pathreplacing}
\usetikzlibrary{calc}
\usetikzlibrary{fixedpointarithmetic}
\usepackage{todonotes}
\usepackage[intoc]{nomencl}
%\usepackage{fancyhdr}
%\pagestyle{fancy}
%\usepackage{apalike}
\usepackage{subfig}
\usepackage{pdfpages}



\theoremstyle{plain}
\newtheorem{thm}{Théorème}[section]
\theoremstyle{definition}
\newtheorem*{thm*}{Théorème}
\theoremstyle{definition} 
\newtheorem{lem}[thm]{Lemme}
\theoremstyle{definition} 
\newtheorem{cor}[thm]{Corollaire}
\theoremstyle{definition} 
\newtheorem{prop}[thm]{Proposition}
\theoremstyle{definition} 
\newtheorem{defi}[thm]{Définition}
\theoremstyle{remark} 
\newtheorem{rem}[thm]{Remarque}
\theoremstyle{remark} 
\newtheorem{exe}[thm]{Exemple}
\theoremstyle{definition}
\newtheorem{nota}[thm]{Notation}

\bibliographystyle{alpha}
\title{Remarks concerning the correction of David Kohel}
\begin{document}
\maketitle
\section{Introduction}
\paragraph{p.13}{Référence à l'article de Jacques Vélu de 1971 "Isogénies entre Courbes Elliptiques" $\Rightarrow$  celui-ci n'est pas pertinent par rapport au problème que l'on étudie car ici on ne connaît pas le noyau de l'isogénie alors que dans l'article de Vélu si, voilà pourquoi je n'ai pas cité celui-ci. \'A moins qu'il n'y ait un article de Vélu que j'ai manqué...}
\paragraph{p.13}{"En pratique ... au cas par cas" en fait ici je voulais faire référence au crosspoint entre l'algorithme de Couveignes et celui de Lercier-Sirvent dans le cas de la moyenne caractéristique. Les complexités de ces algorithmes sont explicitées dans le chapitre 2. Pour l'algorithme de Lercier-Sirvent il faut rajouter le coût de calcul du polynôme modulaire.}
\paragraph{p.14}{Concernant l'article de Couveignes "Hard homogeneous space" Luca m'avait parlé de celui-ci, je le rajoute.}

\paragraph{p.14}{Concernant "action d'isogénie" je l'ai remplacé par "évaluation de l'isogénie"}

\paragraph{p.14}{Concernant les références aux travaux de Pizer et Mestre j'ai rédigé cela de la façon suivante: "Les graphes d'isogénies ont été étudiés pour la première fois par 
\cite{Kohel96}, ses travaux dans le contexte particulier des courbes elliptiques
supersingulières ont été inspirés par ceux de \cite{Mestre86,Pizer90,Pizer95} 
qu'il a adapté et développé dans ce contexte, il a aussi étudié le cas des 
courbes elliptiques ordinaires, tout cela afin de calculer l'anneau des 
endomorphismes d'une courbe 
elliptique."}

\paragraph{p.14}{Concernant "a cherché à déterminer", j'ai changé de la manière suivante: "Kohel détermine le niveau d'une courbe dans le graphe tout en 
connaissant la cardinalité de celle-ci afin de calculer son anneau des 
endomorphismes."}

\paragraph{p.14}{Concernant le travail de Fouquet Morain je l'ai reformulé de la façon suivante: "Fouquet et Morain~\cite{FouquetMorain02} ont par la suite résolu le même 
problème, en adaptant les méthodes de Kohel sans la connaissance préalable de 
la cardinalité de la courbe, afin d'améliorer SEA, et ont nommé les graphes 
d'isogénies \emph{volcans d'isogénies}."}

\paragraph{p.14}{Concernant le travail de Lenstra et de son articulation avec votre thèse, je réfléchis encore à cela pour bien le comprendre avant d'en parler. J'ai du mal à voir notamment dans ce cas comment détecter un niveau de stabilité. J'arrive bien à voir que ainsi en bas du volcan on a un cratère, que la structure sur le cratère est forcément identique. Encore une fois je réfléchis encore à cela.}

\paragraph{p.14}{Concernant le théorème, j'ai bien enlevé la numérotation 
(trop) récemment, j'ai modifié l'énoncé comme suit :"Pour presque tous les 
nombres premiers $q$ et les courbes elliptiques ordinaires $E,E'$ définies sur
$\mathbb{F}_q$, il est possible de résoudre le problème du "Calcul explicite de
l'isogénie" en un temps quasi-quadratique en le degré de l'isogénie, 
avec une dépendance logarithmique en la caractéristique de $\mathbb{F}_q$." le 
fait que le degré de l'isogénie soit connue est pour moi énoncé dans le 
problème du "Calcul explicite de l'isogénie" je ne l'ai donc pas retranscrit à 
nouveau ici}

\paragraph{Question sur le théorème}{Notre algorithme proposé pour résoudre ce 
problème traite différemment deux cas distincts: le cas Atkin et le cas Elkies.
\begin{itemize}

\item {} Pour le cas Elkies nous avons un algorithme fonctionnel avec une complexité
satisfaisante (quasi-quadratique) avec une condition restrictive sur la hauteur
$h$ du volcan de $\ell$-isogénies, concrètement on doit avoir $h< O(\log_{\ell}
(4r))$ lorsque cette condition sur $h$ n'est pas vérifiée alors 
l'algorithme a une complexité supérieure à la complexité quasi-quadratique. 
\item  Concernant le cas Atkin ce n'est pas explicitement écrit dans ma thèse mais 
pour la cas où l'on a un entier de Atkin tel que la hauteur du volcan est 
nulle alors nous obtenons un algorithme quadratique avec une dépendance 
polynomiale en le paramètre $\ell$ moins bonne que dans le cas Elkies. 
\item[Existence de tels cas]\`A l'aide du résultat de 
\cite[Theorem 1]{ShparlinskiSutherland14} si l'on fixe une 
borne $L=O(\log(q))$ et que pour celle-ci nous n'avons que des nombres $\ell$ de 
Elkies avec une hauteur $h$ associée $>\log_{\ell}(4r)$ alors en étudiant 
$d_{\pi}$ borné par $O(\log(q))$, celui-ci($d_{\pi}$) ne peut admettre que des
nombres premiers ramifiés qui le divise et des nombres premiers de Atkin de 
hauteur $\neq 0$ car sinon ce produit de nombres premiers dépasserait largement
$O(\log(q))$ ainsi on doit avoir au moins l'existence d'un entier de Atkin de 
hauteur $0$  ou d'un entier de Elkies de hauteur $h<\log_{\ell}(4r)$. Donc on 
peut trouver un entier soit de Atkin de hauteur $0$ soit de Elkies de hauteur 
$h<\log_{\ell}(4r)$ borné par $O\log(q)$.

\end{itemize}
}

\section{Chapitre 4}

\paragraph{p 61}{Merci pour les remarques autour de $d_{K}$, j'introduis donc de suite la notation $d_{\ell}$ pour le discriminant de l'ordre maximal par rapport à $\ell$ noté $\mathcal{O}_{\ell}$ lui aussi dans la nomenclature}

\paragraph{p.63}{$f^2\frac{d_K (d_K-1)}{4}$ c'est fait.}

\paragraph{p.63}{Concernant le Lemme 6.14, il est vrai que ce résultat est trivial, je l'ai mis sous forme de remarque et j'ai formulé cela comme suit dans la thèse: 
"On fait alors la remarque suivante sur les duales des isogénies:
\begin{rem}
\label{lem:dua:vol}
Soit $\phi : E \mapsto E'$ une isogénie et sa duale $\widehat{\phi}$. Alors:
\begin{enumerate}
\item $\phi$ est une isogénie montante si et seulement si $\widehat{\phi}$ est une isogénie descendante,
\item $\phi$ est une isogénie horizontale si et seulement si $\widehat{\phi}$ est une isogénie horizontale.
\end{enumerate}
\end{rem}"}

\paragraph{p.64}{Merci pour les remarques concernant le graphe non orienté, 
j'ai donc reformulé une partie du paragraphe comme suit: "On a pris comme 
convention de représenter ici un graphe non orienté. En effet chaque arrête 
représente toutes les isogénies de noyaux identiques ainsi que leurs duales."}

\paragraph{p.65}{Pour ce qui est du choix de $D$ ou de $d_{K}$ dans le tableau,
je n'ai effectivement pas précisé que $D$ est le discriminant de l'ordre 
associé à l'anneau des endomorphismes de la courbe $E$, je vais donc mettre 
$d_K$ pour que cela soit plus clair pour le lecteur. Pour ce qui est du nombre 
exact d'isogénies en prenant en compte l'action des automorphismes de la 
courbe sur les noyaux des isogénies, j'ai préféré pour des raisons 
"esthétiques" de disqualifier le cas $j=0$ et $j=1728$ pour le tableau et de 
rajouter la phrase suivante après le tableau: "Pour $j=0$ ou $j=1728$ il faut 
prendre en compte l'action des automorphismes de
$E$ sur les sous-groupes cycliques de $E[\ell]$. Dans ce cas
 les valeurs dans la seconde colonne ne donnent que le nombre 
de représentants sous l'action du groupe des automorphismes de $E$."}

\paragraph{ en référence à la modification précédente}{J'ai donc modifié le théorème qui se trouvait juste avant le corollaire en bas de la p.64: "\begin{thm}
\label{thm:Koh:cas}
Soit $E$ une courbe elliptique ordinaire définie sur un corps fini 
$\mathbb{F}_q$ avec $j(E)\neq 0, 1728$ et soit $\ell$ un nombre premier 
différent de la caractéristique de $\mathbb{F}_q$
\begin{enumerate}
\item Si $\ell \nmid [\mathcal{O}_K : \mathcal{O}]$ alors $E$ admet $ 1 + \left( \frac{d_K}{\ell} \right)$ $\ell$-isogénies horizontales
\item Si $\ell \mid [\mathcal{O}_K : \mathcal{O}]$ alors $E$ admet une $\ell$-isogénie montante
\item Si $\ell \nmid [\mathcal{O} : \mathbb{Z}[\pi]]$ alors $E$ n'admet pas de $\ell$-isogénie descendante
\item Si $\ell \mid [\mathcal{O} : \mathbb{Z}[\pi]]$ et $\ell \nmid [\mathcal{O}_K : \mathcal{O}]$ alors $E$ admet $\ell-\left( \frac{d_K}{\ell} \right)$ $\ell$-isogénies descendantes
\item  Si $\ell \mid [\mathcal{O} : \mathbb{Z}[\pi]]$ et $\ell \mid [\mathcal{O}_K : \mathcal{O}]$ alors $E$ admet $\ell$ isogénies de degré $\ell$ descendantes.
\end{enumerate}
Si $j(E)=0$ ou $j(E)=1728$ alors il faut prendre en compte l'action des 
automorphismes de $E$ sur les sous-groupes cycliques de $E[\ell]$. Dans ce cas
les valeurs énoncées ne donnent que le nombre de représentants 
sous l'action du groupe des automorphismes de $E$.
\end{thm}"}

\paragraph{Définition 4.23}{Il ne doit y avoir dans le manuscrit que des $E_s$, j'ai donc corrigé cela. J'ai reformulé la définition comme suit:"
\begin{defi}
\label{def:haut:vol}
Pour une courbe sur le volcan son niveau est la longueur maximale d'un chemin descendant.
\newline
La hauteur $h$ d'un volcan est le niveau maximal des courbes dans le volcan.
\end{defi}"}

\paragraph{Remarque 4.24}{J'ai modifié la remarque comme suit:
"\begin{rem} 
\label{rem:lie:dp:dk}
Pour un volcan de $\ell$-isogénies de hauteur $h$ par les 
définitions~\ref{def:iso:nom} et ~\ref{def:haut:vol} $\ell^{2h} | d_{\pi}$.
En particulier pour $\ell \neq 2$ $\ell^{2h}||d_{\pi}$, ce résultat ne peut être 
énoncé  pour $\ell=2$ car il est possible que $4 \mid d_K$ par définition de $d_K$
(définition~\ref{def:dis:dK}).
\end{rem}"}

\paragraph{Algorithme de Fouquet-Morain}{J'ai ajouté la référence et j'ai 
enlevé le cherche à calculer: "L'algorithme de 
Fouquet-Morain~\cite{FouquetMorain02} est un algorithme qui 
calcule un chemin descendant d'une courbe elliptique située sur le cratère afin
de calculer la valuation $\ell$-adique du conducteur de $\mathbb{Z}[\pi]$. Cet
algorithme est une adaptation de l'algorithme "Probing the depths" de Kohel \cite[Section 4.2]{Kohel96} au contexte SEA."}

\paragraph{Citation de FM02}{Si je comprends bien il serait préférable que je cite uniquement FM00 lorsque cela est possible ?}

\paragraph{p.67-P.68: Proposition 4.\color{red}{2}\color{black}9 corollaire de 
4.27}{Je suis tout à fait d'accord avec cette remarque qui fait référence aussi
à celle de l'introduction. J'ai donc renommé cette proposition corollaire je 
l'ai placé après la proposition dont il est le corollaire et j'ai rédigé la 
preuve suivante:"\begin{proof}
Comme $E$ et $E'$ sont situées au même niveau alors leurs anneaux 
des endomorphismes sont isomorphes, par la proposition \ref{pro:len:str} on 
obtient le résultat.
\end{proof}"}

\paragraph{prop 4.30}{Je n'ai pas vu de problème de parenthèses}

\paragraph{prop 4.31}{$n_2 \mid n_1$ rajouté}

\paragraph{Attention}{Courbes elliptiques ordinaires ajoutés!}

\section{Chapitre 5}

\paragraph{ligne -1}{J'ai modifié comme suit: "
Dès lors puisque $\ell^{2h} \mid d_{\pi}$ et pour $\ell \neq 2$ $\ell^{2h}
|| d_{\pi}$, pour $\ell=2$ on peut avoir $2^{2h+1} \mid d_{\pi}$ lorsque $2 \mid d_K$ (voir remarques~\ref{rem:def:dpi} 
et ~\ref{rem:lie:dp:dk}) avoir $d_{\pi}$ un résidu quadratique est équivalent 
à avoir $\left( \frac{d_{K}}{\ell} \right)=1$. Ainsi dans le cas Elkies le 
cratère est cyclique par le corollaire~\ref{cor:tab:vol}.
"}

\paragraph{Proposition $5.\color{red}3$}{Concernant $\pi | \mathsf{T}_{\ell}(E)$ j'ai reformulé l'énoncé et la preuve comme suit:
"
\begin{prop}\label{pro:mat:fro}
Soit~$E$ une courbe elliptique ordinaire, telle que $\ell$ est un nombre Elkies
a deux racines distinctes $\lambda$, $\mu$ appartenant à $\mathbb{Z}_{\ell}$,
ainsi le volcan des $\ell$-isogénies de $E$ a un cratère cyclique.
Il existe un unique $e \in [ 0, h]$ tel que 
l'action du Frobenius dans toute base de $\mathsf{T}_{\ell}(E)$  
%$\pi|\mathsf{T}_{\ell}(E)$~
est conjuguée, sur~$\mathbb{Z}_{\ell}$,
à la matrice 
\begin{equation*}
\left ( \begin{matrix}\lambda & \ell^{h-e} \\ 0 & \mu
\end{matrix}\right ).
\end{equation*}
% where $a \in \mathbb{Z}$, $0 \leq a \leq \ell^{h} - 1$,
De plus $e = 0$ si~$E$ se situe sur le cratère,
sinon $h - e$~est le niveau de~$E$ dans le volcan, $e$ est appelé la \emph{profondeur}
de la courbe dans le volcan.
\end{prop}

\begin{rem}
$\left(\begin{smallmatrix} \lambda & \ell^h \\ 0 &
\mu \end{smallmatrix}\right)$ est conjuguée à $\left(\begin{smallmatrix} \lambda & 0 \\ 0 & \mu\end{smallmatrix}\right)$ sur~$\mathbb{Z}_{\ell}$.
\end{rem}

\begin{proof}
Comme le polynôme caractéristique de $\pi$ admet deux valeurs propres sur 
$\mathbb{Z}_{\ell}$ alors pour toute base de $\mathsf{T}_{\ell}(E)$ la matrice
représentant l'action du Frobenius dans cette base est trigonalisable. %\todo{justifier cela Morris Newman Integral Matrices III.\color{red}{15}}
La conjugaison de la matrice $\left ( \begin{smallmatrix}\lambda & a\\0 & \mu
\end{smallmatrix}\right )$ par~$\left ( \begin{smallmatrix}1 & b\\0 & 1
\end{smallmatrix} \right )$ change le coefficient $a$ en $a-b (\lambda - \mu)$, 
et la conjugaison par ~$\left(\begin{smallmatrix} c & 0 \\ 0 &
1\end{smallmatrix}\right)$ change~$a$ en~$c \cdot a$,
ainsi la valuation $\ell$-adique~$h-e = v_{\ell}(a)$ est invariante sous la 
conjugaison matricielle. Cela prouve la première partie de la proposition. 

Pour la seconde partie, par le théorème de 
Tate~\cite[Isogeny theorem III.7.7 (a)]{Silv1}, $\mathcal{O} \otimes 
 \mathbb{Z}_{\ell}$~est isomorphe à l'ordre dans~$\mathbb{Q}_{\ell}[\pi_{\ell}]$
des matrices avec coefficients entiers, cet ordre est engendré par la matrice 
 identité et la matrice $\ell^{-\min (h, v_\ell(a))} (\pi_{\ell}-\lambda)$, dés 
 lors l'indice de cet ordre par rapport à $\mathbb{Z}[\pi_{\ell}]$ est de 
$\ell^{\min (h, v_\ell(a))}$, lui donnant son niveau $\min (h, v_\ell(a))$ dans 
 le volcan des $\ell$-isogénies.
\end{proof}
"}

\paragraph{Proposition 5.4}{Pourquoi ne pas travailler avec une matrice avec 
$\delta = 1$? J'ai choisi de faire un énoncé le plus général possible pour une
application pratique par la suite. Cette proposition serait tout aussi juste 
avec $\delta = 1$. Est-ce nécessaire pour la compréhension du lecteur que je rédige cela avec $\delta = 1$ ?}

\bibliography{Biblio}
\end{document}
%  LocalWords:  isogeny morphisms Isogenies isogenies isogenous
%  LocalWords:  cardinality bijection Couveignes automorphism

% vim: ts=2:
%  LocalWords:  Frobenius endomorphism precomputation morphism
%  LocalWords:  subproduct factorizations
