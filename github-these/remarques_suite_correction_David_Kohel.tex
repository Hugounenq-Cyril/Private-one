\documentclass[10pt,a4paper]{memoir}
\usepackage[utf8]{inputenc}
\usepackage[T1]{fontenc}
\usepackage{hyperref}
%\usepackage{natbib}
\usepackage[francais]{babel}
\usepackage{hyperref}
\usepackage{amssymb,amsthm,amsmath,amsfonts}
\usepackage{algorithm}
\usepackage{algorithmic}
\usepackage{units}
\usepackage{lmodern}
\usepackage{fp}
\usepackage{graphicx}
\usepackage{epsfig}
\usepackage{epstopdf}
\usepackage{makeidx}
%\usepackage{gloss}
\usepackage{xcolor}
%\usepackage[toc]{glossaries}
\usepackage{changepage}
\usepackage{tikz}
\usepackage{array}
\usetikzlibrary{decorations}
\usetikzlibrary{decorations.pathreplacing}
\usetikzlibrary{calc}
\usetikzlibrary{fixedpointarithmetic}
\usepackage{todonotes}
\usepackage[intoc]{nomencl}
%\usepackage{fancyhdr}
%\pagestyle{fancy}
%\usepackage{apalike}
\usepackage{subfig}
\usepackage{pdfpages}



\theoremstyle{plain}
\newtheorem{thm}{Théorème}[chapter]
\theoremstyle{definition}
\newtheorem*{thm*}{Théorème}
\theoremstyle{definition} 
\newtheorem{lem}[thm]{Lemme}
\theoremstyle{definition} 
\newtheorem{cor}[thm]{Corollaire}
\theoremstyle{definition} 
\newtheorem{prop}[thm]{Proposition}
\theoremstyle{definition} 
\newtheorem{defi}[thm]{Définition}
\theoremstyle{remark} 
\newtheorem{rem}[thm]{Remarque}
\theoremstyle{remark} 
\newtheorem{exe}[thm]{Exemple}
\theoremstyle{definition}
\newtheorem{nota}[thm]{Notation}


\title{Remarks concerning the correction of David Kohel}
\begin{document}
\maketitle
\chapter{Introduction}
\paragraph{p.13}{Référence à l'article de Jacques Vélu de 1971 "Isogénies entre Courbes Elliptiques" $\Rightarrow$  celui-ci n'est pas pertinent par rapport au problème que l'on étudie car ici on ne connaît pas le noyau de l'isogénie alors que dans l'article de Vélu si, voilà pourquoi je n'ai pas cité celui-ci. \'A moins qu'il n'y ait un article de Vélu que j'ai manqué...}
\paragraph{p.13}{"En pratique ... au cas par cas" en fait ici je voulais faire référence au crosspoint entre l'algorithme de Couveignes et celui de Lercier-Sirvent dans le cas de la moyenne caractéristique. Les complexités de ces algorithmes sont explicitées dans le chapitre 2. Pour l'algorithme de Lercier-Sirvent il faut rajouter le coût de calcul du polynôme modulaire.}
\paragraph{p.14}{Concernant l'article de Couveignes "Hard homogeneous space" Luca m'avait parlé de celui-ci, je le rajoute.}

\paragraph{p.14}{action d'isogénie...}
\end{document}
%  LocalWords:  isogeny morphisms Isogenies isogenies isogenous
%  LocalWords:  cardinality bijection Couveignes automorphism

% vim: ts=2:
%  LocalWords:  Frobenius endomorphism precomputation morphism
%  LocalWords:  subproduct factorizations
